\subsubsection{Energy recovery}

%%%%%%%%%%%%%%%%%% NOTATION %%%%%%%%%%%%%%%%%%%%%%%%%%
\newcommand{\setEnergyDeltaIdle}[2]{\Delta\setSymbol{E}{\texttt{idle}}{#2}}
\newcommand{\setEnergyHarvest}[2]{\setSymbol{E}{\texttt{harvest}}{#2}}
\newcommand{\setEnergyWUR}[2]{\setSymbol{E}{\texttt{idle-wur}}{#2}}
%%%%%%%%%%%%%%%%%%%%%%%%%%%%%%%%%%%%%%%%%%%%%%%%%%%%%%

When not sampling, aggregating, or broadcasting a node recovers energy at a rate given by the energy harvesting solar panel minus that lost by having the WUR in standby.

\begin{equation}
	\setEnergyDeltaIdle{}{}
	= 
	\setEnergyHarvest{}{}
	-
	\setEnergyWUR{}{}
\end{equation}


\begin{table}
	\begin{tabular}{p{0.1\textwidth}p{0.4\textwidth} p{0.4\textwidth}}
		\hline
		\textbf{Criteria} & \textbf{Description} & \textbf{Agent-system element} \\
		\hline
		Low energy use & Remote location devices can only recover energy slowly through solar panel charging &  Use link energy consumption values to optimse learning towards efficient neighbourhoods \\
		Resilience to node failure & Remote and dangerous locations mean human maintenance is either impossible, expensive, or otherwise restricted & Neighbourhood link adaptation algorithms \\	
		Power consumption distribution & Devices have independent energy recovery mechanisms and would be overwhelmed with too much load. Although other agents would then take over this leads to unstable networks & Link energy consumption value rate accumulation for agents can be used to shape rewards to drive agent local neighbourhood adaptation away from overloaded agents  \\ 
		\hline
	\end{tabular}
	\caption{Key criteria for E-WSN systems}
	\label{table:real_world_systems_criteria}
\end{table}
