\subsection{Tasks and resources}
\label{section:tasks}

\reviewquestion{You say "We define the agent-based system..." but to clarify I think you mean "... at an instant of time"? That is, it is a state of the system rather than the persistent system structure? Otherwise, ra couldn't change for example. Alternatively, you could separately define a 'system' as the definition of a system structure containing all the constant elements of your current definition and a time-dependent 'system state' containing the dynamic elements.
}
\paragraph{Distribution}
In our assumed system there is a geographical area to monitor on which a set of agents controlled by agents $\setAgents{}{}$ are distributed randomly, such as occurs in an aerial deployment \citep{Kumar2013}.  The area is defined by a two dimensional \textit{ location grid} of real numbers of unit length  $(\setRealNumbersUnit{}{} \times \setRealNumbersUnit{}{})$, onto which the \textit{deployment} maps each agent to a distinct location.  

\paragraph{Tasks}
Agents controlling these agents can perform \textit{atomic tasks}, individual sensor measurements, which are typed. Each atomic task targets a sensor measurement at a location on the distribution, the tasks' \textit{demand point}. Some agents will have sets of atomic tasks called \textit{composite tasks} allocated to them from an external source, throughout the systems' lifetime. We term these agents \textit{sinks} (see Section \ref{XXX}). Composite tasks are decomposed into atomic tasks by sinks, with each atomic task then being either executed by the sink, or allocated to other agents to complete or relay further.

\reviewquestion{Paragraph 1: You say that completing atomic tasks requires resources, but you do not mention whether communication between agents requires resources. This feels like an obvious reviewer question: either say that it does or say why it is OK to not include that in the model.}
\reviewquestion{When you say "Completing atomic tasks requires resources, such as energy, of which a node has a fixed amount" it is ambiguous what you mean by 'fixed amount'. As a reader I would naturally assume, given your discussion of irreplaceable batteries and energy usage in Section 2, that each node has 'a given initial amount' which reduces as it performs each task - but does 'fixed' then mean that every node has the same initial amount? Or you could mean that it always has the same amount because whenever it uses energy it is replaced somehow - so each node effectively has infinite energy but only a certain amount to pump into something at any one instant. Either way, you need to clarify.}
\paragraph{Resources}
In our system we focus on three resources,
\begin{itemize}
	\item \textit{compute resources}, used in activating a sensor and taking a reading. Each agent has the same, fixed, amount of compute resource, which it must share amongst its current tasks. The less allocated to a task, the lower the quality of its' completion. 
	\item \textit{memory resources}, this restricts the amount of knowledge an agent can have at any one time. Each agent has the same, fixed, amount of this resource.
	\item \textit{energy resources}, which are required to transmit and receive task requests, results, and knowledge between agents. Each agent has the same, fixed, maximum amount of energy in its battery at system initialisation. Each action uses some of this energy, which is replenished by a solar panel. Over time, the maximum energy storage decreases as the battery wears and degrades.
\end{itemize}

\reviewquestionopen{In the system definition, ra is defined as mapping each 'type of atomic task' to a resource amount, but in the formalisation it mapping from AT, which is defined as a 'set of atomic tasks' not of 'set of types of atomic tasks'. It may be just a language issue, but it is confusing and needs clarifying.
}
\reviewquestionopen{Do you need to set SG? You can seemingly define function sg as mapping to G rather than SG. As 'sink node' is just a role played for an atomic task (as said in Section 3.3) rather than a special type of agent, it seems inconsistent to define a special set for them.
	
	There is an issue that the system definition is both complex, having 10 terms in the tuple, and apparently incomplete because lots of things are mentioned in the following sections that appear to be part of the system but are not in the definition: links, information, transmission energy, etc. Why, for example, is conf part of the system definition but $e_trans$ is not? I wonder if you could present things differently:
	- A system is defined by the fixed sets and constants that are independent of any given agent or task: $(G, AT, CT, R, e_trans, ...)$
	- Everything else is presented separately from the system definition as a function returning some property of an agent, task or resource, i.e. you separately define each of ar, ra, sg, conf, sink, sensor, etc. That is, you are presenting as if the values returned by these functions are already present in the system by being attributes of the function's inputs (which are agents/tasks/resources in a given system) but the functions themselves are independent of any given system.}

\subsection{System definition}
We define the agent-based system as the tuple, $\langle 
	\setAtomicTask{}{},
	\setCompositeTask{}{},
	\setResource{}{},
	\setAgents{}{}
\rangle$, where
\begin{itemize}
	\item $\setAtomicTask{}{}$ is a set of atomic tasks where each task is a measurement task performed by a single agent;
	\item $\setCompositeTask{}{} \subseteq \powerSetSymbol{\setAtomicTask{}{}}{}{}$ is the set of composite tasks that occur in the system;
	\item $\setResource{}{}$ is a set of resources needed to perform atomic tasks;
	\item $\setAgents{}{}$ is a set of agents in the system, each agent $\varAgent{}{}$ being defined by a tuple $\tupleAgent{}{}$ where;
	\begin{itemize}
		\item $\varAgentCapability{}{}\subseteq \setAtomicTaskType{}{}$ is the agent capabilities; i.e., the atomic task types that the agent can perform;
		 \item $\varAgentResponsiblity{}{} \subseteq \setCompositeTaskType{}{}$ is the agent responsibilities; i.e., the composite task types that t    he agent can oversee;
		\item $\varAgentNeighbourhoodConstraint{}{}, \varAgentKnowledgeConstraint{}{} \in \mathbb{N}$, are the resource constraints of the agent, namely the communication and memory constraints (i.e., how many other agents a given agent can communicate with and know about).
	\end{itemize}
\end{itemize}

%%%%%%%%%%%%%%%%%%%%%%%%%%%%%%%%%%%%



\newcommand{\formalAgentResources}[2]{
	\functionFormal{agentres}
	{\setAgents{}{} \times \setResource{}{}}
	{\setRealNumbersNonNegative{}{}}
}
\newcommand{\functionAgentResources}[2]{
	\functionSignature{agentres}{\varAgent{}{}, \varResource{}{}}
}
The \textit{agent resources} function maps each agent and each resource to the amount of that resource that the agent can utilise to complete tasks, $\formalAgentResources{}{}$.

\newcommand{\formalTaskResourceAllocation}[2]{
	\functionFormal{taskres}
	{\setAgents{}{} \times \setAtomicTask{}{}}
	{\powerSetSymbol{\setResource{}{} \times \setRealNumbersNonNegative{}  {}}{}{}}
}
\newcommand{\functionTaskResourceAllocation}[2]{
	\functionSignature{taskres}
	{\varAgent{}{}, \varAtomicTask{}{}}
}
The \text{task resource allocation} maps each agent and type of atomic task, to a value representing the amount of a resource it has assigned to completing tasks of that type, $\formalTaskResourceAllocation{}{}$.

\newcommand{\formalTaskResponsibilities}[2]{
	\functionFormal{resp}
	{\setCompositeTask{}{} \times \powerSetAgent{}{}{}{}}
	({\setCompositeTask{}{} \times \setAgent{}{}})
}
\newcommand{\functionTaskResponsibilities}[2]{
	\functionSignature{resp}
	{\varCompositeTask{}{}, \setAgent{}{}}
}
\textit{Task-responsibilities} is a mapping from each composite task type to the group of agents, with the corresponding responsibilities, that will receive those tasks types and ensure their completion. This mapping is fixed on system initialisation; $\formalTaskResponsibilities{}{}$.

\newcommand{\formalDeployment}[2]{\functionFormal{deploy}{\setAgents{}{}}{(\setRealNumbersNonNegative{}{} \times \setRealNumbersNonNegative{}{})}}
\newcommand{\functionDeployment}[2]{
	\ifx \\1\\
	\functionSignature{deploy}{\setAgents{}{}}
	\else
	\functionSignature{deploy}{#1}
	\fi
}
The \textit{deployment} is a mapping of each agent to its location in the system; $\formalDeployment{}{}$

\newcommand{\formalTaskDemandPoint}[2]{
	\functionFormal{demand}
	{\setAtomicTask  {}{}}{\tupleLocation{}{}}
}
\newcommand{\functionTaskDemandPoint}[2]{\functionSignature{demand}{\varAtomicTask{}{}}}
The \text{task demand points} maps atomic tasks to their respective demand points; $\formalTaskDemandPoint{}{}$ 

%%%%%%%%%%%%%%%%%
\newcommand{\formalTransmissionRange}[2]{
	\functionFormal{range}
	{\powerSetAgents{}{} \times (\setRealNumbersNonNegative{}{} \times \setRealNumbersNonNegative{}{})}
	{\setRealNumbersNonNegative{}{}}
}
\newcommand{\functionTransmissionRange}[2]{\functionSignature{range}{\setAgents{}{}}}

A agents' hardware has a maximum distance it can effectively communicate with other agents \cite{Radman201}. This range can vary between agents as they can broadcast further, but at the expense of using more power., \cite{Padmanabh2008, Song2009}. The effective range is also effected by conditions and obstructions close to the agents location. We define the agents' \textit{transmission range} by $\formalTransmissionRange{}{}$.  
