%%%%%%%%%%%%%%%%%%%%%%%%%%%%%%%%%%%%
\newcommand{\setSinkAgents}[2]{\setSymbol{SG}{#1}{#2}}
\newcommand{\formalAgentResources}[2]{
	\functionFormal{ar}
	{\setAgents{}{} \times \setRealNumbers{}{}}
	{\setRealNumbersNonNegative{}{}}
}
\newcommand{\functionAgentResources}[2]{
	\functionSignature{ar}{\varAgent{}{}, \varResource{}{}}
}
\newcommand{\formalSinkMapping}[2]{
	\functionFormal{sg}
	{\setCompositeTask{}{}}
	{\powerSetSymbol{\setSinkAgents{#1}{#2}}{}{}}
}



\begin{figure}
\centering 
\includegraphics[width=0.9\linewidth, trim={25pt 0pt 24pt 0pt, clip}]{grid_concept}
\caption[WSN deployment terminology]{WSN components and terminology}
\label{fig:grid_concept}
\end{figure}

\subsection{Tasks and resources}

In our assumed system there is a geographical area to monitor defined by a two dimensional grid of real numbers on which a set of agents $\setAgents{}{}$ are distributed randomly. The \textit{deployment configuration} is a mapping from agents to their respective locations on this grid, $\formalDeployment{}{}$. Agents perform tasks, which are typed. These can be either \textit{atomic tasks}, for individual measurements, or \textit{composite tasks}, composed of sets of atomic tasks. Each atomic task targets a measurement at a location on this grid, its \textit{demand point} $\functionTaskDemandPoint{}{}$.

Composite tasks are allocated from outside the system to sink nodes, throughout the systems lifetime. These composite tasks are then decomposed by agents and the corresponding atomic tasks either completed by that agent, or allocated to other agents to complete or relay to further agents. Completing atomic tasks requires \textit{resources}, such as energy, of which an agent has a fixed amount defined by $\formalAgentResources{}{}$. At at time $\varTime{}{}$, each agent carrying out an atomic task has a certain amount of the resources it possesses $\varResource{}{}$, assigned to completion of tasks of that type, given by $\formalTaskResourceAllocation{}{}$

We can therefore define the system as a tuple $\langle \setAtomicTask{}{}, \setCompositeTask{}{},  \setAgents{}{}, \setResource{}{}, ar, sg, conf \rangle$, where
\begin{itemize}
	\item $\setAgents{}{}$ is a set of agents in the system that can complete tasks;
	\item $\setSinkAgents{}{} \subset \setAgents{}{}$ is a set of sink nodes, agents that can receive composite tasks from external to the system;
	\item $\setAtomicTask{}{}$ is a set of atomic tasks where each task is a measurement task performed by a single agent;
	\item $\setCompositeTask{}{} \subseteq \powerSetSymbol{\setAtomicTask{}{}}{}{}$ is the set of composite tasks that occur in the system;
	\item $\setResource{}{}$ is a set of resources needed to perform atomic tasks;
	  \item $\formalAgentResources{}{}$ is a mapping from each agent and each resource to the amount of that resource that the agent possesses.
	\item $\formalSinkMapping{}{}$ is a mapping from each composite task type to the group of sink agents that receive and ensure the completion of tasks of that type.
	\item $\formalDeployment{}{}$: is a mapping of each agent to their location in the system.
\end{itemize}

\subsection{Node roles and task arcs}
%%%%%%%%%%%%%%%%%%%%%%%%%%%%%%%%%%%%%%%%%%%%
\newcommand{\formalSinkRole}[2]{
	\functionFormal{sink}
	{\setAtomicTask{}{}}
	{\setAgents{}{}}
}
\newcommand{\formalSenseRole}[2]{
	\functionFormal{sensor}
	{\setAtomicTask{}{}}
	{\setAgents{}{}}
}
\newcommand{\formalActiveRole}[2]{
	\functionFormal{active}
	{\setAtomicTask{}{}}
	{\powerSetAgents{}{}}
}
\newcommand{\formalIdleRole}[2]{
	\functionFormal{idle_{\setTime{}{}}}
	{\setAtomicTask{}{}}
	{\powerSetAgents{}{}}
}
\newcommand{\formalSleepRole}[2]{
	\functionFormal{sink_{\setTime{}{}}}
	{\setAtomicTask{}{}}
	{\powerSetAgents{}{}}
}
\newcommand{\functionSinkRole}[2]{\functionSignature{sink}{\varAtomicTask{}{}}}
	
\newcommand{\functionSenseRole}[2]{\functionSignature{sensor}{\varAtomicTask{}{}}}
\newcommand{\functionActiveRole}[2]{\functionSignature{active_{#1}}{\varAtomicTask{}{}}}
\newcommand{\functionIdleRole}[2]{\functionSignature{idle}{\varAtomicTask{}{}}}
\newcommand{\functionSleepRole}[2]{\functionSignature{sleep}{\varAtomicTask{}{}}}

To simplify discussion of task execution we distinguish agents by the role they play in a given atomic task. These roles will also define the energy used by the nodes in executing a composite task and component atomic tasks \citep{Gupta2014}.
\begin{itemize}
	\item A \textit{sink node} of an atomic task $\varAtomicTask{}{}$ is the agent that first receives the corresponding composite task, and will broadcast the results, $\formalSinkRole{}{}$.
	\item A \textit{sensing node} is the agent that executes the atomic task and so performs the sensor measurement, $\formalSenseRole{}{}$.
	\item An \textit{active node} is an agent that participates in sub-allocating, or routing, that task, but is neither a sink agent nor a sensing agent, $\formalActiveRole{}{}$..
	\item An \textit{idle node} does not participate in the specific task, but does in other tasks in the system during a time period $\setTime{}{}$, $\formalIdleRole{}{}$.
	\item An \textit{sleeping node} does not participate in any of the tasks in the system during a time period $\setTime{}{}$, $\formalSleepRole{}{}$.
\end{itemize}
With these roles in mind, we can now define the  \textit{ task arc} as a mapping of atomic tasks to ordered sequence of agents $\formalTaskArc{}{}$ that each atomic task $\varAtomicTask{}{}$ is sub-allocated to. The first agent is the sink node that has received the initial composite task, and the last agent is the sensing agent that executes the atomic task, with the sequence of agents in-between relaying the atomic task. So that, for an arc of length $n$, we have
$\functionTaskArc{}{} = \lbrace \functionSinkRole{}{}, \functionActiveRole{i}{}, \functionSenseRole{}{} \rbrace_{i=1}^{n-1}$. 
