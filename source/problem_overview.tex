\subsection{Overview}
\label{problem_overview}
A WSN system is comprised of a set of interconnected hardware \textit{nodes}. Each node is equipped with a microcontroller for computation, memory storage, a battery for power storage, a solar panel for recharging the battery, a wireless transceiver for  transmitting and receiving messages from other nodes, and one or more \textit{sensors} for sensing and measuring some property of the environment such as temperature or radiation levels \citep{muhammad_r_ahmed_2012_1072589}. Each node has one \textit{agent}, a software controller that instructs its actions. For simplicity, we will use the term 'agent' to refer to both the software controller and the hardware node it controls. 

Agents may be deployed to precise locations or through a more random distribution method. Their placement is measured in comparison to a \textit{location grid}, a 2d co-ordinate system overlaid over the systems' environment (although this is also applicable to 3 dimensional environments).  The network structure formed may be flat or hierarchical, with or without clustering, dependent on the choice of protocol used \citep{Carlos-Mancilla2016b}. 

Agents will have to use \textit{resources} such as computation power, memory storage, and energy from their batteries to execute tasks, transmit messages to other agents, and discover new agents in the network. At a given point in time, there is a fixed amount of computational or memory resources, which must be shared. Battery power has a maximum capacity which is depleted when the agent takes actions, but is slowly replenished over time through charging by a solar panel.  

Agents in the system carry out \textit{atomic tasks}, which specify taking a measurement at a specific location. The further away the sensor is from this point, the less value the task has to the system.

At a high-level, the flow of task execution is as follows,
\begin{enumerate}
	\item An agent receives a tasks from an external agent
	\item The agent executes or allocates these tasks to other agents in the system, which may also re-allocate them.
	\item Agents' execute tasks by taking measurements using its sensors.
	\item The task results are transmitted back to the agent that allocated the task.
	\item The results are re-transmitted back to allocating agents until they reach the initial agent.
	\item The initial agent aggregates all the task results and transmits them to the external agent.
\end{enumerate}
In completing tasks, agents will be able to choose from a number of different actions, and assume different roles in executing each individual task. We look at roles and actions in Sections \ref{section:roles} and \ref{section:actions}. In Section \ref{section:energy_consumption} we describe how energy is consumed by the system. Next, in Section \ref{section:task_quality}, we define to what quality a measurement is completed, and how this, energy availability, and energy distribution in the system, effects the quality of overall composite task completion. Finally, we detail task coverage in Section \ref{section:coverage}, and how this relates to the resilience of the network, and the effective lifetime of the system, giving us the problem definition in Section \ref{section:optimisation_problem}.

\begin{example}
	\todo[inline]{XXX}
\end{example}

