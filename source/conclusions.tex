\section{Conclusions and future work}
\label{section:conclusions}

This work detailed and evaluated the novel \acronymWSNOptimisation{}{} algorithm, a combination of the previously described \acronymATARIA{}{} and \acronymMGRAO{}{} algorithms, with extension to hierarchical multi-agent systems. In particular, in their application to wireless sensor network optimisation in dynamic and challenging environments. The algorithm targeted optimisation of the energy available in the system, increasing system lifetime through through task and energy distribution, and the quality of task completion. The algorithm was evaluated on a model WSN system based on a realistic situation where nodes would be randomly dispersed into a geographical area, where maintenance and management is challenging due to harsh or dangerous conditions. The utility of the system was then optimised for sensors taking multiple measurements within the environment, where the quality of the measurement is dependent on the sensors distance to the demand point of the measurement task. Our evaluation showed that the \acronymWSNOptimisation{}{} algorithm optimised the task quality, energy available, and distribution in the system as describe in Section \ref{section:experimental}, and that these components could be varied in their priorities through altering the $\alpha$, $\beta$ and $\gamma$ values of the CTV function, Equation \ref{eq:ctv}. This allowed the algorithm to balance across these different properties in the given systems and optimise for these multiple objectives in different ratios. 

Although the evaluation of the algorithm has been shown in a realistic scenario, the systems represented are still relatively simplistic, and small scale. Future work would look at implementing the algorithm in a larger scale network through simulation, as well as in the real-world, testing how the algorithm performs in a complex environment. There are a number of applications in WSN in which the agents involved are mobile. As both the \acronymATARIA{}{} and \acronymMGRAO{}{} algorithms are designed to handle dynamic environments, where optimisation targets are non-stationary, we expect that the \acronymWSNOptimisation{}{} algorithm will also be able to applicable in these systems. As such evaluation could be extend to simulations with mobile nodes, and tested in real-world vehicle-to-vehicle communications (V2X) systems \citep{Gupta2017, Tong2019}.


  