\section{Conclusions and future work}
\label{section:conclusions}

This work detailed and evaluated the \acronymWSNOptimisation{}{} algorithm and its application to wireless sensor network optimisation in dynamic and challenging environments. This is an extension to hierarchical multi-agent systems of the previously described \acronymATARIA{}{} and \acronymMGRAO{}{} algorithms. The algorithm was shown to optimise the and the quality of task completion and energy available in these systems, and to increase system lifetime through task and energy distribution. The algorithm was evaluated on a model WSN system based on a realistic situation where agents would be randomly distributed across a geographical area, where maintenance and management would be challenging due to harsh or dangerous conditions.  Our evaluation showed that the \acronymWSNOptimisation{}{} algorithm optimised the task quality, energy available, and distribution in the system as describe in Section \ref{section:experimental}, and that these components could be varied in their priorities through altering the $\alpha$, $\beta$ and $\gamma$ values of the CTV function (Eq. \ref{eq:taq}). This allowed the algorithm to balance across these different properties in the given systems and optimise for these multiple objectives in different ratios. 

Future work would look at implementing the algorithm in a larger scale network through simulation, as well as in the real-world, testing how the algorithm performs in a complex environment. There are a number of applications in WSN in which the agents involved are mobile. As the \acronymWSNOptimisation{}{} algorithm is designed to work in dynamic environments, where optimisation targets are non-stationary, we expect that it will also be useful in optimising these types of system. As such evaluation could be extend to simulations with mobile agents, and tested in real-world vehicle-to-vehicle communications (V2X) systems \citep{Gupta2017, Tong2019}. We also expect that testing this work in the case of oceanographic monitoring would be a productive next step \citep{Albaladejo2010a}. The combination of harsh environmental conditions, difficulty of providing maintenance for remote agents, and mobility at slow speeds, should provide ideal conditions for successful practical use of \acronymWSNOptimisation{}{}.


  