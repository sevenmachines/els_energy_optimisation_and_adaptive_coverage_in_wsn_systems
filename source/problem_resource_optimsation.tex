We make the assumption that resource usage can be categorised by the following criteria;
\begin{enumerate}
	\item \textit{operational}, resource usage resulting from an agents' general functionality that are not part of a task execution. e.g. energy usage such as idle and sleep cycle functions, the computational and memory storage costs of background operations. 
	
	\item \textit{significant}, often there is a dominant use of a resource that makes other uses insignificant in comparison. e.g. when completing tasks, computational resources are used activate and coordinate internal component use. We assume the energy costs of sensor use and transmission are large in comparison adn so can disregard the effects of these other uses.
	
	\item \textit{optimisable}, whether or not the resource usage can be actively optimised by the agent. Agents' can shorten their transmission range, meaning less energy is used for broadcasting. By reducing the sampling time of a sensor, they can reduce the computation and energy usage. This makes these actions optimisable, and strategies can be learned.
\end{enumerate} 

In this work we focus on the cost of task completions, and disregard operational factors. Other work covers this area such as on battery duty-cycling, transmission algorithms, and solar-energy harvesting optimisation \citep{Kumar2010,Pinto2012,Matin2012,Escolar2014, Sharma2018}. 

With the justifications above on significance of the resource usage and optimisability, we model the relationships between an agents' actions and resultant resource usage as follows. Agents in the system can perform \textit{actions}. Each action $\varAction{}{}$ is mapped by the function $\functionAgentActionType{}{}$ to one of the action-categories below;

\begin{enumerate}
	\item $\functionExec{}{}$ - the agent will perform a \textit{task execution} of an atomic task $\varAtomicTask{}{}$ itself, using its computational resources. Computational resource usage is adaptable by an agent executing a task. The more of this resource it can allocate to a task, the higher quality that task will be performed to.

	\item $\functionAlloc{}{}$ - the agent will attempt \textit{task allocation} of an atomic task $\varAtomicTask{}{}$ to another agent $\varAgent{}{}$. This will use energy resources to transmit tasks and receive results. Agents that are transmitting messages can reduce energy usage by preferring nearby agents. This will either make the resulting sensor measurement at a greater distance from its demand point, or require relaying the task through more agents.

	\item  $\functionInfo{}{}$ - the agent will make an \textit{information request} to acquire knowledge from another agent $\varAgent{}{}$. This will require energy use for transmit and receive messages, and use memory storage for the results. 
	\item $\functionLink{}{}$ -  an agent carrying out a \textit{connection action} will allocate memory storage to hold knowledge on an agent $\varAgent{}{}$.  Agents will use their memory resources to maintain knowledge and neighbourhood information, but with a fixed maximum.
\end{enumerate}

%%%%%%%%%%%%%%%%%%%%
\newcommand{\formalTransmissionEnergy}[2]{
	\functionFormal{energy_{\symbolTransmission{}{}}}
	{\setAgent{}{} \times \setAgent{}{}}
	{\setRealNumbersPositive{}{}}
}
\newcommand{\functionTransmissionEnergy}[2]{
	\functionSignature{energy_{\symbolTransmission{}{}}}
	{\varAgent{#1}{},\varAgent{#2}{}}
}
\newcommand{\functionTransmissionEnergyIndexed}[2]{
	\functionTransmissionEnergy{\varAgent{i}{}}{\varAgent{i+1}{}}
}
In calculating the energy involved in completing tasks we make the following assumptions;
\begin{itemize}
	\item \textit{transmission energy}, the energy used by nodes to send a message. We assume that there is a linear relationship between the distance separating agents and the energy cost\footnote{A more complex relationship could be modelled with no expected impact on the effectiveness of our algorithms}. So that, in transmitting a message from $\varAgent{1}{}$ to $\varAgent{2}{}$, there is a constant $\varTransmissionEnergy{}{}$ such that;
	\begin{equation}
		\functionTransmissionEnergy{1}{2}
		=  \varTransmissionEnergy{}{}\funcSize{\functionDeployment{\varAgent{2}{}}{} - \functionDeployment{\varAgent{1}{}}{}}
	\end{equation}
	\item \textit{receiver energy}, the energy used by a node to receive a message. We assume the energy use for receiving a message in not dependent on the distance, and ignore the impact of message length, to give a constant value $\varReceiverEnergy{}{}$, 
	\item \textit{sensor energy}, the energy used by a node to execute an atomic measurement task. We make the common assumption that sensor activation is the dominant energy cost when taking measurements, with repeated samples being less significant \citep{Razzaque2014}. This means that the overall energy use of a sensor-type in the system while executing a task approximately constant, $\varSensorEnergy{}{}$. However, this assumption can be less valid depending on the type of sensor, and how its instruments take samples.
\end{itemize}
