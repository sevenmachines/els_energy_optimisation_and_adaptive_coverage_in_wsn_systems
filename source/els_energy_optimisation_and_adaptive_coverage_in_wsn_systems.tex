%% 
%% Copyright 2019-2020 Elsevier Ltd
%% 
%% This file is part of the 'CAS Bundle'.
%% --------------------------------------
%% 
%% It may be distributed under the conditions of the LaTeX Project Public
%% License, either version 1.2 of this license or (at your option) any
%% later version.  The latest version of this license is in
%%    http://www.latex-project.org/lppl.txt
%% and version 1.2 or later is part of all distributions of LaTeX
%% version 1999/12/01 or later.
%% 
%% The list of all files belonging to the 'CAS Bundle' is
%% given in the file `manifest.txt'.
%% 
%% Template article for cas-sc documentclass for 
%% single column output.

%\documentclass[a4paper,fleqn,longmktitle]{cas-sc}
\documentclass[a4paper,fleqn]{cas-sc}

%\usepackage[numbers]{natbib}
%\usepackage[authoryear]{natbib}
\usepackage{natbib}
\usepackage[textsize=small]{todonotes}
\usepackage{subcaption}
\usepackage{float}
\usepackage[ruled,linesnumbered]{algorithm2e}
\usepackage{graphicx}
\usepackage[]{caption}
\usepackage{subcaption}
\usepackage{placeins}

%%%%%%%%%%%%%% 	SYMBOLS		%%%%%%%%%%%%%%%%%
%                                    		%
%	Define all abstract set, variable,		%
%	function, and other generic symbols   	%
%                               			% %                              				%
%%%%%%%%%%%%%%%%%%%%%%%%%%%%%%%%%%%%%%%%%%%%%
\newtheorem{ass}{Assumption}
\newcommand{\assumption}[2]{
	\begin{ass}{(\textit{#1})}
		#2
	\end{ass}
}

\newcommand{\probability}[1]{
	P(#1)
}
\newcommand{\varSymbol}[3]{
\ifx \\#2\\
	\ifx \\#3\\ \lowercase{#1}
	\else
	\lowercase{#1}^{#3}
	\fi
\else
	\ifx \\#3\\ \lowercase{#1}_{#2}
	\else
		\lowercase{#1}_{#2}^{#3}
	\fi
\fi
}

\newcommand{\optimalVarSymbol}[3]
{
	\ifx\\#2\\
	\ifx\\#3\\ \optimal{\lowercase{#1}}
	\else
	\optimal{\lowercase{#1}}\textsuperscript{#3}
	\fi
	\else
	\ifx\\#3\\ \optimal{\lowercase{#1}}\textsubscript{#2}
	\else
	\optimal{\lowercase{#1}}\textsubscript{#2}\textsuperscript{#3}
	\fi
	\fi
}

\newcommand{\specialSetSymbol}[3]
{
	\ifx\\#2\\ \ifx\\#3\\  \overline{\uppercase{#1}}
	\else
	 \overline{\uppercase{#1}}\textsuperscript{#3}
	\fi
	\else
	\ifx\\#3\\  \overline{\uppercase{#1}}\textsubscript{#2}
	\else
	 \overline{\uppercase{#1}}\textsubscript{#2}\textsuperscript{#3}
	\fi
	\fi
}

\newcommand{\optimalSetSymbol}[3]
{
	\ifx\\#2\\ \ifx\\#3\\ \optimal{\uppercase{#1}}
	\else
	\optimal{\uppercase{#1}}\textsuperscript{#3}
	\fi
	\else
	\ifx\\#3\\ \optimal{\uppercase{#1}}\textsubscript{#2}
	\else
	\optimal{\uppercase{#1}}\textsubscript{#2}\textsuperscript{#3}
	\fi
	\fi
}

\newcommand{\estimatedOptimalVarSymbol}[3]
{
	\ifx\\#2\\
	\ifx\\#3\\ \optimal{\widetilde{\lowercase{#1}}}
	\else
	\optimal{\widetilde{\lowercase{#1}}}\textsuperscript{#3}
	\fi
	\else
	\ifx\\#3\\ \optimal{\widetilde{\lowercase{#1}}}\textsubscript{#2}
	\else
	\optimal{\widetilde{\lowercase{#1}}}\textsubscript{#2}\textsuperscript{#3}
	\fi
	\fi
}

\newcommand{\estimatedOptimalSetSymbol}[3]
{
	\ifx\\#2\\
	\ifx\\#3\\ \optimal{\widetilde{\uppercase{#1}}}
	\else \optimal{\widetilde{\uppercase{#1}}}\textsuperscript{#3}
	\fi
	\else \ifx\\#3\\ \optimal{\widetilde{\uppercase{#1}}}\textsubscript{#2}
	\else \optimal{\widetilde{\uppercase{#1}}}\textsubscript{#2}\textsuperscript{#3}
	\fi
	\fi
}



%%%%%%%%%% FUNCTIONS %%%%%%%%%%

\newcommand{\functionSymbol}[1]{\lowercase{#1}}
\newcommand{\functionOptimalSymbol}[1]{\lowercase{#1}^*}
\newcommand{\functionTextSymbol}[1]{\texttt{#1}}

\newcommand{\functionFormalSignature}[2]{
	\functionSymbol{#1}\colon #2
}

\newcommand{\functionFormal}[3]{
	\functionFormalSignature{#1}{#2} \rightarrow #3
}

\newcommand{\functionGenerator}[2]{
	#1 \rightarrow #2
}

\newcommand{\functionDefinition}[2]{
	#1 = #2
}

\newcommand{\functionInstance}[3]{
	\functionSignature{#1}{#2} \rightarrow #3
}

\newcommand{\functionSignature}[2]{
\ifx \\#2\\
	\functionSymbol{#1}
\else
	\functionSymbol{#1}(#2)
\fi
}

\newcommand{\funcInstance}[2]{
	\functionSignature{#1}{#2}
}



%%%%%%%%%% SETS %%%%%%%%%%

\newcommand{\setSymbol}[3] {
\ifx\\#2\\\ifx\\#3\\\uppercase{#1}
\else
\uppercase{#1}^{#3}
\fi
\else
\ifx\\#3\\\uppercase{#1}_{#2}
\else
\uppercase{#1}_{#2}^{#3}
\fi
\fi
}

\newcommand{\varSymbolHat}[3]
{
	\ifx \\#2\\
	\ifx \\#3\\
	\widehat{\lowercase{#1}}
	\else
	\widehat{\lowercase{#1}}\textsuperscript{#3}
	\fi
	\else
	\ifx \\#3\\
	\widehat{\lowercase{#1}}\textsubscript{#2}
	\else
	\widehat{\lowercase{#1}}\textsubscript{#2}\textsuperscript{#3}
	\fi
	\fi
}
\newcommand{\setSymbolHat}[3]
{
	\ifx\\#2\\
	\ifx\\#3\\
	\widehat{\uppercase{#1}}
	\else
	\widehat{\uppercase{#1}}\textsuperscript{#3}
	\fi
	\else
	\ifx\\#3\\
	\widehat{\uppercase{#1}}\textsubscript{#2}
	\else
	\widehat{\uppercase{#1}}\textsubscript{#2}\textsuperscript{#3}
	\fi
	\fi
}
\newcommand{\varSymbolPlus}[2]
{
	\ifx\\#2\\
	{\lowercase{#1}}\textsuperscript{$\oplus$}
	\else
	{\lowercase{#1}}\textsubscript{#2}\textsuperscript{\oplus}
	\fi
}

\newcommand{\varSymbolMinus}[2]
{
	\ifx\\#2\\
	{\lowercase{#1}}\textsuperscript{$\ominus$}
	\else
	{\lowercase{#1}}\textsubscript{#2}\textsuperscript{$\ominus$}
	\fi
}


\newcommand{\setSymbolPlus}[2]
{
	\ifx\\#2\\
	{\uppercase{#1}}\textsuperscript{$\oplus$}
	\else
	{\uppercase{#1}}\textsubscript{#2}\textsuperscript{\oplus}
	\fi
}

\newcommand{\setSymbolMinus}[2]
{
	\ifx\\#2\\
	{\uppercase{#1}}\textsuperscript{$\ominus$}
	\else
	{\uppercase{#1}}\textsubscript{#2}\textsuperscript{$\ominus$}
	\fi
}

\newcommand{\setOfSetsSymbolMinus}[2]
{
	\ifx\\#2\\
	{\mathcal{\uppercase{#1}}}\textsuperscript{$\ominus$}
	\else
	{\mathcal{\uppercase{#1}}}\textsubscript{#2}\textsuperscript{$\ominus$}
	\fi
}

\newcommand{\varSymbolHatPlus}[2]
{
	\ifx\\#2\\
	\widehat{\lowercase{#1}}\textsuperscript{$\oplus$}
	\else
	\widehat{\lowercase{#1}}\textsubscript{#2}\textsuperscript{\oplus}
	\fi
}

\newcommand{\setSymbolHatPlus}[2]
{
	\ifx\\#2\\
	\widehat{\uppercase{#1}}\textsuperscript{$\oplus$}
	\else
	\widehat{\uppercase{#1}}\textsubscript{#2}\textsuperscript{\oplus}
	\fi
}

\newcommand{\varSymbolHatMinus}[2]
{
	\ifx\\#2\\
	\widehat{\lowercase{#1}}\textsuperscript{$\ominus$}
	\else
	\widehat{\lowercase{#1}}\textsubscript{#2}\textsuperscript{$\ominus$}
	\fi
}

\newcommand{\setSymbolHatMinus}[2]
{
	\ifx\\#2\\
	\widehat{\uppercase{#1}}\textsuperscript{$\ominus$}
	\else
	\widehat{\uppercase{#1}}\textsubscript{#2}\textsuperscript{$\ominus$}
	\fi
}

\newcommand{\setOfSetsSymbolHatMinus}[2]
{
	\ifx\\#2\\
	\widehat{\mathcal{\uppercase{#1}}}\textsuperscript{$\ominus$}
	\else
	\widehat{\mathcal{\uppercase{#1}}}\textsubscript{#2}\textsuperscript{$\ominus$}
	\fi
}


\newcommand{\setOptimalSymbol}[3]
{
	\ifx\\#2\\
	\ifx\\#3\\
	\optimal{\uppercase{#1}}
	\else
	\optimal{\uppercase{#1}}\textsuperscript{#3}
	\fi
	\else
	\ifx\\#3\\
	\optimal{\uppercase{#1}}\textsubscript{#2}
	\else
	\optimal{\uppercase{#1}}\textsubscript{#2}\textsuperscript{#3}
	\fi
	\fi
}

\newcommand{\powerSetSymbol}[3]
{
	2^{\setSymbol{#1}{#2}{#3}}
}

\newcommand{\powerSetSymbolP}[3]
{
	\ifx \\#2\\ \ifx\\#3\\ \mathcal{P}(\uppercase{#1})
	\else \mathcal{P}({\uppercase{#1}})\textsuperscript{#3}
	\fi
	\else
	\ifx\\#3\\ \mathcal{P}({\uppercase{#1}})\textsubscript{#2}
	\else
	\mathcal{P}({\uppercase{#1}})\textsubscript{#2}\textsuperscript{#3}
	\fi
	\fi
}


\newcommand{\setOfSetsSymbol}[3]{\ifx\\#2\\\ifx\\#3\\{\mathcal{\uppercase{#1}}}\else {\mathcal{\uppercase{#1}}}\textsuperscript{#3}\fi\else \ifx\\#3\\{\mathcal{\uppercase{#1}}}\textsubscript{#2}\else{\mathcal{\uppercase{#1}}}\textsubscript{#2}\textsuperscript{#3}\fi\fi}


\newcommand{\setOfSetsOptimalSymbol}[3]
{
	\ifx\\#2\\ \ifx\\#3\\ \optimal{\mathcal{\uppercase{#1}}}
	\else
	\optimal{\mathcal{\uppercase{#1}}}\textsuperscript{#3}
	\fi
	\else
	\ifx\\#3\\ \optimal{\mathcal{\uppercase{#1}}}\textsubscript{#2}
	\else
	\optimal{\mathcal{\uppercase{#1}}}\textsubscript{#2}\textsuperscript{#3}
	\fi
	\fi
}


\newcommand{\optimal}[1]{{#1}^{\ast}}

\newcommand{\setEstimatedSymbol}[3]
{
	\ifx\\#2\\
	\ifx\\#3\\
	\estimated{\uppercase{#1}}
	\else
	\estimated{\uppercase{#1}}\textsuperscript{#3}
	\fi
	\else
	\ifx\\#3\\
	\estimated{\uppercase{#1}}\textsubscript{#2}
	\else
	\estimated{\uppercase{#1}}\textsubscript{#2}\textsuperscript{#3}
	\fi
	\fi
}

\newcommand{\estimated}[1]
{
	\widehat{{#1}}
}




%%%%%%%%%% SCRIPT %%%%%%%%%%

\newcommand{\scriptSymbol}[3]{\ifx\\#2\\
	\mathcal{\uppercase{#1}}
	\else
	\ifx\\#3\\
	\mathcal{\uppercase{#1}}\textsubscript{#2}
	\else
	\mathcal{\uppercase{#1}}\textsubscript{#2}\textsuperscript{#3}
	\fi
	\fi
}

%%%%%%%%%% ACTIONS  %%%%%%%%%%

\newcommand{\action}[2]{
	$\texttt{#1}(#2)$
}

%%%%%%%%%% FORMATTING %%%%%%%%%%

\newcommand\capitalise[1]{\capitaliseaux#1\relax}
\def\capitaliseaux#1#2\relax{\uppercase{#1}\lowercase{#2}}


\newcommand{\setBuilder}[3]{
	#1
	\IfSubStr{#1}{(}{\funcdef}{=}
	\lbrace
	\IfSubStr{#2}{,}{(#2)}{#2}
	\ifx \\#3\\ \rbrace \else \suchthat #3 \rbrace \fi
}

\newcommand{\funcdef}{=}
\newcommand{\matrixdef}{=}
\newcommand{\funcupdate}{\leftarrow}
\newcommand{\suchthat}{:\ }


% Standard sets
\newcommand{\setRealNumbers}[2]{\mathbb{R}_{#1}^{#2}}
\newcommand{\setRealNumbersPositive}[2]{\mathbb{R}_{>0}^{#2}}
\newcommand{\setRealNumbersNonNegative}[2]{\mathbb{R}_{\geq0}^{#2}}
\newcommand{\setRealNumbersUnit}[2]{\mathbb{R}_{#1}^{#2}[0,1]}
\newcommand{\setRealNumbersPositiveUnit}[2]{\mathbb{R}_{>0}^{#2}[0,1]}

\newcommand{\setIntegers}[2]{\mathbb{Z}_{#1}{#2}}
\newcommand{\setIntegersPositive}[2]{\mathbb{Z}_{+}^{#2}}
\newcommand{\setIntegersNonNegative}[2]{\mathbb{Z}_{\geq0}^{#2}}
\newcommand{\setNaturalNumbers}[2]{\mathbb{N}_{0}^{#2}}
\newcommand{\setNaturalNumberPositive}[2]{\mathbb{N}_{}^{*}}
\newcommand{\varProbability}[2]{\varSymbol{p}{#1}{#2}}


% Common
\newcommand{\varX}[2]{\varSymbol{x}{#1}{#2}}
\newcommand{\varY}[2]{\varSymbol{y}{#1}{#2}}
\newcommand{\varZ}[2]{\varSymbol{z}{#1}{#2}}
\newcommand{\setX}[2]{\setSymbol{x}{#1}{#2}}
\newcommand{\setY}[2]{\setSymbol{y}{#1}{#2}}
\newcommand{\setZ}[2]{\setSymbol{z}{#1}{#2}}

\newcommand{\varK}[2]{\varSymbol{k}{#1}{#2}}
\newcommand{\setK}[2]{\setSymbol{K}{#1}{#2}}
\newcommand{\varJ}[2]{\varSymbol{j}{#1}{#2}}
\newcommand{\setJ}[2]{\setSymbol{J}{#1}{#2}}
\newcommand{\varN}[2]{\varSymbol{n}{#1}{#2}}
\newcommand{\setN}[2]{\setSymbol{N}{#1}{#2}}
\newcommand{\varA}[2]{\varSymbol{A}{#1}{#2}}
\newcommand{\setA}[2]{\setSymbol{A}{#1}{#2}}
\newcommand{\varB}[2]{\varSymbol{B}{#1}{#2}}
\newcommand{\setB}[2]{\setSymbol{B}{#1}{#2}}


% Functions
\newcommand{\funcHadamard}[2]{
	#1 \circ #2
}
\newcommand{\funcSumNorm}[1]{
	\functionTextSymbol{sumnorm}(#1)
}


\newcommand{\functionSumNorm}[1]{
	#1 \leftarrow \functionTextSymbol{sumnorm}(#1)
}

\newcommand{\funcMax}[1]{\functionTextSymbol{max}(#1)}
\newcommand{\funcMaxExtended}[2]{\functionTextSymbol{max}_{\tiny #1}(#2)}
\newcommand{\funcMin}[1]{\functionTextSymbol{min}(#1)}
\newcommand{\funcMinExtended}[2]{\functionTextSymbol{min}_{\tiny #1}(#2)}
\newcommand{\funcBoltzmann}[1]{\functionTextSymbol{boltzmann}(#1)}
\newcommand{\functionBoltzmann}{
	P(\varAction{i}{}) = \dfrac{e^{(\varQ{i}{}/\tau)}}{\sum_{j=1}^{N} e^{(\varQ{j}{}/\tau)}}
}

\newcommand{\argmax}[2]{\underset{#1}{argmax\ }}
\newcommand{\argmin}[2]{\underset{#1}{argmin\ }}

\newcommand{\funcSize}[1]{\lvert #1 \rvert}

\newcommand{\funcInstanceSoftmaxSymbol}[1]{
	\sigma(#1)
}
\newcommand{\funcSoftMax}[1]{\functionTextSymbol{softmax}(#1)}

\newcommand{\functionNorm}{
	f(\varX{i}{}) = \dfrac{{\varX{i}{}}}
	{\sum_{j=1}^{N} {\varX{j}{}}}
}

\newcommand{\functionSoftMax}{
	\sigma(\varX{i}{}) = \dfrac{e^{\varX{i}{}}}
	{\sum_{j=1}^{N} e^{\varX{j}{}}}
}


\newcommand{\functionSoftMaxExtended}[1]{
	\sigma(#1_{i}^{}) = \dfrac{e^{#1_{i}^{}}}
	{\sum_{j=1}^{N} e^{#1_{j}^{}}}
}


\newcommand{\funcRand}[1]{\functionTextSymbol{rand}(#1)}
\newcommand{\functionRand}{
	P(\varX{i}{}) = U(\varX{i}{})
}



%%%Author macros
\def\tsc#1{\csdef{#1}{\textsc{\lowercase{#1}}\xspace}}
\tsc{WGM}
\tsc{QE}
\tsc{EP}
\tsc{PMS}
\tsc{BEC}
\tsc{DE}
%%%
\newtheorem{example}{Example}[section]

\newcommand{\nosemic}{\renewcommand{\@endalgocfline}{\relax}}% Drop semi-colon ;
\newcommand{\dosemic}{\renewcommand{\@endalgocfline}{\algocf@endline}}% Reinstate semi-colon ;
\newcommand{\pushline}{\Indp}% Indent
\newcommand{\popline}{\Indm\dosemic}% Undent
\let\oldnl\nl% Store \nl in \oldnl
\newcommand{\nonl}{\renewcommand{\nl}{\let\nl\oldnl}}% Remove line number for one line

\newcommand{\DEBUG}{}%
\ifdefined\DEBUG
	\usepackage{mdframed}
	\newcommand{\reviewquestion}[1]{
		\begin{mdframed}[backgroundcolor=blue!20,rightline=false,leftline=false,topline=false,bottomline=false, innerleftmargin=2pt, innerrightmargin=2pt, innertopmargin=5pt, innerbottommargin=5pt]
			#1
		\end{mdframed}
	}%
	\newcommand{\reviewquestionopen}[1]{
		\begin{mdframed}[backgroundcolor=red!20,rightline=false,leftline=false,topline=false,bottomline=false, innerleftmargin=2pt, innerrightmargin=2pt, innertopmargin=5pt, innerbottommargin=5pt]
			#1
		\end{mdframed}
	}%
	\newcommand{\reviewresponse}[1]{
		\begin{mdframed}[backgroundcolor=yellow!20,rightline=false,leftline=false,topline=false,bottomline=false, innerleftmargin=2pt, innerrightmargin=2pt, innertopmargin=5pt, innerbottommargin=5pt]
			#1
		\end{mdframed}
	}%
\newcommand{\reviewtodo}[1]{
	\begin{mdframed}[backgroundcolor=orange!20,rightline=false,leftline=false,topline=false,bottomline=false, innerleftmargin=2pt, innerrightmargin=2pt, innertopmargin=5pt, innerbottommargin=5pt]
		#1
	\end{mdframed}
}%
\else
	\newcommand{\reviewquestion}[1]{}
	\newcommand{\reviewquestionopen}[1]{}
	\newcommand{\reviewresponse}[1]{#1}
	\newcommand{\reviewtodo}[1]{}
\fi
	
\begin{document}
	\let\WriteBookmarks\relax
	\def\floatpagepagefraction{1}
	\def\textpagefraction{.001}
	\shorttitle{Multi-objective optimisation in WSN systems}
	\shortauthors{N Creech et~al.}
	%\begin{frontmatter}
	
	\title [mode = title]{Multi-objective optimisation in WSN systems}                      
	
	\author[1]{Niall Creech}[type=editor,	orcid=0000-0002-9573-0991]
	\ead{niall.creech@kcl.ac.uk}
	\cortext[cor1]{Corresponding author}	
	\credit{Conceptualization of this study, Methodology, Software}
	
	\author[1]{Natalia Criado}[]
	\ead{natalia.criado@kcl.ac.uk}
	\credit{Supervision, Writing - Review and Editing}

	\author[1]{Simon Miles}[]
	\ead{simon.miles@kcl.ac.uk}
	\credit{Supervision, Writing - Review and Editing}

	\address[1]{Department of Informatics, King's College London, Bush House, Strand Campus, 30, Aldwych, London WC2B 4BG}

	\begin{abstract}
Wireless sensor networks (WSN) are applicable across a wide range of industries from vehicle-to-vehicle communications to large-scale environmental monitoring. They need to manage energy usage to maintain availbility, distribute tasks effectively, and handle distruption to network changes and agent loss. Decentralised algorithms are commonly used to meet these challenges, with hierarchical cluster formation or reinforcement learning techniques. There are challenges however in getting these algorithms to perform well in large distributed systems where there are multiple objectives, dynamic agents, or connectivity change. In this work we propose the novel \acronymWSNOptimisationExtended{}{}
algorithm to optimise WSN systems based on the multiple objectives of maximising energy availability, distribution, and task quality, while maintaining task coverage in a dynamic network. This integrates and extends upon the previously defined \acronymATARIAExtended{}{} and \acronymMGRAOExtended{}{} algorithms, adding hierarchical task allocation with multi-objective composite task value calculation. We evaluate the algorithms' performance in environmental monitoring-based simulated systems where there are a number of measurement tasks to be completed within the system. The \acronymWSNOptimisation{}{} algorithm showed a \resultsCTVBalancedDiff{}{} system utility improvement in the simple $10$ node system, and \resultsCTVBalancedExtDiff{}{} in a more complex $25$ node system over $500$ episodes. Energy availability was increased by \resultsEnergyBalancedDiff{}{} and \resultsEnergyBalancedExtDiff{}{} respectively. Evaluation of changing algorithm parameters to balance between energy availability, distribution, and task quality showed that these individual components could be prioritised in different ratios depending on the requirements of the optimisation required in the system.  
\end{abstract}

%\begin{graphicalabstract}
%	\includegraphics{figs/grabs.pdf}
%\end{graphicalabstract}
	
	%\begin{highlights}
	%	\item Research highlights item 1
	%	\item Research highlights item 2
	%	\item Research highlights item 3
	%\end{highlights}
	
	\begin{keywords}
		multi-agent system
		\sep multi-agent reinforcement learning
		\sep wireless sensor network
	\end{keywords}

	\newcommand{\acronymATARIA}[2]{ATA-RIA}
\newcommand{\acronymATARIAExtended}[2]{agent task allocation with risk-impact awareness (\acronymATARIA{}{})}
\newcommand{\acronymMGRAO}[2]{MG-RAO}
\newcommand{\acronymMGRAOExtended}[2]{multi-group resource allocation optimisation (\acronymMGRAO{}{})}
\newcommand{\acronymWSNOptimisation}[2]{AN-HTAO}
\newcommand{\acronymWSNOptimisationExtended}[2]{agent networks with hierarchical task allocation optimisation (\acronymWSNOptimisation{}{})}

\newcommand{\acronymWSNOptimisationSink}[2]{AN-HTAO (Sink agents)}
\newcommand{\acronymWSNOptimisationSinkExtended}[2]{agent networks with hierarchical task allocation optimisation, sink agents (\acronymWSNOptimisationSink{}{})}
\newcommand{\acronymWSNOptimisationArc}[2]{AN-HTAO (Task-path agents)}
\newcommand{\acronymWSNOptimisationArcExtended}[2]{agent networks with hierarchical task allocation optimisation, arc agents (\acronymWSNOptimisationArc{}{})}

\newcommand{\simulationSimple}[2]{\texttt{simple}}
\newcommand{\simulationExtended}[2]{\texttt{extended}}
\newcommand{\simulationNodeFailure}[2]{\texttt{node-failure}}


\newcommand{\algorithmSymbol}[2]{#1}
\newcommand{\algorithmBalanced}[2]{\algorithmSymbol{an-htao}{}}\newcommand{\algorithmBalancedSimple}[2]{\algorithmSymbol{an-htao}{} (simple)}
\newcommand{\algorithmBalancedExt}[2]{\algorithmSymbol{an-htao (extended)}{}}
\newcommand{\algorithmFailure}[2]{\algorithmSymbol{an-htao (failure)}{}}
\newcommand{\algorithmEnergy}[2]{\algorithmSymbol{an-htao (energy)}{}}
\newcommand{\algorithmQuality}[2]{\algorithmSymbol{an-htao (quality)}{}}
\newcommand{\algorithmDistribution}[2]{\algorithmSymbol{an-htao (distribution)}{}}
\newcommand{\algorithmQRouting}[2]{\algorithmSymbol{q-routing}{}}
\newcommand{\algorithmQRoutingSimple}[2]{\algorithmSymbol{q-routing}{}(simple)}
\newcommand{\algorithmQRoutingExt}[2]{\algorithmSymbol{q-routing}{} (extended)}
\newcommand{\algorithmQRoutingFailure}[2]{\algorithmSymbol{q-routing}{} (failure)}

\newcommand{\algorithmBaseline}{\algorithmQRouting{}{}}
\newcommand{\acronymQRouting}[2]{Q-Routing}


	
\newcommand{\functionComplete}[2]{
	\functionSignature{complete}{#1}
}
\newcommand{\functionNotComplete}[2]{
	\functionSignature{\neg complete}{#1}
}
\newcommand{\functionWait}[2]{
	\functionSignature{wait}{#1}
}

	
%%%%%%% SIMPLE %%%%%%%%
\newcommand{\resultsSimpleCTVBalancedStart}[2]{$11.2$}
\newcommand{\resultsSimpleCTVQRoutingStart}[2]{$11.0$}
\newcommand{\resultsSimpleCTVBalancedEnd}[2]{$14.1$}
\newcommand{\resultsSimpleCTVQRoutingEnd}[2]{$13.8$}
\newcommand{\resultsSimpleCumulativeCTVComparison}[2]{$138$}
\newcommand{\resultsSimpleCTVBalancedDiff}[2]{$25.9\%$}
\newcommand{\resultsSimpleCTVQRoutingDiff}[2]{$25.4$}

\newcommand{\resultsSimpleEnergyBalancedStart}[2]{$81.5\%$}
\newcommand{\resultsSimpleEnergyQRoutingStart}[2]{$81.5\%$}
\newcommand{\resultsSimpleEnergyBalancedEnd}[2]{$84.4\%$}
\newcommand{\resultsSimpleEnergyQRoutingEnd}[2]{$83.8\%$}
\newcommand{\resultsSimpleEnergyBalancedDiff}[2]{$3.6\%$}
\newcommand{\resultsSimpleEnergyQRoutingDiff}[2]{$2.8\%$}
\newcommand{\resultsSimpleCumulativeEnergyComparison}[2]{$600\%$}
%%%%%%%%%%%%%%%%%%%%%%%

%%%%%%% NODE FAILURE %%%%%%%%
\newcommand{\resultsNodeFailureCTVBalancedStart}[2]{$17.9$}
\newcommand{\resultsNodeFailureCTVQRoutingStart}[2]{$17.6$}
\newcommand{\resultsNodeFailureCTVBalancedEnd}[2]{$18.3$}
\newcommand{\resultsNodeFailureCTVQRoutingEnd}[2]{$17.8$}
\newcommand{\resultsNodeFailureCTVBalancedDiff}[2]{$2.2\%$}
\newcommand{\resultsNodeFailureCTVQRoutingDiff}[2]{$1.1\%$}
\newcommand{\resultsNodeFailureCTVBalancedImpactStart}[2]{$18.3$}
\newcommand{\resultsNodeFailureCTVQRoutingImpactStart}[2]{$18.3$}
\newcommand{\resultsNodeFailureCTVBalancedImpactEnd}[2]{$17.9$}
\newcommand{\resultsNodeFailureCTVQRoutingImpactEnd}[2]{$17.2$}
\newcommand{\resultsNodeFailureCTVBalancedImpactDiff}[2]{$-2.1\%$}
\newcommand{\resultsNodeFailureCTVQRoutingImpactDiff}[2]{$-6.0\%$}
\newcommand{\resultsNodeFailureCumulativeCTVComparison}[2]{$40$}

\newcommand{\resultsNodeFailureEnergyBalancedStart}[2]{$77\%$}
\newcommand{\resultsNodeFailureEnergyQRoutingStart}[2]{$77\%$}
\newcommand{\resultsNodeFailureEnergyBalancedEnd}[2]{$77\%$}
\newcommand{\resultsNodeFailureEnergyQRoutingEnd}[2]{$72\%$}
\newcommand{\resultsNodeFailureEnergyBalancedDiff}[2]{$0\%$}
\newcommand{\resultsNodeFailureEnergyQRoutingDiff}[2]{$-5\%$}
\newcommand{\resultsNodeFailureEnergyBalancedImpactStart}[2]{$83\%$}
\newcommand{\resultsNodeFailureEnergyQRoutingImpactStart}[2]{$83\%$}
\newcommand{\resultsNodeFailureEnergyBalancedImpactEnd}[2]{$68\%$}
\newcommand{\resultsNodeFailureEnergyQRoutingImpactEnd}[2]{$63\%$}
\newcommand{\resultsNodeFailureEnergyBalancedImpactDiff}[2]{$-15\%$}
\newcommand{\resultsNodeFailureEnergyQRoutingImpactDiff}[2]{$-20\%$}
\newcommand{\resultsNodeFailureCumulativeEnergyComparison}[2]{$230\%$}

\newcommand{\resultsNodeFailureFailedAgentsBalancedStart}[2]{$0\%$}
\newcommand{\resultsNodeFailureFailedAgentsQRoutingStart}[2]{$0\%$}
\newcommand{\resultsNodeFailureFailedAgentsBalancedEnd}[2]{$35\%$}
\newcommand{\resultsNodeFailureFailedAgentsQRoutingEnd}[2]{$35\%$}
\newcommand{\resultsNodeFailureFailedAgentsBalancedDiff}[2]{$35\%$}
\newcommand{\resultsNodeFailureFailedAgentsQRoutingDiff}[2]{$35\%$}

\newcommand{\resultsNodeFailureCoverageBalancedStart}[2]{$100\%$}
\newcommand{\resultsNodeFailureCoverageQRoutingStart}[2]{$100\%$}
\newcommand{\resultsNodeFailureCoverageBalancedEnd}[2]{$78.5\%$}
\newcommand{\resultsNodeFailureCoverageQRoutingEnd}[2]{$73.0\%$}
\newcommand{\resultsNodeFailureCoverageBalancedDiff}[2]{$-21.5\%$}
\newcommand{\resultsNodeFailureCoverageQRoutingDiff}[2]{$-27.0\%$}
%%%%%%%%%%%%%%%%%%%%%%%


%%%%%%% EXT %%%%%%%%
\newcommand{\resultsExtendedCTVBalancedStart}[2]{$XXX$}
\newcommand{\resultsExtendedCTVQRoutingStart}[2]{$XXX$}
\newcommand{\resultsExtendedCTVBalancedEnd}[2]{$XXX$}
\newcommand{\resultsExtendedCTVQRoutingEnd}[2]{$XXX$}
\newcommand{\resultsExtendedCumulativeCTVComparison}[2]{$XXX$}
\newcommand{\resultsExtendedCTVBalancedDiff}[2]{$XXX\%$}
\newcommand{\resultsExtendedCTVQRoutingDiff}[2]{$XXX$}

\newcommand{\resultsExtendedEnergyBalancedStart}[2]{$XXX\%$}
\newcommand{\resultsExtendedEnergyQRoutingStart}[2]{$XXX\%$}
\newcommand{\resultsExtendedEnergyBalancedEnd}[2]{$XXX\%$}
\newcommand{\resultsExtendedEnergyQRoutingEnd}[2]{$XXX\%$}
\newcommand{\resultsExtendedEnergyBalancedDiff}[2]{$XXX\%$}
\newcommand{\resultsExtendedEnergyQRoutingDiff}[2]{$XXX\%$}
\newcommand{\resultsExtendedCumulativeEnergyComparison}[2]{$XXX\%$}
%%%%%%%%%%%%%%%%%%%%%%%


\newcommand{\resultsCTVBalancedStart}[2]{$68\%$}
\newcommand{\resultsCTVBalancedEnd}[2]{$90\%$}



\newcommand{\resultsCTVGainBalancedEnd}[2]{$22\%$}
\newcommand{\resultsCTVBalancedDiff}[2]{$22\%$}

\newcommand{\resultsCTVQRoutingStart}[2]{$68\%$}
\newcommand{\resultsCTVQRoutingEnd}[2]{$91\%$}
\newcommand{\resultsCTVGainQRoutingEnd}[2]{$22\%$}
\newcommand{\resultsCTVQRoutingDiff}[2]{$4\%$}

\newcommand{\resultsCTVBalancedExtStart}[2]{$45\%$}
\newcommand{\resultsCTVBalancedExtEnd}[2]{$52\%$}
\newcommand{\resultsCTVBalancedExtDiff}[2]{$14\%$}

\newcommand{\resultsEnergyBalancedStart}[2]{$67\%$}
\newcommand{\resultsEnergyBalancedEnd}[2]{$73\%$}
\newcommand{\resultsEnergyGainBalancedEnd}[2]{$22\%$}
\newcommand{\resultsEnergyBalancedDiff}[2]{$6\%$}

\newcommand{\resultsEnergyQRoutingStart}[2]{$67\%$}
\newcommand{\resultsEnergyQRoutingEnd}[2]{$71\%$}
\newcommand{\resultsEnergyGainQRoutingEnd}[2]{$22\%$}
\newcommand{\resultsEnergyQRoutingDiff}[2]{$4\%$}



\newcommand{\resultsEnergyBalancedExtStart}[2]{$68\%$}
\newcommand{\resultsEnergyBalancedExtEnd}[2]{$97\%$}
\newcommand{\resultsEnergyGainBalancedExtEnd}[2]{$XX\%$}
\newcommand{\resultsEnergyBalancedExtDiff}[2]{$29\%$}

\newcommand{\resultsEnergyQualityStart}[2]{$XXX\%$}
\newcommand{\resultsEnergyQualityEnd}[2]{$XXX\%$}
\newcommand{\resultsEnergyQualityDiff}[2]{$XXX\%$}
\newcommand{\resultsEnergyDistStart}[2]{$XXX\%$}
\newcommand{\resultsEnergyDistEnd}[2]{$XXX\%$}
\newcommand{\resultsEnergyDistDiff}[2]{$XXX\%$}
\newcommand{\resultsEnergyEnergyStart}[2]{$XXX\%$}
\newcommand{\resultsEnergyEnergyEnd}[2]{$XXX\%$}
\newcommand{\resultsEnergyEnergydDiff}[2]{$XXX\%$}

\newcommand{\resultsQEQualityStart}[2]{$2\%$}
\newcommand{\resultsQEQualityEnd}[2]{$13\%$}
\newcommand{\resultsQEQualityDiff}[2]{$11\%$}
\newcommand{\resultsQEDistStart}[2]{$1\%$}
\newcommand{\resultsQEDistEnd}[2]{$4\%$}
\newcommand{\resultsQEDistDiff}[2]{$3\%$}

\newcommand{\resultsTaskDistBalancedExtStart}[2]{$XXX$}
\newcommand{\resultsTaskDistBalancedExtEnd}[2]{$XXX$}
\newcommand{\resultsTaskDistBalancedExtDiff}[2]{$XXX$}
\newcommand{\resultsTaskDistBalancedExtPercent}[2]{$XXX\%$}
\newcommand{\resultsTaskDistQualityStart}[2]{$0.560$}
\newcommand{\resultsTaskDistQualityEnd}[2]{$0.552$}
\newcommand{\resultsTaskDistQualityDiff}[2]{$-0.008$}
\newcommand{\resultsTaskDistQualityPercent}[2]{$1.4\%$}
\newcommand{\resultsTaskDistDistStart}[2]{$0.579$}
\newcommand{\resultsTaskDistDistEnd}[2]{$0.572$}
\newcommand{\resultsTaskDistDistDiff}[2]{$0.005$}
\newcommand{\resultsTaskDistDistPercent}[2]{$-0.9\%$}
\newcommand{\resultsTaskDistEnergyStart}[2]{$0.505$}
\newcommand{\resultsTaskDistEnergyEnd}[2]{$0.488$}
\newcommand{\resultsTaskDistEnergydDiff}[2]{$0.017$}
\newcommand{\resultsTaskDistEnergydPercent}[2]{$3.4\%$}

\newcommand{\resultsArcDepthBalancedExtStart}[2]{$XXX$}
\newcommand{\resultsArcDepthBalancedExtEnd}[2]{$XXX$}
\newcommand{\resultsArcDepthBalancedExtDiff}[2]{$XXX$}
\newcommand{\resultsArcDepthBalancedExtPercent}[2]{$XXX\%$}
\newcommand{\resultsArcDepthQualityStart}[2]{$3.55$}
\newcommand{\resultsArcDepthQualityEnd}[2]{$3.38$}
\newcommand{\resultsArcDepthQualityDiff}[2]{$0.17$}
\newcommand{\resultsArcDepthQualityPercent}[2]{$4.7\%$}
\newcommand{\resultsArcDepthDistStart}[2]{$3.10$}
\newcommand{\resultsArcDepthDistEnd}[2]{$3.06$}
\newcommand{\resultsArcDepthDistDiff}[2]{$0.04$}
\newcommand{\resultsArcDepthDistPercent}[2]{$1.2\%$}
\newcommand{\resultsArcDepthEnergyStart}[2]{$3.02$}
\newcommand{\resultsArcDepthEnergyEnd}[2]{$2.50$}
\newcommand{\resultsArcDepthEnergyDiff}[2]{$0.52$}
\newcommand{\resultsArcDepthEnergyPercent}[2]{$17.2\%$}


%%%%%%% NODE-FAILURE %%%%%%%%
\newcommand{\resultsNodeFailureCTV}[2]{$XXX$}
\newcommand{\resultsNodeFailureCumulativeCTV}[2]{$XXX$}
\newcommand{\resultsNodeFailureEnergy}[2]{$XXX$}
\newcommand{\resultsNodeFailureCumulativeEnergy}[2]{$XXX$}
%%%%%%%%%%%%%%%%%%%%%%%

%%%%%%% SIMPLE %%%%%%%%
\newcommand{\resultsExtendedCTV}[2]{$XXX$}
\newcommand{\resultsExtendedCumulativeCTV}[2]{$XXX$}
\newcommand{\resultsExtendedEnergy}[2]{$XXX$}
\newcommand{\resultsExtendedCumulativeEnergy}[2]{$XXX$}
%%%%%%%%%%%%%%%%%%%%%%%

		%%%%%%%%%%%%%%%%%%%% NOTATION %%%%%%%%%%%%%%%%%%%%
\newcommand{\acronymTaskAllocation}{\texttt{ATA-RIA}}
\newcommand{\acronymTaskAllocationExtended}{agent task allocation with risk-impact awareness (\acronymTaskAllocation)}
\newcommand{\acronymRewardTrends}{RT-ARP}
\newcommand{\acronymRewardTrendsExtended}{reward trends for action-risks probabilities (\acronymRewardTrends)}
\newcommand{\acronymMemoryRetention}{\texttt{SAS-KR}}
\newcommand{\acronymMemoryRetentionExtended}{state-action space knowledge-retention (\acronymMemoryRetention)}
\newcommand{\acronymNeighbourhoodPruningAlgorithm}{\texttt{N-Prune}}
\newcommand{\acronymNeighbourhoodPruningAlgorithmExtended}{neighbourhood update (\acronymNeighbourhoodPruningAlgorithm)}
\newcommand{\acronymDistributedSystem}{{DTAS}}
\newcommand{\acronymDistributedSystemExtended}{distributed task-allocation system (\acronymDistributedSystem)}

\newcommand{\acronymActionInformationQuality}[2]{\ifx&#1&action information quality
	\else
	Action information quality
	\fi}
\newcommand{\acronymTD}{\texttt{TD-Update}}
\newcommand{\acronymTDExtended}{temporal-difference update (\acronymTD)}
\newcommand{\acronymRewardSet}{\texttt{TSQM}}
%%%%%%%%%%%%%%%%%%%%%%%%%%%%%%%%%%%%%%%%%%%%%%%%%%
%%%%%%% NOTATION %%%%%%%%%%%%%%
\newcommand{\acronymResourceAllocationAlgorithm}[2]{MG-RAO}
\newcommand{\acronymResourceAllocationAlgorithmExtended}[2]{multi-group resource allocation optimisation (MG-RAO)}
	%%%%%%%%%%%%%%%%%%%%%%%%%%%%%%%%%%%%%
%\newcommand{\varAtomicTask}[2]{\varSymbol{at}{#1}{#2}}
%\newcommand{\setAtomicTask}[2]{\setSymbol{at}{#1}{#2}}
%\newcommand{\varCompositeTask}[2]{\varSymbol{ct}{#1}{#2}}
%\newcommand{\setCompositeTask}[2]{\setSymbol{ct}{#1}{#2}}
%\newcommand{\varAgent}[2]{\varSymbol{g}{#1}{#2}}
%\newcommand{\varChildAgent}[2]{\varSymbol{g}{c}{#2}}
%\newcommand{\varParentAgent}[2]{\varSymbol{g}{p}{#2}}
%\newcommand{\varAction}[2]{\varSymbol{A}{#1}{#2}}

\newcommand{\setAtomicTaskSubset}[2]{\setSymbol{at'}{#1}{#2}}
\newcommand{\setAtomicTaskSys}[2]{\setSymbol{at}{sys}{#2}}
\newcommand{\varAtomicTaskType}[2]{\varSymbol{ap}{#1}{#2}}
\newcommand{\setAtomicTaskType}[2]{\setSymbol{ap}{#1}{#2}}
\newcommand{\functionAtomicTaskMappingSymbol}[2]{\functionSymbol{type_a}}
\newcommand{\functionAtomicTaskMapping}[2]{\functionSymbol{type_a}(#1)}

\newcommand{\varCompositeTaskType}[2]{\varSymbol{cp}{#1}{#2}}
\newcommand{\setCompositeTaskType}[2]{\setSymbol{CP}{#1}{#2}}
\newcommand{\functionCompositeTaskMappingSymbol}[2]{\functionSymbol{type_c}}
\newcommand{\functionCompositeTaskMapping}[2]{\functionSymbol{type_c}(#1)}

\newcommand{\setAgent}[2]{\setSymbol{G}{#1}{#2}}

\newcommand{\setChildAgent}[2]{\setSymbol{G}{c}{#2}}

\newcommand{\setParentAgent}[2]{\setSymbol{G}{p}{#2}}
\newcommand{\setAgentSubset}[2]{\setSymbol{G'}{#1}{#2}}
\newcommand{\setAgentSys}[2]{\setSymbol{G}{sys}{#2}}
\newcommand{\powerSetAgent}[2]{\powerSetSymbol{G}{#1}{#2}}


\newcommand{\setAction}[2]{\setSymbol{A}{#1}{#2}}
\newcommand{\setNeighbourhoodTargetActions}[2]{
	\ifx \\#1\\
	\setAction{g \succ \setAgent{}{}}{}
	\else
	\setAction{#1 \succ #2}{}
	\fi
}
\newcommand{\setJointSystemAllocation}[2]{\setSymbol{AL}{#1}{#2}}
\newcommand{\setJointSystemAllocationSubset}[2]{\setSymbol{AL'}{#1}{#2}}


%%%%%%%%%%%%%%%%%%%%%%%%%%%%%%%
\newcommand{\tupleAgent}[2]{\langle c,r, \delta_n, \delta_k \rangle}
\newcommand{\varAgentCapability}[2]{\varSymbol{c}{#1}{#2}}
\newcommand{\functionAgentCapability}[2]{\functionSignature{\varAgentCapability{}{}}{\varAgent{}{}}{#2}}
\newcommand{\varAgentResponsiblity}[2]{\varSymbol{r}{#1}{#2}}
\newcommand{\varAgentNeighbourhoodConstraint}[2]{\delta_n}
\newcommand{\functionAgentNeighbourhoodConstraint}[2]{\functionSignature{\varAgentNeighbourhoodConstraint{}{}}{\varAgent{}{}}{#2}}
\newcommand{\varAgentKnowledgeConstraint}[2]{\delta_k}
\newcommand{\functionAgentKnowledgeConstraint}[2]{\functionSignature{\varAgentKnowledgeConstraint{}{}}{\varAgent{}{}}{#2}}
%%%%%%%%%%%%%%%%%%%%%%%%%%%%%%%


%%%%%%%%%%%%%%%%%%%%%%%
\newcommand{\functionNeighbourhoodSignature}[2]{
	\ifx \\#1\\
	\setSymbol{N}{}{}(\varAgent{}{})
	\else
	\setSymbol{N}{}{}(#1)
	\fi
}
\newcommand{\functionKnowledgeSignature}[2]{
	\ifx \\#1\\
	\setSymbol{K}{}{}(\varAgent{}{})
	\else
	\setSymbol{K}{}{}(#1)
	\fi
}
\newcommand{\tupleAllocation}[2]{\langle \{at\}, t, g ,a\rangle}
%%%%%%%%%%%%%%%%%%%%%%%


%%%%%%%%%%%%%%%%%%%% NOTATION %%%%%%%%%%%%%%%%%%%%
\newcommand{\functionNISignature}[2]{
	\functionSignature{NI}
	{
		\setAtomicTask{}{},
		\setX{}{},
		\setY{}{},
		\setJointSystemAllocation{}{}
	}
}


\newcommand{\functionMNISignature}[2]{
	\ifx \\#1\\
	\functionSignature{MNI}{
		\setAtomicTask{}{},
		\setK{}{},
		\setJointSystemAllocation{}{}
	}
	\else
	\functionSignature{MNI}{#1}
	\fi
}

\newcommand{\functionKISignature}[2]{
	\ifx \\#1\\
	\functionSignature{KI}
	{
		\setAtomicTask{}{},
		\setJ{}{},
		\setK{}{},
		\setJointSystemAllocation{}{}
	}
	\else
	\functionSignature{KI}{#1}
	\fi
}
%%%%%%%%%%%%%%%%%%%% NOTATION %%%%%%%%%%%%%%%%%%%%
\newcommand{\varProbabilityNeighbourhoodDelta}{
	\varSymbol{p}{N_{x \cap y}}{}
}
\newcommand{\varProbabilityKnowledgeDelta}{
	\varSymbol{p}{K_{j \cap k}}{}
}
%%%%%%%%%%%%%%%%%%%% NOTATION %%%%%%%%%%%%%%%%%%%%
\newcommand{\functionAISignature}[2]{
	\functionSignature{AI}
	{
		\setAtomicTask{}{},
		\setX{}{},
		\setY{}{},
		\setJ{}{},
		\setK{}{},
		\setJointSystemAllocation{}{}
	}
}
\newcommand{\functionAIEstimatedSignature}[2]{
	\functionSignature{\estimated{AI}}
	{
		\setAtomicTask{}{},
		\setX{}{},
		\setY{}{},
		\setJ{}{},
		\setK{}{},
		\setJointSystemAllocation{}{}
	}
}

%%%%%%%%%%%%%%%%%%%% NOTATION %%%%%%%%%%%%%%%%%%%%
\newcommand{\setRiskAction}[2]{\setSymbol{W}{#1}{#2}}
%%%%%%%%%%%%%%%%%%%%%%%%%%%%%%%%%%%%%%%%%%%%%%%%%%

%%%%%%%%%%%%%%%%%%%% NOTATION %%%%%%%%%%%%%%%%%%%%
\newcommand{\functionSymbolOLMetric}[2]{
	d_{#1}^{#2}
}

\newcommand{\functionOLMetricLocalSignature}[2]{
	\ifx \\#1\\
	\functionSignature{d_{\texttt{loc}}^{#2}}
	{
		\varAgent{}{},
		\setAtomicTask{}{},
		\setAgent{}{},
		\setJointSystemAllocation{}{}
	}
	\else
	\functionSignature{d_{\texttt{loc}}^{#2}}
	{#1}
	\fi
}

%%%%%%%%%%%%%%%%%%%% NOTATION %%%%%%%%%%%%%%%%%%%%
\newcommand{\functionOLMetricSystemSignature}[2]{
	\ifx \\#1\\
	\functionSignature{d_{\texttt{sys}}^{#2}}
	{
		\varAgent{}{},
		\setAtomicTask{}{},
		\setAgent{}{},
		\setJointSystemAllocation{}{}
	}
	\else
	\functionSignature{d_{\texttt{sys}}^{#2}}
	{#1}
	\fi
}
%%%%%%%%%%%%%%%%%%%%%%%%%%%%%%%%%%%%%%%%%%%%%%%%%%
%%%%%%%%%%%%%%%%%%%% NOTATION %%%%%%%%%%%%%%%%%%%%
\newcommand{\setRewardSet}[2]{\setSymbol{\Lambda}{#1}{#2}}

\newcommand{\defSetRewardSet}[3]{ $\setRewardSet{#1}{#2}$, The \acronymRewardSet{}, a matrix of sum    marised quality trends#3.}
\newcommand{\functionUpdateTQSMSignature}[2]{
	\functionSignature{\texttt{UPDATETQSM}}{\setRewardSet{\varAgent{}{}}{}, \varAtomicTaskQualityValue{}{}}
}

%%%%%%%%%%%%%%%%%%%% NOTATION %%%%%%%%%%%%%%%%%%%%
\newcommand{\functionImpactInterpolationSignature}[2]{
	\ifx \\#1\\
	\functionSignature{II}
	{\varX{}{}}
	\else
	\functionSignature{II}
	{#1}
	\fi
}
\newcommand{\varDecay}[2]{\varSymbol{\delta}{#1}{#2}}

%%%%%%%%%%%%%%%%%%%% NOTATION %%%%%%%%%%%%%%%%%%%%
\newcommand{\functionImpactTransformationSignature}[2]{
	\ifx \\#1\\
	\functionSignature{IT}
	{\varX{}{}}
	\else
	\functionSignature{IT}
	{#1}
	\fi
}
%%%%%%%%%%%%%%%%%%%% NOTATION %%%%%%%%%%%%%%%%%%%%
\newcommand{\varExplore}[2]{\varSymbol{\epsilon}{#1}{#2}}
\newcommand{\varRiskAction}[2]{\varSymbol{\texttt{w}}{#1}{#2}}
%%%%%%%%%%%%%%%%%%%%%%%%%%%%%%%%%%%%%%%%%%%%%%%%%%



%%%%%%%%%%%%%%%%%%%%%%%%%%%%%%%%%%%
\newcommand{\functionActionAllocSignature}[2]{ALLOC(g, at, n)}
\newcommand{\functionActionExecSignature}[2]{EXEC(g, at)}
\newcommand{\functionActionInfoSignature}[2]{INFO(g, t, n)}
\newcommand{\functionActionLinkSignature}[2]{LINK(g, k)}
%%%%%%%%%%%%%%%%%%%%%%%%%%%%%%%%%%%


%%%%%%%%%%%%%
\newcommand{\setAgentActions}[2]{\setAction{\varAgent{}{}}{}}
\newcommand{\functionAgentActionType}[2]{
	\ifx \\#1\\
	\functionSignature{category}{\varAction{}{}}
	\else
	\functionSignature{category}{#1}
	\fi
}
\newcommand{\varActionType}[2]{\functionAgentActionType{\varAction{}{}}{}}
\newcommand{\setActionType}[2]{\functionAgentActionType{\setAction{}{}}{}}

%%%%%%%%%%%%%%%


%%%%%%%%%%%%%%%%%%%% NOTATION %%%%%%%%%%%%%%%%%%%%
\newcommand{\varActionSample}[2]{\varSymbol{\psi}{#1}{#2}}
\newcommand{\varSampleTime}[2]{\varSymbol{t}{#1}{#2}}
\newcommand{\varSampleValue}[2]{\varAtomicTaskQualityValue{#1}{#2}}
\newcommand{\setActionSample}[2]{\setSymbol{\Psi}{#1}{#2}}
\newcommand{\functionActionSampleSelectorSignature}[2]{
	\ifx \\#1\\
	\functionSignature{S}
	{\setActionSample{}{}, \setAction{}{}}
	\else
	\functionSignature{S}
	{#1}
	\fi
}
\newcommand{\setAgentActionSamples}[2]{
	\setActionSample{\varAgent{}{}}{}
}
\newcommand{\functionLastActionSampleSignature}[2]{
	\ifx \\#1\\
	\functionSignature{LS}{
		\setActionSample{}{}, \setAction{}{}
	}{}
	\else
	\functionSignature{LS}{#1}
	\fi
}
%%%%%%%%%%%%%%%%%%%% NOTATION %%%%%%%%%%%%%%%%%%%%

\newcommand{\functionUncertainInformationSymbol}[2]{\functionSymbol{MV}{#1}{#2}}

\newcommand{\functionUncertainInformationSignature}{
	\functionSignature{\functionUncertainInformationSymbol{}{}}
	{
		\setActionSample{}{}, \varAction{}{}, \varSampleTime{}{}
	}
}

\newcommand{\varUncertainInformationThreshold}[2]{
	\varSymbol{\hat\mu}{\texttt{min}}{#2}
}
%%%%%%%%%%%%%%%%%%%% NOTATION %%%%%%%%%%%%%%%%%%%%
\newcommand{\functionNeighbourhoodQualitySignature}[2]{
	\ifx \\#1\\
	\functionSignature{NQ}{
		\setActionSample{}{}, \varAgent{}{}, \setAgent{}{}
	}
	\else
	\functionSignature{NQ}{#1}
	\fi
}

%%%%%%%%%%%%%%%%%%%% NOTATION %%%%%%%%%%%%%%%%%%%%
\newcommand{\functionMQNSignature}[2]{
	\ifx \\#1\\
	\functionSignature{MQN}
	{\setActionSample{}{}, \varAgent{}{}}
	\else
	\functionSignature{MQN}
	{#1}
	\fi
}
%%%%%%%%%%%%%%%%%%%%%%%%%%%%%%%%%%%%%%%%%%%%%%%%%%

%%%%%%%%%%%%%%%%%%%% NOTATION %%%%%%%%%%%%%%%%%%%%
\newcommand{\varQ}[2]{\varSymbol{{q}}{#1}{#2}}
\newcommand{\setQ}[2]{\setSymbol{{Q}}{#1}{#2}}
\newcommand{\tupleQ}[2]{
	(\varAction{}{}, \varQ{}{})
}
%%%%%%%%%%%%%%%%%%%%%%%%%%%%%%%%%%%%%%%%%%%%%%%%%%
%%%%%%%%%%%%%%%%%%%% NOTATION %%%%%%%%%%%%%%%%%%%%
\newcommand{\functionQMappingSignature}[2]{
	\ifx \\#1\\
	\functionSignature{QM}
	{\setAgent{}{}, \setAtomicTaskType{#1}{}}
	\else
	\functionSignature{QM}
	{#1}
	\fi
}
\newcommand{\formalQMapping}[2]{
	\functionFormal{qm}{\setAgent{}{} \times \setAtomicTaskType{}{}}{\setQ{}{}}
}
\newcommand{\functionQMappingIndexedSignature}[3]{
	\ifx \\#1\\
	\functionSignature{QM}
	{\setAgent{}{}, \setAtomicTaskTypeUnallocated{#1}{}}
	\else
	\functionSignature{QM_{#3}}
	{#1}
	\fi
}
\newcommand{\functionInstanceQMappingSignature}[2]{
	\functionQMappingSignature
	{\varAgent{}{}, \setAtomicTaskTypeUnallocated{#1}{}}{}
}

\newcommand{\functionInstanceQMappingIndexedSignature}[3]{
	\functionQMappingIndexedSignature
	{\varAgent{}{}, \setAtomicTaskTypeUnallocated{#1}{}}
	{}
	{#3}
}
%%%%%%%%%%%%%%%%%%%% NOTATION %%%%%%%%%%%%%%%%%%%%
\newcommand{\setAvailableActions}[2]{\setSymbol{A}{#1}{\oplus}}
\newcommand{\setUnavailableActions}[1]{\setSymbol{A}{#1}{\ominus}}
\newcommand{\setAvailableQ}[2]{\setSymbol{{Q}}{#1}{\oplus}}
\newcommand{\setUnavailableQ}[2]{\setSymbol{{Q}}{#1}{\ominus}}
%%%%%%%%%%%%%%%%%%%%%%%%%%%%%%%%%%%%%%%%%%%%%%%%%%
%%%%%%%%%%%%%%%%%%%% NOTATION %%%%%%%%%%%%%%%%%%%%

%%%%%%%%%%%%%%%%%%%%%%%%%%%%%%%%%%%%%%%%%%%%%%%%%%
%%%%%%%%%%%%%%%%%%%% NOTATION %%%%%%%%%%%%%%%%%%%%
\newcommand{\setAtomicTaskUnallocated}[2]{\setSymbolMinus{AT}{#1}{#2}}
\newcommand{\setAtomicTaskTypeUnallocated}[2]{
	\functionAtomicTaskMapping{\setAtomicTaskUnallocated{}{}}{}
}
\newcommand{\functionSymbolTD}[2]{
	\functionSymbol{TD}
}
\newcommand{\functionTDUpdateSignature}[2]{
	\functionSignature{\functionSymbolTD{}{}}
	{
		\varAgent{}{}, \setAtomicTaskTypeUnallocated{t}{},
		\setAtomicTaskTypeUnallocated{t+1}{},
		\varAtomicTaskQualityValue{}{},
		\varLearningRate{}{}, \varDiscountFactor{}{}
	}
}
\newcommand{\varDiscountFactor}[2]{\lambda}

\newcommand{\varLearningRate}[2]{\alpha}
\newcommand{\functionTDUpdate}{
	\functionInstanceQMappingIndexedSignature{}{}{t}
	\funcupdate
	(1 - \varLearningRate{}{})\underbrace{
		\functionInstanceQMappingIndexedSignature{}{}{t}}_{
		\texttt{current}
	} + \varLearningRate{}{} \overbrace{
		\lbrack \varAtomicTaskQualityValue{}{} + \varDiscountFactor{}{} \underbrace{
			\texttt{max}_{\varAction{}{}}\ \functionInstanceQMappingIndexedSignature{}{}{t+1}
		}_{
			\texttt{future estimate}} \rbrack}^{\texttt{learned value}
	}
}
%%%%%%%%%%%%%%%%%%%%%%%%%%%%%%%%%%%%%%%%%%%%%%%%%%

%%%%%%%%%%%%%%%%%%%% NOTATION %%%%%%%%%%%%%%%%%%%%
\newcommand{\functionSASSignature}[2]{
	\text{\acronymMemoryRetention{}} (\setAtomicTaskUnallocated{}{},
	\functionNeighbourhoodSignature{}{}, \functionKnowledgeSignature{}{},
	\setAgentActionSamples{}{}, \varUncertainInformationThreshold{max}{})
}
\newcommand{\functionSAS}[2]{
	\functionKnowledgeSignature{}{} \leftarrow \funcSAS{}{}
}

\newcommand{\functionNPruneSignature}{
	\acronymNeighbourhoodPruningAlgorithm(
	\functionNeighbourhoodSignature{}{},
	\setActionSample{}{}
	)
}

\newcommand{\functionNPrune}{
	\functionNeighbourhoodSignature{}{} \leftarrow \functionNPruneSignature{}{}
}

\newcommand{\functionRATSignature}[2]{
	\text{\acronymRewardTrends{}}(\setAtomicTaskTypeUnallocated{}{}, \setRiskAction{}{}, \setRewardSet{}{}, \varExplore{base}{})
}
\newcommand{\functionRAT}[2]{
	\varAction{}{} \leftarrow \functionRATSignature{}{}
}

\newcommand{\functionRTARP}{
	\functionFormal {\acronymRewardTrends{}{} }
	{
		(\setAgent{}{}, \setCompositeTaskTypeUnallocated{}{}, \setRiskAction{}{}, \setRewardSet{}{})
	}
	{
		\setAction {} {}
	}
}

%%%%%%%%%%%%%%%%%%%% NOTATION %%%%%%%%%%%%%%%%%%%%
\newcommand{\varNeighbour}[2]{
	\varSymbol{n}{}{}
}
\newcommand{\defVarDiscountFactor}[3]{$\varDiscountFactor{#1}{#2}$, a value $\setRealNumbersPositiveUnit{}{}$, weighting importance of future rewards#3}
\newcommand{\defVarLearningRate}[3]{$\varLearningRate{#1}{#2}$, a value $\setRealNumbersPositiveUnit{}{}$, weighting the rate of Q-value update#3}
\newcommand{\defSetRiskAction}[3]{$\setRiskAction{#1}{#2}$, The potential change on neighbourhoods on taking an action#3.}
%%%%%%%%%%%%%%%%%%%%%%%%%%%%%%%%%%%%%%%%%%%%%%%%%%

%%%%%%%%%%%%%%%%%%%% NOTATION %%%%%%%%%%%%%%%%%%%%
\newcommand{\varExploreBase}[2]{
	\varExplore{\texttt{base}}{}
}
\newcommand{\varExploreImpact}[2]{
	\varExplore{\texttt{ief}}{}
}
\newcommand{\setQTransformed}[2]{
	\setQ{tr}{}
}
\newcommand{\acronymInformationRetentionThreshold}{information retention threshold}
\newcommand{\defVarUncertainInformationThreshold}[3]{ $\varUncertainInformationThreshold{}{}$,  The \acronymInformationRetentionThreshold#3.}
%%%%%%%%%%%%%%%%%%%%%%%%%%%%%%%%%%%%%%%%%%%%%%%%%%

%%%%%%%%%%%%%%%%%%%% NOTATION %%%%%%%%%%%%%%%%%%%%
\newcommand{\functionAtomicAllocationSymbol}[1]{\functionSignature{al}{#1}}
\newcommand{\formalAtomicAllocation}[2]{
	\functionFormal{\functionAtomicAllocationSymbol{}{}}
	{\setAtomicTask{}{} \times  \setAgent{}{}}
	{\powerSetSymbol{\setAtomicTask{}{} \times \setAgent{}{}}{}{}}
}
\newcommand{\functionAtomicAllocationSignature}[2]{
	\ifx \\#1\\
	\functionAtomicAllocationSymbol{\setAtomicTask{}{}, \setAgent{}{}}{}
	\else
	\functionAtomicAllocationSymbol{#1}{}
	\fi
}
\newcommand{\functionAtomicAllocationSubsetSignature}[2]{
	\ifx \\#1\\
	\functionAtomicAllocationSymbol
	{\setAtomicTaskSubset{}{}, \setAgentSubset{}{}}
	\else
	\functionAtomicAllocationSymbol
	{#1}
	\fi
}
\newcommand{\functionInstanceAtomicAllocationSignature}[2]{
	\ifx \\#1\\
	\functionAtomicAllocationSymbol
	{\varAtomicTask{}{}, \varAgent{}{}}
	\else
	\functionAtomicAllocationSymbol
	{#1}
	\fi
}
\newcommand{\functionAtomicAllocationIndexedSignature}[2]{
	\ifx \\#2\\
	\functionSignature{al_{#1}}
	{\varAtomicTask{#1}{}, \setAgent{#1}{}}
	\else
	\functionSignature{al_{#1}}
	{#2}
	\fi
}
\newcommand{\functionAtomicAllocationOptimalNeighbourhoodSignature}[2]{
	\functionSignature
	{\optimal{al}}
	{\setAtomicTask{}{}, \functionNeighbourhoodSignature{}{}}
}

\newcommand{\functionJointSystemAllocationSignature}[2]{
	\functionSignature{JL}{\setAtomicTaskSys{}{}, \setAgentSys}
}
\newcommand{\functionAgentJointAllocationSignature}[2]{
	\functionSignature{JL}{\setAtomicTask{}{}, \varAgent{}{}}
}


\newcommand{\setConcurrentAllocations}{
	\funcSize{\setJointSystemAllocation{\varAgent{}{}}{}
	}
}
%%%%%%%%%%%%%%%%%%%% NOTATION %%%%%%%%%%%%%%%%%%%%
\newcommand{\varAtomicTaskQualityValue}[2]{\varSymbol{\omega}{#1}{#2}}
\newcommand{\functionSymbolAtomicQuality}[2]{\functionSymbol{\omega_g}{}{#2}}

\newcommand{\formalAtomicTaskQuality}[2]{
	\functionFormal{\functionSymbolAtomicQuality{}{}}
	{\setAtomicTaskType{}{} \times \setNaturalNumbers{}{}}
	{\setRealNumbersNonNegative{}{}}
}
\newcommand{\functionJointAtomicQualitySignature}[2]{
	\functionSignature{jq}{\setAtomicTask{}{}, \setAgent{}{}}
}
\newcommand{\functionQLSignature}[2]{
	\functionSignature{QL}
	{\setAtomicTask{}{}, \setAgent{}{}, \setJointSystemAllocation{}{}}
}
\newcommand{\functionQLSubsetSignature}[2]{
	\functionSignature{QL}
	{\setAtomicTask{}{}, \setAgentSubset{}{}, \setJointSystemAllocation{}{}}
}
\newcommand{\functionQLSingleSignature}[2]{
	\functionSignature{QL}
	{
		\functionAtomicAllocationIndexedSignature{}{\varAtomicTask{}{}, \varAgent{}{}}, \setJointSystemAllocation{}{}
	}
}

\newcommand{\varSystemUtility}{u}

%%%%%%%%%%%%%%%%%%%%%%%%%%%%%%%%%%%%%%%%%%%%%%%%%%



	\newcommand{\matrixSymbol}[3] {
	\ifx\\#2\\\ifx\\#3\\\textbf{\uppercase{#1}}
	\else
	\uppercase{\textbf{#1}}^{#3}
	\fi
	\else
	\ifx\\#3\\\uppercase{\textbf{#1}}_{#2}
	\else
	\uppercase{\textbf{#1}}_{#2}^{#3}
	\fi
	\fi
}





\newcommand{\varSystem}{s}
%\newcommand{\varParentAgent}[2]{\ifx \\#1\\ pg \else pg_{#1}^{#2} \fi}
%\newcommand{\setParentAgent}[2]{\ifx \\#1\\ PG \else PG_{#1}^{#2} \fi}
%\newcommand{\varChildAgent}[2]{\ifx \\#1\\ cg \else cg_{#1}^{#2} \fi}
%\newcommand{\setChildAgent}[2]{\ifx \\#1\\ CG \else CG_{#1}^{#2} \fi}
\newcommand{\powerSetParentAgent}{2^{PG}}
%\newcommand{\varAtomicTaskType}[2]{\ifx \\#1\\ tp \else tp_{#1}^{#2} \fi}
%\newcommand{\setAtomicTaskType}[2]{\ifx \\#1\\ TP \else TP_{#1}^{#2} \fi}
%\newcommand{\varCompositeTaskType}[2]{\ifx \\#1\\ \hat{tp} \else \hat{tp}_{#1}^{#2} \fi}
%\newcommand{\setCompositeTaskType}[2]{\ifx \\#1\\ \hat{TP} \else \hat{TP}_{#1}^{#2} \fi}
\newcommand{\powerSetTaskTypes}{2^{\setAtomicTaskType{}{}}}
\newcommand{\varCapabilityMap}{q}
\newcommand{\varAtomicTaskQuality}[2]{\varCapabilityMap_{\varChildAgent{}{}}}

\newcommand{\varTaskGroupMap}{tg}
%\newcommand{\varAtomicTask}{t}
%\newcommand{\setAtomicTask}{T}
\newcommand{\powerSetTasks}{2^T}
\newcommand{\varAtomicTaskInstanceDetails}{\pi}
%\newcommand{\varCompositeTask}{\hat{t}}
%\newcommand{\setCompositeTask}{\hat{T}}
\newcommand{\atomicTaskTypeFunction}{type_a}
\newcommand{\compositeTaskTypeFunction}{type_c}
\newcommand{\varFrequencyFunction}{tf}
%\newcommand{\varResource}[2]{\ifx \\#1\\ r \else r_{#1}^{#2} \fi}
%\newcommand{\setResource}{R}
\newcommand{\varResourceMap}{ar}
\newcommand{\setResourceAssignment}{2^{\setResource{}{} \times \mathbb{R}_{>=0}}}
\newcommand{\varTaskAllocation}{tl}
\newcommand{\setPossibleAllocation}{2^{\setChildAgent{}{} \times \setAtomicTask{}{}}}
\newcommand{\varComponentTasksResult}{ctr}
\newcommand{\powersetTaskResult}{2^{\setAtomicTask{}{} \times \mathbb{R}}}
\newcommand{\varTaskAllocationQuality}{taq}
\newcommand{\varComponentTasksValue}{ctv}
\newcommand{\varResourceWeighting}[2]{\ifx \\#1\\ w \else w_{#1}^{#2} \fi}
\newcommand{\setResourceWeighting}{W}
\newcommand{\varResourceAllocation}{ra}
\newcommand{\setResourceAllocation}{RA}
\newcommand{\sysResourceAllocation}{sra}
%\newcommand{\varTime}{\phi}
%\newcommand{\setTime}{\Phi}
\newcommand{\absoluteTaskValue}{atv}
\newcommand{\functionAbsoluteTaskValue}[2]{
	\functionSignature{\absoluteTaskValue{}{}}{\varCompositeTask{#1}{}, \varAtomicTask{#2}{}}
}
\newcommand{\formalResourceAllocation}[2]{
	\functionFormal{\varResourceAllocation_{\varChildAgent{}{}}{}}{\setAtomicTask{}{}}{\setResourceAssignment{}{}}
}
\newcommand{\functionResourceAllocation}[2]{
	\functionSignature{\varResourceAllocation_{\varChildAgent{}{}}{}}
	{\varAtomicTask{}{}}
}
\newcommand{\formalResourceMap}[2]{
	\functionFormal{\varResourceMap{}{}}
	{\setChildAgent{}{} \times \setResource{}{}}
	{\setRealNumbersNonNegative{}{}}
}
\newcommand{\functionResourceMap}[2]{
	\functionSignature{\varResourceMap{}{}}
	{\varChildAgent{}{}, \varResource{}{}}	
}


%%%%%%%%%%%%%%%%%%%
\newcommand{\formalPAG}{pag_{\varChildAgent{}{}}}
\newcommand{\functionPAG}[2]{
	\ifx \\#1\\
	\formalPAG(\varParentAgent{}{})
	\else
	\formalPAG(#1)
	\fi
}
\newcommand{\functionPAGSet}[2]{
	\ifx \\#1\\
	\formalPAG(\setParentAgent{}{})
	\else
	\formalPAG(\setParentAgent{#1}{#2})
	\fi
}

\newcommand{\formalPGroup}[2]{
	\functionFormal{pgroup_{\varChildAgent{}{}}}{\setParentAgent{}{}}{\powerSetParentAgent{}{}}
}
\newcommand{\functionPGroup}[2]{
	\ifx \\#1\\
	\functionSignature{pgroup_{\varChildAgent{}{}}}{\varParentAgent{}{}}
	\else
	\functionSignature{pgroup_{\varChildAgent{}{}}}{#1}
	\fi
}
\newcommand{\functionPGroupSet}[2]{
	\functionPGroup{\setParentAgent{}{}}{}
}
\newcommand{\formalPAGResourceWeighting}{
	\functionFormal
	{pgw_{\varChildAgent{}{}, \varResource{}{}, \varTime{}{}}}
	{\functionPGroupSet{}{} \times \setAtomicTaskType{}{}}
	{ \mathbb{R}}
}

\newcommand{\functionPAGResourceWeighting}[2]{
	\functionSignature
	{pgw_{\varChildAgent{}{}, \varResource{}{}, \varTime{}{}}}
	{\setParentAgent{#1}{}, \setAtomicTaskType{#2}{}}
}
\newcommand{\setResourceWeight}[2]{
	\setSymbol{W}{\varChildAgent{}{}, \varResource{}{}}{#2}
}
\newcommand{\matrixResourceWeight}[2]{
	\matrixSymbol{PW}{\varChildAgent{}{}, \varResource{}{}}{#2}
}
\newcommand{\varPAGResourceWeight}[2]{
	{pw}_{\setParentAgent{#1}{}, \varAtomicTaskType{#2}{}}
}
\newcommand{\setPAGResourceWeight}[2]{
	{pw}_{\setParentAgent{#1}{}, \setAtomicTaskType{#2}{}}
}
\newcommand{\varParentTaskIndex}{pti_{\varChildAgent{}{}}}
\newcommand{\formalFunctionParentTaskIndex}{
	\functionFormal{\varParentTaskIndex{}{}}
	{\setParentAgent{}{} \times \setAtomicTaskType{}{}}
	{\setIntegersPositive{}{} \times \setIntegersPositive{}{}}
}
\newcommand{\functionParentTaskIndex}{
	\functionSignature
	{\varParentTaskIndex{}{}}
	{\varParentAgent{}{}, \varAtomicTaskType{}{}}
}

%%%%%%%%% NOTATION %%%%%%%%%%
\newcommand{\formalPAGSampleCount}{
	\functionFormal{pags_{\varChildAgent{}{}, \varTime{}{}}{}}
	{\setParentAgent{}{}}
	{\setIntegersNonNegative{}{}}
}
\newcommand{\functionPAGSampleCount}[2]{
	\ifx \\#1\\
	\functionSignature{pags_{\varChildAgent{}{},\varTime{}{}}{}}
	{\setParentAgent{}{}}
	\else
	\functionSignature{pags_{\varChildAgent{}{},\varTime{}{}}{}}
	{#1}
	\fi
}
%%%%%%%%% NOTATION %%%%%%%%%%
\newcommand{\varKullbackLiebler}{dkl}
\newcommand{\functionKullbackLiebler}[2]{\varKullbackLiebler(\setParentAgent{#1}{})}
%%%%%%%%% NOTATION %%%%%%%%%%
\newcommand{\setWeightBlending}{B_{\varChildAgent{}{}}}

\providecommand{\functionSumNormSymbol}[2]{\texttt{norm}{#1}{#2}}
\providecommand{\functionSumNormSignature}[2]{
	\ifx \\#1\\
	\functionSignature{\functionSumNormSymbol{}{}}
	{\setX{}{}}
	\else
	\functionSignature{\functionSumNormSymbol{}{}}
	{#1}
}
\newcommand{\functionSumNormRowSymbol}[2]{\texttt{norm}{#1}{#2}}
\newcommand{\functionSoftmaxSymbol}[2]{\varSymbol{\sigma}{#1}{#2}}
\newcommand{\functionSoftmaxSignature}[2]{
	\ifx \\#1\\
	\functionSignature{\functionSoftmaxSymbol{}{}}
	{\varX{}{}}
	\else
	\functionSignature{\functionSoftmaxSymbol{}{}}
	{#1}
}
\newcommand{\functionSoftmaxSet}[2]{
	\sigma(X) = \lbrace \frac{e^{x_i}}{\sum_{j=1}^{\funcSize{X}} e^{x_j}}\rbrace_{\forall x_i \in X}
}

\newcommand{\varCombinedResourceWeights}{crw}
\newcommand{\formalFunctionCombinedResourceWeights}{crw_{\varChildAgent{}{}}}
\newcommand{\formalFunctionCombinedResourceWeightsSignature}{\formalFunctionCombinedResourceWeights(\setWeightBlending{}{}, \matrixResourceWeight{}{})}
\newcommand{\functionCombinedResourceWeightsSignature}{\formalFunctionCombinedResourceWeights(\setWeightBlending{}{}, \matrixResourceWeight{}{})}
\newcommand{\matrixCombinedResourceWeights}{\matrixSymbol{C}}
\newcommand{\varEligibilityTrace}[2]{\varSymbol{e}{#1}{#2}}
\newcommand{\matrixEligibilityTrace}[2]{\matrixSymbol{E}{\varChildAgent{}{}}{#2}}
\newcommand{\functionEligibilityTraceUpdateSignature}[2]{
	\functionSignature{etu}
	{
		\matrixEligibilityTrace{}{}, \varParentAgent{}{},
		\varAtomicTaskType{}{},
		\gamma
	}
}
%%%%%%%%%%%%%%%%%%%%%%%%%%%%%
%%%%%%%%%%%%%%%%%%%%%%%%%%%%%
%%%%%%%%%%%%%%%%%%%%%%%%%%%%%
	\newtheorem{thm}{Theorem}
	\newdefinition{definition}{Definition}
	\maketitle
	\ifdefined\DEBUG  \listoftodos \else \fi
	\section{Introduction}

\todo[inline]{Why is the subject/problem area important?}
\textit{Wireless sensor networks (WSNs)} have many applications and research studies in areas such as environmental monitoring, agriculture, and military uses (See Table \ref{table:applications}). More recently, the availability and lower cost of low-power wireless transmitters \citep{902661}, solar-harvesting components \citep{Prauzek2018}, and micro-electro-mechanical systems \citep{1045391} has allowed large deployments sizes and scope of use, expanding their real-world use and opening up new areas for practical research \citep{Kandris2020}.



\todo[inline]{What are the current solutions, what are the problems and how are we improving them?}
\textit{centralised}\\
\todo[inline]{Quick centralised reference, this is non-scalable and not robust to hard conditions?}
\textit{decentralised}\\
\todo[inline]{Quick decentralised overview, this gives us resilience but we must have autonomous behaviour. Now we have problems in working from local information and coordination.}
Broadly there are three main categories for decentralisation \citep{10.1007/978-3-642-11814-2_4, 10.1504/IJCNDS.2012.048871},
\begin{itemize}
	\item \textit{Clustering}
	\item \textit{Hierarchical}
	\item \textit{Reinforcement learning}
\end{itemize}

\todo[inline]{What is the solution and contributions we present?}
The solution we present here is based on the following algorithms previously developed by the authors. We use the \acronymATARIA{}{} algorithm to optimise the task of measurements and coverage, minimise the energy consumption of the network, while adapting to the dynamic nature of WSNs \citep{creech2021dynamic}. Through the \acronymMGRAO{}{} algorithm we enable sensors that are taking measurements to optimise the allocation of their resources to meet the overall system goal \citep{creech2021resource}. By combining and evaluating these algorithms in a simulated WSN deployed in a realistic environment, we show that the overall solution can be successfully utilised to balance the systems' multiple objectives of minimising energy consumption, maximising system lifetime, as well as optimising the quality of the measurement tasks while still maintaining geographical coverage.

In Section \ref{section:background} we look at related research in this area, allowing us to concretely define the problem in Section \ref{section:problem}. Section \ref{section:solution} sets out our solution, followed by definition of the simulated environment and evaluation of the solution in Section \ref{section:experimental}. We close with the summary of our conclusions and future work in Section \ref{section:conclusions}.


\begin{table}
	\footnotesize
	\begin{tabular}{|p{0.2\textwidth}|p{0.2\textwidth}|p{0.5\textwidth}|}
		\hline
		Area & References & Summary \\
		\hline
		Ocean monitoring and the marine environment & \cite{Mahdy2008a, Albaladejo2010, 6973877} & XXX \\
		Radiation contamination & \cite{Gomez2015} & XXX \\
		Water quality & \cite{Fang2010} & XXX \\
		Agriculture  & \cite{8745854} & XXX \\
		Volcano monitoring  & \cite{Werner-Allen2006} & XXX \\
		Flood monitoring  & \cite{Castillo-effen2004} & XXX \\
		Military & \cite{6268958} & XXX \\
		\hline
	\end{tabular}
	\caption{Real-world applications of wireless sensor networks}
	\label{table:applications}	
\end{table}

	\section{Background}
\label{section:background}
Lorem ipsum dolor sit amet, consectetur adipiscing elit, sed do eiusmod tempor incididunt ut labore et dolore magna aliqua. Ut enim ad minim veniam, quis nostrud exercitation ullamco laboris nisi ut aliquip ex ea commodo consequat. Duis aute irure dolor in reprehenderit in voluptate velit esse cillum dolore eu fugiat nulla pariatur. Excepteur sint occaecat cupidatat non proident, sunt in culpa qui officia deserunt mollit anim id est laborum.

\todo[inline]{REFERENCES}
\begin{itemize}
	\item \cite{Sengupta2013}: Multi-objective node deployment in WSNs: In search of an optimal trade-off among coverage, lifetime, energy consumption, and connectivity 
	
	\item \cite{Mahdy2008}: Marine wireless sensor networks: Challenges and applications
	
	\item \cite{Islam12}: Overview of Wireless Sensor Network Security Technology
	
	\item  \cite{doi:10.1155/2014/765182}: Performance Analysis of Resource-Aware Task Scheduling Methods in Wireless Sensor Networks
	
	\item\cite{Xu2014f}: Applications of wireless sensor networks in marine environment monitoring: A survey
	
	\item \cite{Prauzek2018}: Energy harvesting sources, storage devices and system topologies for environmental wireless sensor networks: A review
	
	\item \cite{Akyildiz}: Wireless sensor networks: A survey
	
	\item \cite{Albaladejo2010}: Wireless sensor networks for oceanographic monitoring: A systematic review
	
	\item \cite{Oliveira2011}: Wireless sensor networks: A survey on environmental monitoring
	
\end{itemize}

	%\section{Background}
\label{section:background}

\todo{This section needs to be pulled out from a journal paper as we do lots of comparisons to the ATARIA approach}
\subsection{Sampling in contaminated radiation-zones}
\begin{figure}
	\centering
	\includegraphics[width=0.5\linewidth]{chernobyl_plant_map}
	\caption{Chernobyl - Map of the plant and colling ponds. The Chernobyl plant and cooling pond are shown in this overhead satellite image captured 10 years after the initial accident}
	\label{fig:chernobylplantmap}
\end{figure}



{\begin{figure}
		\centering
		\includegraphics[width=0.5\linewidth]{chernobyl_caesium_137}
		\caption{Chernobyl - Map of caesium-137. The highly radioactive isotope caesium-137 has a medium-term lifespan, with a half-life of around 30 years. However, the damage it causes environmentally is magnified by the ease at which is spreads in the environment due to its water solubility}
		\label{fig:chernobylcaesium137}
	\end{figure}
}{}
\begin{figure}
	\centering
	\includegraphics[width=0.5\linewidth]{chernobyl_restricted_zones}
	\caption{Chernobyl - Map of restricted zones - Due to contamination there were exclusion zones set up around the Chernobyl plant over substantial tracts of land in the Ukraine, into Belarus, and affecting as far away as Russia}
	\label{fig:chernobylrestrictedzones}
\end{figure}

\begin{figure}
	\centering
	\includegraphics[width=0.7\linewidth]{WSN_simulation_standard_algorithms}
	\caption{WSN standard algorithms - Some standard algorithms for networking in WSN. LEACH uses cluster of nodes each with a lead node for data aggregation and coordination which is rotated periodically to balance power usage. PEGASUS uses chains of nodes with one lead node communicating to the base station.}
	\label{fig:wsnsimulationstandardalgorithms}
\end{figure}

In the event of an incident at a nuclear facility there is a risk that there can be leaks of contaminated radioactive material. This can occur as an atmospheric release or as water that has passed through the cooling system escapes directly into the ground water around the facility instead of being captured in storage ponds for decontamination. In the case of a serious meltdown event such as Chernobyl \cite{Steinhauser2014, Evangeliou2016, Konoplev1992, Fesenko2007, Kortov2013} shown in Figure \ref{fig:chernobylplantmap} we can see a large release of radioactive material into the atmosphere which is then spread globally by winds, contaminating large swathes of the local geographical area, europe, and beyond. We also see this in the many events doumented at the troubled Indian Point plant\cite{TheGuardian} and others where atmospheric releases and ground water contimation have happened repeatedly to varying scales over many years.                                                                                                                                                                                                                                                                                                                                                                                                                                                                                
\newline
\newline


In Chernobyl, lethal levels of radiation were spread over large expanses of remote terrain and have persisted for decades, the land remaining sealed off and uninhabited for over 30 years. These levels mean that humans cannot spend significant time inside the contamination zone without risking rapidly succumbing to the effects of radiation poisoning. Figures \ref{fig:chernobylcaesium137} and \ref{fig:chernobylrestrictedzones} show the contamination areas resulting from the release of the radioactive isotope caesium-137 and corresponding human exclusion zones. Situations like this are unfortunately not that uncommon, although not frequently at such a devastating scale as the Chernobyl or Fukishima disasters. The environmental degradation, essential monitoring characteristics, and harsh external conditions for both humans and machines shown in these cases will prove a fitting and practical testbed for a sensor network deployment based around our agent system.
\newline
\newline


Wide area radioactive contamination gives us a real-world condition that draws out the reasons behind some of the assumptions and criteria used to drive the agent systems learning and operation. In particular, there can be little to no human involvement in deployment or maintenance of sensors deployed to such an environment, the risk to life is simply too large. With this in mind, deployment of sensors is likely to be at-a-distance, and in being so, will have relatively ad-hoc placement. For example, sensors dropped from \textit{Unmanned Aerial Vehicles (UAV)} with no ability to place or move the sensors once released. With a contamination event we may also require sensor readings for many decades, how the adaptation and resilience of an agent-based sensor depoyment can provide for that would prove extremely beneficial to its viability as a solution. Lastly we will see how we can utilise the learning-guided inter-agent linking of the system to give us a robust and resilient networking capability. In addition, dependent on our strategy we can target optimisation of energy utilisation for these low-power remote sensor systems, or should more accurate data be required we can adapt to preferring a greater amount of more granular aggregation of data but at a higher energy cost. 




\subsection{Networking and routing in E-WSN}

In real world deployments it has shown to have been key to balance network robusiness and resilience to damage, alongside the demanding nature of the low energy usage requirements, in order to make the deployments practical, maintainable, and cost-effective. Aside from standard star and peer-to-peer network topologies \cite{Oliveira2011} there have been efforts to develop more targeted architectures based on experience of past WSN \ref{Singh2017, BaniHani, Mahapatra2015, MdZin2014}. Figure \ref{fig:wsnsimulationstandardalgorithms} shows two such algorithms illustrating generalised groupings of these approaches. LEACH uses the idea of sub-clusters of nodes with a nominated lead node that acts as the co-ordintaor and broadcaster of the data back to a base station. These lead nodes are on-hop in that they broadcast directly to the base station, the energy drain caused by this is mitigated by occasionally rotating the lead nodes within the cluster. PEGASUS shows the other main approach to networking with one node acting as the lead node to orchestrate communication with the base station, but there is a multi-hop chain of sub-nodes that relay data through the system using a nearest-neighbour strategy for data request and return. Should a node in the chain go down then the chain will align itself to reform a complete chain back to the base station again. 
\newline
\newline


Comparing these standard algorithms back to the agent system we can see both the similarities with both approaches, but also substantial benefits of the agent-based learning enhancements. The hierarchical approach to agent networking is similar to that we take here, and has been investigated for holoarchy-style agent learning systems such as that of Holmesparker et al \ref{Holmesparker}. The agent system we have set out has both clusters and chains of nodes and clusters as a natural outcome of its operation, with both single-hop and multi-hop both possible outcomes of the structure development through learning links and channels. The networking behaviour is shaped by the definition of local neighbourhood learning for agents and is guided by the higher-level information conveyed by rewards rather than any explicit definitions of networking or agent-system design patterns. While a hierarchical design with lead-agents learning to guide teams of other agents may bear some resemblance to our neighbourhood learning based system, in practice the behaviours within our system are flexible and self-organised to a much greater extent. Scalability itself is incorporated automatically as a direct outcome of placing constraints on neighbourhood formations.
\newline
\newline
\begin{figure}
	\centering
	\includegraphics[width=0.7\linewidth]{WSN_hierarchical_resolution}
	\caption{Agent system hierarchical neighbourhood resolution - This diagram shows one of the possible modes of hierarchical networking that can be formed by the agent system through per-agent local neighbourhood learning. The initial agent passes requests through smaller and smaller neighbourhoods until the granulaity of the data meets that of the request}
	\label{fig:wsnhierarchicalresolution}
\end{figure}
\begin{figure}
	\centering
	\includegraphics[width=0.7\linewidth]{WSN_simulation_map_failure_adaptation}
	\caption{WSN simulation failure adaptation - The first figure shows a set of agents aggregating data across a defined granularity geographical area using their learned neighbourhoods. In the second we see the adaptation of each agents local neighbourhood as they individually react to device failures}
	\label{fig:wsnsimulationmapfailureadaptation}
\end{figure}

The idea of nearest-neighbour is also applied, but in a more flexible manner and at a higher-level abstraction. While there are definite physical and geographical consequences of deployment such as the range of the transceiver and occlusion behind objects, the agents learn routes with a much wider range of information. Elements such as transfer rate and reliability are large factors in the shape of the reward signals received by agents, but the signal for local neighbourhood formation contains a multitude of other factors and can also be shaped to fit the desired application. So congestion, data consistency, energy cost for transmission and many other factors could be used as elements of the reward heuristic, giving the agents local neighbourhood the shape that best meets with our particular demands on a situational basis. Figure \ref{fig:wsnhierarchicalresolution} illustrates how the agent system works with hierarchical clusters of local neighbourhoods. In reality, the diagram masks the dynamic nature of these neighbourhoods for each agent. Additionally the route through the network of any requests is heavily dependent on the agent that receives an initial request as its neighbourhood will heavily influence the subsequent paths of the requests.
\newline
\newline
With our approach of neighbourhood learning and adaptation for agents we automatically get networking recovery in the event of device failure. Figure \ref{fig:wsn_simulation_failure_adaptation} shows a simplified version of how this might apply when used in agent devices spread over a geographical data, with the grid representing the necessary granularity of the sensor readings. In the first figure, we have an agent aggregating other agent sensor readings as defined by its learned local neighbourhood. We then lose two of the sensors that were on devices that formed the most-optimal data reading request channels in other agents neighbourhoods. The second figure shows how the agents then quickly learn to prioritise data requests over other channels to form an adapted neighbourhood, although this step could also have introduced new channels as part of a state-action space adaptation by the agent to recover connectivity. This shows how we can deploy an agent-system that can adaptively recover from significant outages across the sensor network.


\subsection{Power consumption and recovery}

Another important aspect in the envisaged E-WSN system is the price of inter-agent communication as dictated by the energy consumption of each sensors transceiver while broadcasting. To make the deployment feasible we need a low-energy use system, while still having agents dynamically self-coordintate and using their communication channels to retrieve data readings. With the concept of channels already included within our agent system we can add this easily to the simulation by providing an energy cost to links through these channels. This equates to the power necessary to broadcast throught this communication channel, simulating distance between agents, occlusions, and atmospheric effects that may force higher power requirements on the transeiver if utilised in a real environment. In this way, each message passed through a link will not only shape the subsequent reward through return time, but also through the energy cost of the message. This will ensure that we are not only having agents learn the most efficent data transfer behaviours, but also have the energy efficiency of their requests impact the learning. We also must take in to account the need to spread power distribution across the system. There cannot be a small subset of nodes that are handling the majority of the functionality and therefore the energy usage as their batteries and solar top-up systems will become overwhelmed. Instead we must learn to spread the load, indeed this is exactly what the congestion simulation proviously did through delaying data return times.
\newline
\newline
Now having extended this concept and added an energy consumption measure to agents channels which are proportional to the energy use generated by messages passing through their links, we can use this to accumulate energy consumption rates for those linked-to agents, and use this to drive the requesting agent learning towards adaptation in the face of overloading links. Using $E\textsubscript{cost} $ as the energy cost of making a broadcast request through a link, and $E\textsubscript{rec}$ as the energy recovered by the solar panels of the device per unit of time. We can then use the set of actions on a link $l$ at time $t$ as  $\mathbb{A}_l^t$ to define the actions that cause the link to broadcast, and therefore use energy. We can therefore specify the current energy usage $E\textsubscript{usage} $ of the link as,

\begin{equation}
	E\textsubscript{usage} =  \sum\limits_{i=0}^{t}
	(E\textsubscript{cost} - E_{rec}^{(t-i)})
	\times \mathbb{A}_l^i
\end{equation}
\newline
\newline

We then use the energy usage of a link as a negative factor in our rewards for the requesting agent. Energy costs such as altering links and channels are already encapsulated within the agent system as fixed negative rewards.
	\section{Balancing energy and measurement in a WSN system}
	\section{Problem}
\label{section:problem}
%%%%%%%%%%%%%%%%%% NOTATION %%%%%%%%%%%%%%%%%%%%%%%%%%
\newcommand{\varTime}[2]{\phi}
\newcommand{\setTime}[2]{\Phi}
\newcommand{\varAtomicTask}[2]{\varSymbol{tp}{#1}{#2}}
\newcommand{\varCompositeTask}[2]{\varSymbolHat{tp}{#1}{#2}}
\newcommand{\varAgent}[2]{g_{#1}^{#2}}
\newcommand{\setAgents}[2]{G_{#1}^{#2}}
\newcommand{\varIdleAgent}[2]{\varAgent{i}{}}  
\newcommand{\varSleepAgent}[2]{\varAgent{s}{}}  
\newcommand{\varChildAgent}[2]{\varAgent{c}{}}    
\newcommand{\varParentAgent}[2]{\varAgent{p}{}}

\newcommand{\varOrchestrationEnergy}[2]{OE}
\newcommand{\varTransmissionEnergy}[2]{TE}
\newcommand{\varReceiverEnergy}[2]{RE}
\newcommand{\varIdleEnergy}[2]{IE}
\newcommand{\varSleepEnergy}[2]{SE}
\newcommand{\varSystemEnergyConsumption}[2]{sec_{\setTime{}{}}}
\newcommand{\varTaskEnergy}[2]{tec_{\varCompositeTask{}{}}}
\newcommand{\setTaskEnergy}[2]{TEC_{\varCompositeTask{}{}}}
\newcommand{\varSystemEnergyVariability}[2]{sev_{\varTime{}{}}}

\newcommand{\varAgentEnergyAvailable}[2]{aea_{\varAgent{}{},\varTime{}{}}}
\newcommand{\functionAgentEnergyTotal}[2]{
	\functionSignature{aet_{\varAgent{}{},\varTime{}{}}}
	{\varAgent{#1}{#2}}
}

\newcommand{\setAgentEnergyAvailable}[2]{\mathrm{AEA}_{\varAgent{}{},\varTime{}{}}}
\newcommand{\functionSystemEnergyAvailable}[2]{
	\functionSignature{sea_{\varTime{}{}}}{\setAgents{}{}}
}

\newcommand{\functionProbabilityInit}[2]{
	\functionSignature{pinit}{\varAgent{}{}}
}
\newcommand{\functionProbabilityRand}[2]{
	\functionSignature{prand}{\varAgent{}{}}
}
\newcommand{\functionProbabilityWear}[2]{
	\functionSignature{pwear}{\varAgent{}{}}
}
\newcommand{\functionProbabilityFail}[2]{
	\functionSignature{pfail}{\varAgent{}{}}
}
\newcommand{\varConstantInit}[2]{{C}{INIT}^{#2}}
\newcommand{\varConstantRand}[2]{{C}{RAND}^{#2}}
\newcommand{\varConstantWear}[2]{{C}{WEAR}^{#2}}
\newcommand{\varEnergyInit}[2]{{E}{INIT}^{#2}}
\newcommand{\varEnergyWear}[2]{{E}{WEAR}^{#2}}
\newcommand{\functionCompositeTaskCoverage}[2]{
	\functionSignature{ctc}{\varCompositeTask{}{}, \varCompositeTask{}{*}}
}
%%%%%%%%%%%%%%%%%% NOTATION %%%%%%%%%%%%%%%%%%%%%%%%%%
\newcommand{\varSample}[2]{\varSymbol{\psi}{#1}{#2}}
\newcommand{\setSample}[2]{\setSymbol{\Psi}{#1}{#2}}
\newcommand{\varMeasurement}[2]{\varSymbol{m}{#1}{#2}}
\newcommand{\varPeriod}[2]{\varSymbol{p}{#1}{#2}}
\newcommand{\varError}[2]{\varSymbol{\omega}{#1}{#2}}
\newcommand{\setError}[2]{\setSymbol{\Omega}{#1}{#2}}
\newcommand{\tupleVarSample}[2]{
	(\varTime{#1}{#2}, \varMeasurement{#1}{#2}, \varError{#1}{#2})
}
\newcommand{\tupleSetSample}[2]{
	(\setTime{#1}{#2}, \setMeasurement{#1}{#2}, \setError{#1}{#2})
}

%%%%%%%%%%%%%%%%%% NOTATION %%%%%%%%%%%%%%%%%%%%%%%%%%
\newcommand{\varEnergy}[2]{\varSymbol{e}{\varAgent{}{}}{#2}}
\newcommand{\setEnergy}[2]{\setSymbol{E}{#1}{#2}}
\newcommand{\varEnergyMax}[2]{\varEnergy{\varAgent{}{}}{max}}
%%%%%%%%%%%%%%%%%% NOTATION %%%%%%%%%%%%%%%%%%%%%%%%%%
\newcommand{\setEnergyDeltaMeasurement}[2]{\Delta\setSymbol{E}{\texttt{measure}}{#2}}
\newcommand{\setEnergyDataAcquisition}[2]{\setSymbol{E}{\texttt{DAQ}}{#2}}
%%%%%%%%%%%%%%%%%% NOTATION %%%%%%%%%%%%%%%%%%%%%%%%%%
\newcommand{\setEnergyDeltaAggregate}[2]{\Delta\setSymbol{E}{\texttt{aggregate}}{#2}}
\newcommand{\setEnergyBroadcast}[2]{\setSymbol{E}{\texttt{BX}}{#2}}
\newcommand{\setEnergyCompute}[2]{\setSymbol{E}{\texttt{compute}}{#2}}
\newcommand{\setEnergyTransmission}[2]{\setSymbol{E}{\texttt{TX}}{#2}}
\newcommand{\setEnergyReceived}[2]{\setSymbol{E}{\texttt{RX}}{#2}}
%%%%%%%%%%%%%%%%%%%%%%%%%%%%%%%%%%%%%%%%%%%%%%%%%%%%%%
\subsection{WSN system}

\subsection{Energy in the system}

\subsubsection{Availability of energy}

Each agent $\varAgent{}{}$ has a battery that can store energy $\varEnergy{}{}$ up to a maximum value $\varEnergyMax{}{}$.

\begin{definition}[Agent energy available]
	The agent energy available, $\varAgentEnergyAvailable{\varAgent{}{}}{}$, is the fractional energy available to an agent $\varAgent{}{}$ to utilise.
	\begin{equation}
		\varAgentEnergyAvailable{}{} = \varEnergy{}{} / \varEnergyMax{}{}
	\end{equation}
\end{definition}

\begin{definition}[System energy available]
	The system energy available, $\functionSystemEnergyAvailable{}{}$, is the energy available to all agents $\setAgents{}{}$ in the system as a whole.
	\begin{equation}
		\functionSystemEnergyAvailable{}{} 
		= \sum_{\forall \varAgent{}{} \in \setAgents{}{}} \varAgentEnergyAvailable{\varAgent{i}{}}{}
	\end{equation}
	Note, the task energy consumption also includes any orchestration tasks agents execute such as information requests.
\end{definition}

\begin{definition}[System energy variability]
	The system energy variability, $\varSystemEnergyVariability{}{}$, is the variance\footnote{Using the standard definition of variability of a discrete set $X$, $\sigma^2(X) = \frac{\sum (x_i - \bar{x})^2}{\funcSize{X}-1}$} of energy available to each agent in the system as a whole at time $\varTime{}{}$.
	\begin{equation}
		\varSystemEnergyVariability{}{} 
		= \sigma^2(\setAgentEnergyAvailable{}{})
	\end{equation}
\end{definition}

\subsubsection{Energy consumption}

\begin{definition}[Task energy consumption]
	The task energy consumption, $\varTaskEnergy{}{}$, is the energy used by all agents in the system in executing a composite task $\varCompositeTask{}{}$. Transmission energy, $\varTransmissionEnergy{}{}$, the power used by a parent agent $\varParentAgent{}{}$ broadcasting a message to a child agent $\varChildAgent{}{}$, or a child agent replying to a parent. Receiver energy, $\varReceiverEnergy{}{}$, is the energy used an agent receives a message. Finally idle energy, $\varIdleEnergy{}{}$, for any inactive agents $\varIdleAgent{}{}$ in idle, power saving mode, and sleep mode energy $\varIdleEnergy{}{}$ for agents $\varSleepAgent{}{}$ in sleep mode.
   	\begin{equation}
   		\varTaskEnergy{}{} 
   		= \varTransmissionEnergy{}{} \sum (\varParentAgent{}{} + \varChildAgent{}{})
   		+ \varReceiverEnergy{}{} (1 + \sum \varChildAgent{}{})
   		+ \varIdleEnergy{}{} \sum \varIdleAgent{}{}
   		+ \varSleepEnergy{}{} \sum \varSleepAgent{}{}
   	\end{equation}
\end{definition}



\begin{definition}[System energy consumption]
	The system energy consumption, $\varSystemEnergyConsumption{}{}$, is the energy used by all of the $m$ tasks that are completed within the time period $\setTime{}{}$.
	\begin{equation}
		\varSystemEnergyConsumption{}{} 
		= \sum_{1}^{m} \varTaskEnergy{}{}
	\end{equation}
	Note, the task energy consumption also includes any orchestration tasks agents execute such as information requests.
\end{definition}

\subsubsection{Energy recovery}

%%%%%%%%%%%%%%%%%% NOTATION %%%%%%%%%%%%%%%%%%%%%%%%%%
\newcommand{\setEnergyDeltaIdle}[2]{\Delta\setSymbol{E}{\texttt{idle}}{#2}}
\newcommand{\setEnergyHarvest}[2]{\setSymbol{E}{\texttt{harvest}}{#2}}
\newcommand{\setEnergyWUR}[2]{\setSymbol{E}{\texttt{idle-wur}}{#2}}
%%%%%%%%%%%%%%%%%%%%%%%%%%%%%%%%%%%%%%%%%%%%%%%%%%%%%%

When not sampling, aggregating, or broadcasting a node recovers energy at a rate given by the energy harvesting solar panel minus that lost by having the WUR in standby.

\begin{equation}
	\setEnergyDeltaIdle{}{}
	= 
	\setEnergyHarvest{}{}
	-
	\setEnergyWUR{}{}
\end{equation}

\subsubsection{Sensor lifetime in WSN}
We can model the failure of nodes through a simple bathtub model as described by the three phases of node lifespan as seen in Figure \ref{fig:node_reliability_lifespan},
\begin{figure}
	\centering
	\includegraphics[width=0.7\linewidth]{node_reliability_lifespan}
	\caption[Probability of failure with lifetime energy usage for a node]{Probability of failure with lifetime energy usage for a node}
	\label{fig:node_reliability_lifespan}
\end{figure}
\begin{enumerate}
	\item $\functionProbabilityInit{}{}$, the initial failure probability, component failures early in lifespan, decreasing with time.
	\item $\functionProbabilityRand{}{}$, the random failure probability, the constant background failure rate.
	\item $\functionProbabilityWear{}{}$, the wear-out failure probability, the increasing failure rate towards the end of a nodes expected lifespan.
\end{enumerate}
For each node, we use $\functionAgentEnergyTotal{}{}$ as a proxy for time so the probability of an agent $\varAgent{}{}$ having a permanent failure increases with usage.

\definition[Probability of node failure]{
	The probability of node failure for an agent $\varAgent{}{}$ is $\functionProbabilityFail{}{}$, the combination of the probabilities of initial, random, and wear-out failures given the agents total energy usage in its lifetime $\functionAgentEnergyTotal{}{}$.
\begin{equation}
	\functionProbabilityFail{}{} = \underbrace{max(0, (1 - \functionAgentEnergyTotal{}{}/\varEnergyInit{}{})) \times \varConstantInit{}{}}_{\functionProbabilityInit{}{}} + \underbrace{\varConstantRand{}{}}_{\functionProbabilityRand{}{}} + \underbrace{max(0, ( \functionAgentEnergyTotal{}{}/\varEnergyWear{}{}))\times \varConstantWear{}{}}_{\functionProbabilityWear{}{}}
\end{equation}
	Where $\varEnergyInit{}{}{}{}$ is the energy level where initial failures are effectively zero,  and $\varEnergyWear{}{}{}{}$ where wear-out failures become a factor. $\varConstantInit{}{}$, $\varConstantRand{}{}$, and $\varConstantWear{}{}$  are constants chosen to scale the effects of initial failures and wear-out failures respectively.
}

\subsection{Coverage and resilience of routing}

\subsubsection{Coverage}

\begin{definition}[Composite task coverage]
	Given a composite task $\varCompositeTask{}{}$, and completes or partially completes this task $\varCompositeTask{}{*} \subseteq \varCompositeTask{}{}$ then the \textit{composite task coverage} of $\varCompositeTask{}{}$ is the fraction of successfully completed atomic tasks of the composite task.
	\begin{equation}
		\functionCompositeTaskCoverage{}{} = \frac{
			\funcSize{\varCompositeTask{}{*}
			}
		}{
			\funcSize{\varCompositeTask{}{}
		}
	}
	\end{equation}
\end{definition}


\subsubsection{Route adaptation}
Figure \ref{fig:wsnhierarchicalresolution} shows one of the possible modes of hierarchical networking that can be formed by the agent system through per-agent local neighbourhood learning. The initial agent passes requests through smaller and smaller neighbourhoods until the granulaity of the data meets that of the request

\begin{figure}[ht]
	\centering
	\includegraphics[width=0.5\linewidth]{WSN_hierarchical_resolution}
	\caption{Agent system hierarchical neighbourhood resolution}
	\label{fig:wsnhierarchicalresolution}
\end{figure}
In Figure \ref{fig:wsnsimulationmapfailureadaptation}, the first figure shows a set of agents aggregating data across a defined granularity geographical area using their learned neighbourhoods. In the second we see the adaptation of each agents local neighbourhood as they individually react to device failures
\begin{figure}[ht]
	\centering
	\includegraphics[width=0.5\linewidth]{WSN_simulation_map_failure_adaptation}
	\caption{WSN simulation failure adaptation}
	\label{fig:wsnsimulationmapfailureadaptation}
\end{figure}

\subsection{Defining the problem}
\todo[inline]{TODO}\subsection{Components and energy usage}

Example values in \ref{table:components_energy_usage} show the energy and time costs of varios operations given an hourly sampling period for a node.


\begin{table}
\begin{tabular}{p{0.5\textwidth}p{0.2\textwidth} p{0.2\textwidth} }
\hline
Function & power(mW) & time (s) \\
\hline
Data aquisition (DAQ) & 0.5 & 60 \\
Transmission (TX) & 7 & 0.1 \\
Broadcast (BX) & 70 & 0.1 \\
Idle wake-up-radio (WUR) & 0.07 & 3509 \\
\end{tabular}
\caption{Component types and energy usage}
\label{table:components_energy_usage}
\end{table}

\section{Problem definition}

\todo[inline]{State the optimisation problem explicitly given the system as described}

	
In this section we will formally set out the problem we look to solve. We define the base concepts of our WSN system in Sections \ref{section:terminology} and \ref{section:tasks}, looking at,
\reviewquestion{Opening list: For neatness of the sentence, please un-capitalise "Nodes", "Agents", "Tasks" and "Resources" and consistently end the first three items with commas or semi-colons (personally I'd use semi-colons as each item is a long clause).
}
\reviewquestion{Final sentence of intro: "in in"}
\begin{itemize}
	\item nodes, the hardware devices connected to the network;
	\item agents, the software controllers for each node;
	\item tasks, the actions that agents should execute, in this case, returning a measurement in a geographic location;
	\item resources, required by agents to complete tasks; 
\end{itemize}
In completing tasks, agents will be able to choose from a number of different actions, and assume different roles in executing each individual task, which we look at in Sections \ref{section:roles} and \ref{section:actions}. In Section \ref{section:energy_consumption} we describe how energy is consumed by the system. Next, in Section \ref{section:task_quality}, we define to what quality a measurement is completed, and how this, energy availability, and energy distribution in the system, effects the quality of overall composite task completion. Finally, we detail task coverage in Section \ref{section:coverage}, and how this relates to the resilience of the network, and the effective lifetime of the system, giving us the problem definition in Section \ref{section:optimisation_problem}.

\subsection{Wireless sensor network components and terminology}
\label{section:terminology}

A WSN system is comprised of a set of \textit{nodes}. Each node is equipped with a microcontroller for computation, a battery for power storage, a solar panel for recharging, a wireless transceiver for  transmitting and receiving messages from other nodes, and one or more \textit{sensors} for sensing and measuring some property of the environment such as temperature or radiation levels \citep{muhammad_r_ahmed_2012_1072589}. Each node has one \textit{agent}, a software controller that instructs its actions. Nodes may be deployed to precise locations or through a more random distribution method. The network structure formed may be flat or hierarchical, with or without clustering, dependent on the choice of protocol used \citep{Carlos-Mancilla2016b}. 

\reviewquestionopen{Reading through Section 3, it feels like there is something missing before Section 3.1 on the characteristics of the environment into which the WSN is deployed. There are hints of things given along the way, but no coherent list of assumptions. In particular, I'm thinking of things like the 'location grid' in Section 3.6 and the opening sentences of Section 3.7 on the relation between spatial distance and radiation levels (i.e. a gradual change over space).
}
	\subsection{Overview}
\label{problem_overview}
A WSN system is comprised of a set of interconnected hardware \textit{nodes}. Each node is equipped with a microcontroller for computation, memory storage, a battery for power storage, a solar panel for recharging the battery, a wireless transceiver for  transmitting and receiving messages from other nodes, and one or more \textit{sensors} for sensing and measuring some property of the environment such as temperature or radiation levels \citep{muhammad_r_ahmed_2012_1072589}. Each node has one \textit{agent}, a software controller that instructs its actions. For simplicity, we will use the term 'agent' to refer to both the software controller and the hardware node it controls. 

\todo[inline]{location grid concept needs to make more sense}
Agents may be deployed to precise locations or through a more random distribution method. Their placement is measured in comparison to a \textit{location grid}, a 2d co-ordinate system overlayed over the systems' environment (although this is also applicable to 3 dimensionsal environments).  The network structure formed may be flat or hierarchical, with or without clustering, dependent on the choice of protocol used \citep{Carlos-Mancilla2016b}. 

Agents will have to use \textit{resources} such as computation power, memory storage, and energy from their batteries to execute tasks, transmit messages to other agents, and discover new agents in the network. At a given point in time, there is a fixed amount of computational or memory resources, which must be shared. Battery power has a maximum capacity which is depleted when the agent takes actions, but is slowly replenished over time through charging by a solar panel.  

Agents in the system carry out \textit{tasks{}, which specify taking a measurement at a specific location. The further away the sensor is from this point, the less value the task has to the system.

At a high-level, the flow of task execution is as follows,
\begin{enumerate}
	\item An agent receives a tasks from an external agent
	\item The agent executes or allocates these tasks to other agents in the system, which may also re-allocate them.
	\item Agents' execute tasks by taking measurements using its sensors.
	\item The task results are transmitted back to the agent that allocated the task.
	\item The results are re-transmitted back to allocating agents until they reach the initial agent.
	\item The initial agent aggregates all the task results and transmits them to the external agent.
\end{enumerate}
In completing tasks, agents will be able to choose from a number of different actions, and assume different roles in executing each individual task. We look at roles and actions in Sections \ref{section:roles} and \ref{section:actions}. In Section \ref{section:energy_consumption} we describe how energy is consumed by the system. Next, in Section \ref{section:task_quality}, we define to what quality a measurement is completed, and how this, energy availability, and energy distribution in the system, effects the quality of overall composite task completion. Finally, we detail task coverage in Section \ref{section:coverage}, and how this relates to the resilience of the network, and the effective lifetime of the system, giving us the problem definition in Section \ref{section:optimisation_problem}.

\begin{example}
	\todo[inline]{XXX}
\end{example}


	\paragraph{System}
\label{section:system_definition}
\todo[inline]{Define the system as things that are constant, and persistent, not dependent on time}

\reviewquestionopen{	
	There is an issue that the system definition is both complex, having 10 terms in the tuple, and apparently incomplete because lots of things are mentioned in the following sections that appear to be part of the system but are not in the definition: links, information, transmission energy, etc. Why, for example, is conf part of the system definition but $e_trans$ is not? I wonder if you could present things differently:
	- A system is defined by the fixed sets and constants that are independent of any given agent or task: $(G, AT, CT, R, e_trans, ...)$
	- Everything else is presented separately from the system definition as a function returning some property of an agent, task or resource, i.e. you separately define each of ar, ra, sg, conf, sink, sensor, etc. That is, you are presenting as if the values returned by these functions are already present in the system by being attributes of the function's inputs (which are agents/tasks/resources in a given system) but the functions themselves are independent of any given system.
}


We define the agent-based system as the tuple, $\langle 
	\setAtomicTask{}{},
	\setCompositeTask{}{},
	\setResource{}{},
	\setAgents{}{}
\rangle$, where
\begin{itemize}
	\item $\setAtomicTask{}{}$ is a set of atomic tasks where each task is a measurement task performed by a single agent;
	\item $\setCompositeTask{}{} \subseteq \powerSetSymbol{\setAtomicTask{}{}}{}{}$ is the set of composite tasks that occur in the system;
	\item $\setResource{}{}$ is a set of resources needed to perform atomic tasks;
	\item $\setAgents{}{}$ is a set of agents in the system, each agent $\varAgent{}{}$ being defined by a tuple $\tupleAgent{}{}$ where;
	\begin{itemize}
		\item $\varAgentCapability{}{}\subseteq \setAtomicTaskType{}{}$ is the agent capabilities; i.e., the atomic task types that the agent can perform;
		 \item $\varAgentResponsiblity{}{} \subseteq \setCompositeTaskType{}{}$ is the agent responsibilities; i.e., the composite task types that the agent can oversee;
		\item $\varAgentNeighbourhoodConstraint{}{}, \varAgentKnowledgeConstraint{}{} \in \mathbb{N}$, are the resource constraints of the agent, namely the communication and memory constraints (i.e., how many other agents a given agent can communicate with and know about).
	\end{itemize}
\end{itemize}

	\subsection{System state}

Composite tasks arrive in the WSN from external agents $\setExternalAgent{}{}$, with a constant or slowly varying frequency. At a point in time, a sink will process one such task, and return it to the external agent when completed.
\begin{enumerate}
	\item An external agent sends a composite task $\varCompositeTask{}{}$ of type $\varCompositeTaskType{}{}$ to an agent $\varAgent{}{} \in \functionTaskResponsibilities{}{}$. 
	\item $\varAgent{}{}$ decomposes the composite task into atomic tasks.
	\item $\varAgent{}{}$ allocates the atomic tasks to agents it knows of, which may include itself.
	\item $\varAgent{}{}$ aggregates the results of the atomic tasks as their results are returned.
	\item $\varAgent{}{}$ returns the aggregated results to the external agent $\varExternalAgent{}{}$.
\end{enumerate}

For an agent $\varAgent{}{}$ to be able to allocate tasks to another agent $\varAgent{}{'}$, it must have \textit{knowledge} of $\varAgent{}{'}$, i.e. be aware that agent and its capabilities to execute tasks. It also must be \textit{connected} to the agent, i.e. have communication links established with $\varAgent{}{'}$. We define the set of agents an agent is connected to as its \textit{neighbourhood}. An agent can only allocate tasks to agents that it is connected to. Agents can also make requests to agents in its neighbourhood for knowledge, enabling it to discover new agents in the system.
 

\begin{definition}[Agent State]
	\label{def:agent-state}
	Given an agent $g=\langle c, r, \delta_n, \delta_k \rangle$, we define its state at a particular point in time as a tuple $\langle K, N\rangle$, where:
	\begin{itemize}
		\item $K\subseteq G$ is the knowledge of the agent\footnote{For simplicity, we represent the knowledge about a particular agent by the agent identifier, but the knowledge also includes other information such as the agent capabilities and qualities when performing particular actions, etc. }.
		\item $N\subset K$ is the neighbourhood of the agent.
	\end{itemize}
\end{definition}

	\section{Resource optimisation}
	We categorise resource usage by the following criteria\footnote{For many systems the simplifications discussed are reasonable. In some cases however, sensors may have significant energy requirements as an example. This increases the complexity of the modelling but can be incorporated into the same framework we present.};
\begin{enumerate}
	\item \textit{operational}, resource usage resulting from an agents' general functionality that are not part of a task execution. e.g. energy usage such as idle and sleep cycle functions, the computational and memory storage costs of background operations. 
	
	\item \textit{significant}, often there is a dominant use of a resource that makes other uses insignificant in comparison. e.g. when completing tasks, the energy used to transmit and receive results are large compared to the resources used to activate and coordinate other internal components. 
	
	\item \textit{optimisable}, whether or not the resource usage can be actively optimised by the agent. Agents' can shorten their transmission range, meaning less energy is used for broadcasting. By reducing the sampling time of a sensor, they can reduce the computation and energy usage. This makes these actions optimisable, and strategies can be learned.
\end{enumerate} 

In this work we focus on the cost of task completions, and disregard other operational factors. Other work covers this area such as on battery duty-cycling, transmission algorithms, and solar-energy harvesting optimisation \citep{Kumar2010,Pinto2012,Matin2012,Escolar2014, Sharma2018}. With the justifications above on significance of the resource usage and optimisability, we model the relationships between an agents' actions and resultant resource usage as follows. 

\paragraph{Actions}
Agents in the system can perform \textit{actions}. Each action $\varAction{}{}$ is mapped by the function $\functionAgentActionType{}{}$ to one of the action-categories below;

\begin{enumerate}
	\item $\functionExec{}{}$ - the agent will perform a \textit{task execution} of an atomic task $\varAtomicTask{}{}$ itself, using its computational resources. Computational resource usage is adaptable by an agent executing a task. The more of this resource it can allocate to a task, the higher quality that task will be performed to.

	\item $\functionAlloc{}{}$ - the agent will attempt \textit{task allocation} of an atomic task $\varAtomicTask{}{}$ to another agent $\varAgent{}{}$. This will use energy resources to transmit tasks and receive results, the amount being dependant on the distance between the two agents. Therefore, agents that are transmitting messages can reduce energy usage by preferring nearby agents. This may; make the distance of the resulting sensor measurement from the tasks' demand point larger, or require relaying, i.e. receiving and re-transmitting, the task through a greater number of agents.

	\item  $\functionInfo{}{}$ - the agent will make an \textit{information request} to acquire knowledge from another agent $\varAgent{}{}$. This will require energy use to transmit and receive messages, and use memory storage for the results. 
	
	\item $\functionLink{}{}$ -  an agent carrying out a \textit{connection action} will allocate memory storage to hold knowledge on an agent $\varAgent{}{}$.  Agents will use their memory resources to maintain knowledge and neighbourhood information, but within their fixed maximum capacity.
\end{enumerate}

\paragraph{Dominant energy costs}
\label{section:problem:dominant_energy_costs}
%%%%%%%%%%%%%%%%%%%%
\newcommand{\formalTransmissionEnergy}[2]{
	\functionFormal{energy_{\symbolTransmission{}{}}}
	{\setAgent{}{} \times \setAgent{}{}}
	{\setRealNumbersPositive{}{}}
}
\newcommand{\functionTransmissionEnergy}[2]{
	\functionSignature{energy_{\symbolTransmission{}{}}}
	{\varAgent{#1}{},\varAgent{#2}{}}
}
\newcommand{\functionTransmissionEnergyIndexed}[2]{
	\functionTransmissionEnergy{\varAgent{i}{}}{\varAgent{i+1}{}}
}
\newcommand{\functionTransmissionEnergySink}[2]{
	\functionSignature{energy_{\symbolTransmission{}{}}}
	{\functionSinkRoleAtomic{}{},\varAgent{2}{}}
}
\newcommand{\functionTransmissionEnergyRelay}[2]{
	\functionSignature{energy_{\symbolTransmission{}{}}}
	{\varAgent{#1}{},\varAgent{#2}{}}
}
\newcommand{\functionTransmissionEnergyDetector}[2]{
	\functionSignature{energy_{\symbolTransmission{}{}}}
	{\varAgent{#1}{},\varAgent{#2}{}}
}
In calculating the energy involved in completing tasks we make the following assumptions;
\begin{itemize}
	\item \textit{transmission energy}, the energy used by agents to send a message. We assume that there is a linear relationship between the distance separating agents and the energy cost\footnote{A more complex relationship could be modelled with no expected impact on the effectiveness of our algorithms}, so that in transmitting a message from $\varAgent{1}{}$ to $\varAgent{2}{}$, there is a constant $\varTransmissionEnergy{}{}$ such that;
	\begin{equation}
		\functionTransmissionEnergy{1}{2}
		=  \varTransmissionEnergy{}{}\funcSize{\functionDeployment{2}{} - \functionDeployment{1}{}}
	\end{equation}
	\item \textit{receiver energy}, the energy used by an agent to receive a message. We assume the energy use for receiving a message in not dependent on the distance, and ignore the impact of message length, to give a constant value $\varReceiverEnergy{}{}$, 
	\item \textit{sensor energy}, the energy used by an agent to execute an atomic measurement task. We make the common assumption that sensor activation is the dominant energy cost when taking measurements, with repeated samples being less significant \citep{Razzaque2014}. This means that the overall energy use of a sensor-type in the system while executing a task approximately constant, $\varSensorEnergy{}{}$. However, this assumption can be less valid depending on the type of sensor, and how its instruments take samples.
\end{itemize}

	\subsection{Agent actions}
\label{section:actions}
\reviewquestion{Given what you said about failures in Section 2, for alloc(at,g) do you mean that "the agent will *attempt* to allocate the atomic task..."?
}
\reviewquestion{It is unclear what 'information' an agent could request with info(g). No exchange of information is described in earlier sections, and agents do not obviously hold information according to the system definition in Section 3.2.
}
\reviewquestion{Links seem mentioned out of the blue here without adequate explanation. If the resources allocated to atomic task types are part of the system definition, why are an agent's links not part of that definition? Where do you formalise a link and say what it is?
}
Agents in the system can perform \textit{actions}. Each action $\varAction{}{}$ is mapped by the function $\functionAgentActionType{}{}$ to one of the action-categories below,

\begin{enumerate}
	\item $\functionExec{}{}$, The agent will execute the atomic task $\varAtomicTask{}{}$ itself, this may require some of its available resources.
	\item $\functionAlloc{}{}$, the agent will attempt to allocate the atomic task $\varAtomicTask{}{}$ to another agent $\varAgent{}{}$.
	\item $\functionInfo{}{}$, the agent will request knowledge from another agent $\varAgent{}{}$.
	\item $\functionLink{}{}$, the agent will allocate resources to hold knowledge on the agent $\varAgent{}{}$ and maintain a connection. 
\end{enumerate}
		\subsection{Node roles and task paths}
\label{section:roles}

%%%%%%%%%%%%%%%%%%%%%%%%%%%%%%%%%%%%%%%%%%%%
\newcommand{\formalSinkRole}[2]{
	\functionFormal{sink}
	{\setCompositeTask{}{}}
	{\powerSetAgents}{}{}
}
\newcommand{\functionSinkRole}[2]{
	\functionSignature{sink}
	{\setAtomicTaskInstance{}{}, \varAgent{}{}}
}

\newcommand{\formalAtomicTaskRole}[2]{
	\functionFormal{#1}{\setAtomicTaskInstance{}{} \times \setAgent{}{}}{ \setAgent{}{}}
}

\newcommand{\formalDetectorRole}[2]{
	\functionFormal{detect}{\setAtomicTaskInstance{}{} \times \setAgent{}{}}{ \setAgent{}{}}
}

\newcommand{\formalRelayRole}[2]{
	\functionFormal{relay}{\setAtomicTaskInstance{}{} \times \setAgent{}{}}{ \setAgent{}{}}
}

\newcommand{\functionDetectorRole}[2]{
	\functionSignature{detect}{\varAtomicTask{}{}, \varAgent{}{}}
}

\newcommand{\functionRelayRole}[2]{
	\functionSignature{relay}{\varAtomicTask{}{}, \varAgent{}{}}
}

\reviewquestion{What happens to the measurement, e.g. is it stored by the sensor node, or communicated by the sensor node directly out of the system, or back along the task-path to the sink node, or along a potentially different path to the sink node, or something else? While I know you go on to explain more detail in later sections, it seems odd to omit the end of the description of what happens to a request at this point.
}
\reviewquestion{Figure 2 seems inconsistent with the text early in the section.
	- How can there be nodes that are distinctly 'sensor nodes' when you say every node has 'one or more sensors'? When you say "...a final measurement is made by a sensor node" do you mean "a sensor node *for that task*", i.e. it is a role played for a given task rather than a distinct kind of node?  If it is meant to be a role within a task, 'sensor node' sounds like a type rather than a role, so there may be a better name to use.
	- When every node has an agent by your definition, then why are some elements in the figure called nodes and others (apparently of the same kind) called agents? 
}
\reviewquestionopen{Why does one agent (and possibly by implication other agents) have a 'broadcast radius of node' shown, but this broadcast radius is not mentioned as a property of an agent or node in the text?}


\reviewquestion{This is clearer than in Section 3.1, but the use of terms seems inconsistent again. In Section 3.1, nodes were hardware while agents were software. In Section 3.2, the terms were interchangeable. In Section 3.3, nodes are roles that agents can play with regard to an atomic task. This is too confusing - there needs to be some consistency.
}
\reviewquestion{The distinction between idle node and sleep node roles for an atomic task seems odd, as neither appears to have any part in the completion of the task. This is just a note not necessarily a point of concern - maybe the reason for distinguishing them becomes clearer later in the paper.
}
\reviewquestion{As said for Section 3.1, I still don't understand what happens to the measurement once it is taken.
}

For each atomic task, agents will have different \textit{roles} for that task,  what behaviours they take on to help complete the task. A \textit{sink} is an agent that  receives a set of tasks from outside the system, $\formalSinkRole{}{}$. A \textit{relay} is an agent that is allocated an atomic task, but does not complete it, instead re-allocating it to other agents, $\formalRelayRole{}{}$. A \textit{detector} is an agent that  executes an atomic task, and so performs the measurement, $\formalDetectorRole{}{}$.

The steps of a task completion are,

\begin{enumerate}
	\item A set of atomic tasks $\setAtomicTaskInstance{}{}$ is sent from an agent external to the system.
	\item  $\functionSinkRole{}{}$ can either complete each atomic task $
	\varAtomicTask{}{} \in \setAtomicTaskInstance{}{}$ itself, or allocate it to another agent.
	 \item $\functionRelayRole{}{}$ receives an atomic task, and passes it on to another agent.
	 \item $\functionDetectorRole{}{}$ performs the atomic task.
	 \item The detector returns the results of the atomic task execution back to the agent that allocated it $\varAtomicTask{}{}$.
	 \item Relay agents pass back the atomic task results to the agents that allocated the corresponding atomic task to them, until the results reach the initiating sink agent.
	 \item The sink agent aggregates the results of all the atomic tasks $ \varAtomicTask{}{} \in \setAtomicTaskInstance{}{}$ and sends this aggregated result back to the external agent that sent it the set of tasks $\setAtomicTaskInstance{}{}$.
\end{enumerate}

With these roles in mind, we can now define the  \textit{task-path} as a mapping of atomic tasks to ordered sequence of agents $\formalTaskArc{}{}$ that each atomic task $\varAtomicTask{}{}$ is sub-allocated to. The first agent is that of the sink node that has received the initial composite task, and the last agent is that of the sensor node that executes the atomic task, with the sequence of agents in-between those of the active nodes relaying the atomic task. So for a task path of depth $n$, we have
$\functionTaskArc{}{} = \lbrace \functionSinkRole{}{}, \functionRelayRole{i}{}, \functionDetectorRole{}{} \rbrace_{i=1}^{n-2}$. 


\begin{figure}
	\centering 
	\includegraphics[width=0.9\linewidth, trim={25pt 0pt 24pt 0pt, clip}]{grid_concept}
	\caption[WSN deployment terminology]{WSN components and terminology}
	\label{fig:grid_concept}
\end{figure}


 These concepts are illustrated in Figure \ref{fig:grid_concept} and will be detailed in the following sections.


	
\paragraph{Task paths}
We can now define the \textit{task-path} as a mapping of atomic tasks to ordered sequence of agents that work together to complete those tasks, $\formalTaskArc{}{}$. In completing an atomic task, the first agent in the sequence is the sink that received the associated composite task. The last agent is the detector that executes the atomic task. The sequence of agents in-between are those agents relaying the atomic task, their position in this sequence denoted by the subscript $i$ where required. So for a task path of depth $n$, we have;
\begin{equation}
	\functionTaskArc{}{} = \lbrace \functionSinkRoleAtomic{}{}, \functionRelayRole{i}{i+1}{}, \functionDetectorRole{}{} \rbrace_{i=2}^{n-1}
\end{equation}
\begin{figure}
	\centering 
	\includegraphics[width=0.9\linewidth, trim={25pt 0pt 24pt 0pt, clip}]{grid_concept}
	\caption[WSN deployment terminology]{WSN components and terminology}
	\label{fig:grid_concept}
\end{figure}

	%%%%%%%%%%%%%%%%%%%%%%%%%%%%%%%%%%%
\newcommand{\functionTaskEnergyConsumption}[2]{
	\functionSignature{atec}
	{\varAtomicTask{}{}}
}
%%%%%%%%%%%%%%%%%%%%%%%%%%%%%%%%%%%%%
\subsection{Energy consumption and availability}
The \textit{atomic task energy consumption} is the energy used by all agents in the system in executing an atomic task $\varAtomicTask{}{}$. Transmission energy, $\varTransmissionEnergy{}{}$, the power used by a node to  broadcast a message to allocate a task to another node, or reply with a task result. Receiver energy is the energy used by an agent to receives a message. Both transmission and receiving in an atomic task sequence of allocations involve all members of the arc twice, apart from the sink and sensing nodes which will only transmit and receive once. Note, we make the assumption that the energy of a broadcast of receipt of a message is independent of range. 
\begin{equation}
	\functionTaskEnergyConsumption{}{} 
	= 2\varTransmissionEnergy{}{} (\funcSize{\functionTaskArc{}{}}{} - 1)
	+ 2\varReceiverEnergy{}{} (2 \funcSize{\functionTaskArc{}{}}{} - 1)
\end{equation}

\newcommand{\functionSystemEnergyConsumption}[2]{
	\functionSignature{ctec}
	{\setCompositeTask{}{}}
}
We can then go on to define the \textit{composite task energy consumption} as the energy used by all of the composite tasks $\setCompositeTask{}{}$ that are completed within a time period $\setTime{}{}$.
\begin{equation}
	\functionSystemEnergyConsumption{}{} 
	= 
	\sum_{\forall \varCompositeTask{}{} \in \setCompositeTask{}{}}
	\sum_{\forall \varAtomicTask{}{} \in \varCompositeTask{}{}} \functionTaskEnergyConsumption{}{}
\end{equation}

Although nodes would still use energy when in an idle power saving mode, or sleep mode, we disregard these for this formulation as we look to optimise the active node power usage only. There are other algorithms that can be used to optimise the cycling of power cell usage \cite{DUMMY}.

%%%%%%%%%%%%%%%%%%%%%%%%%%%%%%%%%%%%%%%%%
\newcommand{\formalAgentEnergyAvailable}[2]{
	\functionFormal{fae}
	{\setAgents{}{}}
	{\setRealNumbersUnit{}{}}
}
\newcommand{\functionAgentEnergyAvailable}[2]{
	\functionSignature{fae_{\varTime{}{}}}
	{\varAgent{}{}}
}
\newcommand{\functionEnergyVariability}[2]{
	\ifx \\#1\\
	\functionSignature{rev_{\varTime{}{}}}
	{\setAgents{}{}}
	\else
	\functionSignature{rev_{\varTime{}{}}}{#1}
	\fi
}

\newcommand{\functionEnergyAvailable}[2]{
	\ifx \\#1\\
	\functionSignature{ea_{\varTime{}{}}}{\setAgents{}{}}
	\else
	\functionSignature{ea_{\varTime{}{}}}{#1}
	\fi
}
The \textit{fractional agent energy} maps the an agent to its available energy, as a fraction of the batteries' maximum capacity, $\formalAgentEnergyAvailable{}{}$. We can then specify the \textit{fractional energy availability}, as the sum of the agent energy of all agents in a set $\setAgents{}{}$.
\begin{equation}
	\functionEnergyAvailable{}{} 
	= \dfrac{\sum_{\forall \varAgent{}{} \in \setAgents{}{}} \functionAgentEnergyAvailable{\varAgent{i}{}}{}}
	{\funcSize{\setAgents{}{}}}
\end{equation}
To optimise for distribution of energy usage, we want to minimise the variance\footnote{Using the standard definition of variability of a discrete set $X$, $\sigma^2(X) = \frac{\sum (x_i - \bar{x})^2}{\funcSize{X}-1}$} of $\functionAgentEnergyAvailable{}{}$. As we look to optimise by maximising across multiple goals, and the values of $\functionAgentEnergyAvailable{}{}$ are bounded by $[0, 1]$, we can rephrase this as maximising the distance between the variance and the maximum possible, $1/4$. So we specify the \textit{relative energy variability},
\begin{equation}     	
	\functionEnergyVariability{}{} 
	= \frac{1}{4} - \sigma^2(
	\lbrace \functionAgentEnergyAvailable{}{}
	\rbrace_{\forall \varAgent{}{} \in \setAgents{}{}}
	)
\end{equation}



	\paragraph{Task quality}
\label{section:task_quality}
%%%%%%%%%%%%%%%%%
\newcommand{\functionAtomicTaskQualitySymbol}[2]{\functionSymbol{atq_{#1}^{#2}}}
\newcommand{\functionAtomicTaskQualitySignature}[2]{
	\ifx&#1&
	\functionSignature{\functionAtomicTaskQualitySymbol{}{}} {\varAtomicTask{}{}, \varAgent{}{}}
	\else
	\functionSignature{\functionAtomicTaskQualitySymbol{}{}}{#1, #2}
	\fi
}
\newcommand{\functionAtomicTaskQualitySensor}[2]{
	\functionSignature{\functionAtomicTaskQualitySymbol{}{}} {\varAtomicTask{}{}, \functionDetectorRole{}{}}
}

\newcommand{\functionComponentTaskValue}[2]{
	\functionSignature{ctv}{\varAtomicTask{}{}}
}
\newcommand{\formalComponentTaskValue}[2]{
	\functionFormal{ctv}{\setAtomicTask{}{}}{\setRealNumbersNonNegative{}{}}
}
\newcommand{\functionSystemUtility}[2]{\functionSignature{utility}{\setTime{}{}}}
%%%%%%%%%%%%%%%%%%%
\newcommand{\functionCompositeTaskQuality}[2]{
	\functionSignature{taq_\varTime{}{}}{\varCompositeTask{}{}}
}
\newcommand{\functionTaskAbsoluteValue}[2]{
	\functionSignature{atv}{\varCompositeTask{}{}, \varAtomicTask{}{}}
}

\newcommand{\functionRelativeDistance}[2]{
	\functionSignature{dist}{\varAtomicTask{}{}, \varAgent{}{}}
}
%%%%%%%%%%%%%%%%%%%%%
\newcommand{\formalExecutionRange}[2]{
	\functionFormal{execrange}
	{\setAtomicTask{}{} \times \setAgent{}{}}
	{\setRealNumbersNonNegative{}{}}
}
\newcommand{\functionExecutionRange}[2]{
	\functionSignature{execrange}
	{\varAtomicTask{}{}, \varAgent{}{}}
}
An agent will complete an atomic task with a certain \textit{atomic task quality}, how well it has performed the task. This value is dependant on two components,
\begin{itemize}
	\item \textit{execution range}, the distance between the agent and the demand point of the task. As the maximum distance between any agent and demand point in our normalised coordinate system\footnote{As discussed in Section \ref{section:task_and_resources:distribution}} is $\sqrt{2}$, we can scale $\funcSize{ \functionTaskDemandPoint{}{} - \functionSignature{deploy}{\varAgent{}{}}}$ to the range $\lbrack 0, 1 \rbrack$, ensuring the quality also stays in this range.
	\item \textit{resource fraction}, the fraction of the resources the task requests for execution that the agent has actually dedicated to tasks of that type. To simplify the quality function, we assume reducing resource allocation for a task has an equal and linear impact across all tasks and resources. However, a more complex resource fraction component could easily be defined given a specific systems' set of resources and tasks.
\end{itemize}

\begin{equation}
	\label{eq:atomic_task_quality}
	\functionAtomicTaskQualitySignature{}{} = 
\bigg (
\underbrace{
	1 - \frac{\funcSize{
			\functionTaskDemandPoint{}{} - \functionDeployment{\varAgent{}{}}{} 
		}{}}{{\sqrt{2}}}
}_{\text{distance decay}}
\bigg )
\bigg (
\underbrace{
	\frac{1}{\funcSize{\setResource{}{}}}\sum_{\varResource{}{} \in \setResource{}{}} 
	\frac
	{\functionTaskResourceAllocationInstance{}{}}
	{\functionRequiredResourcesInstance{}{}}
}_{\text{resource fraction}}
\bigg )
\end{equation}

When a composite task is completed, the value of an atomic task to its final outcome can be measured. This is not the same as task quality, at it may be dependent on other factors. For example, 
an atomic task may may have a demand point very close to one that has already been completed as part of the composite task, making it less valuable than if it was significantly separated from other tasks' demand points. Therefore, we define the \textit{component task value} as the mapping, $\formalComponentTaskValue{}{}$,  of each atomic task of a composite task to the fractional value of each of them to the composite tasks' completion, such that $\sum\limits_{\forall \varAtomicTask{}{} \in \varCompositeTask{}{}} \functionComponentTaskValue{}{} = 1$.

For example, the longer the sample time of the measurement the more accurate and higher quality the reading will be, but at the expense of using more energy resources. 
\example{Accuracy and sample times in an environment sensor}{}

	\section{Optimising for system utility}
	\paragraph{Composite task quality}
\label{section:composite_task_quality}
The quality of a composite task will not only be dependent on the value of its atomic tasks, but also on how well nodes in their respective task-paths have minimised energy usage, and energy distribution. Therefore we define the \textit{composite task quality} with these multiple objectives included as terms,
\begin{equation}
	\label{eq:ctv}
	\functionCompositeTaskQuality{}{} = 
	\sum\limits_{\forall \varAtomicTask{}{} \in \varCompositeTask{}{}}
	\big\lbrack
	\alpha\underbrace{\functionEnergyAvailable{\functionTaskArc{}{}}{}}_{\text{energy available}}
	+ \beta\underbrace{\functionEnergyVariability{\functionTaskArc{}{}}{}}_{\text{energy distribution}}
	+ 
	\gamma\underbrace{
		\functionComponentTaskValue{}{}
		\functionAtomicTaskQualitySignature{\varAtomicTask{}{}}{\functionDetectorRole{}{}}
	}_{\text{task quality}}
\big\rbrack
\end{equation}
where $\alpha$, $\beta$, and $\gamma$, are proportions chosen at system initialisation to weight the influence of energy, distribution, and task quality respectively. Each \textit{atomic task's absolute value} to the system will then be the product of the respective composite task's quality and the fractional contribution of the atomic task to that quality. This value can be used by agents in a task-path to measure how useful their actions were in completing the atomic task overall, as judged by sinks that completed the associated composite task.
\begin{equation}
	\functionTaskAbsoluteValue{}{} = 
	\functionCompositeTaskQuality{}{}
	\functionAtomicTaskQualitySignature{}{}
\end{equation}

\paragraph{System utility}
\label{section:utility}
The overall \textit{ utility} of the system over a time period $\setTime{}{}$ will then be the sum of the composite task qualities of all the composite tasks $\setCompositeTask{}{}$ completed during that period.
	\begin{equation}
		\label{eq:system_utility}
		\functionSystemUtility{}{} = \sum\limits_{\varTime{}{} \in \setTime{}{}}
		\sum\limits_{\forall \varCompositeTask{}{} \in \setCompositeTask{\setTime{}{}}{}}
		\functionCompositeTaskQuality{}{}
	\end{equation}
The goal of a system of agents $\setAgents{}{}$ is to maximise $\functionSystemUtility{}{}$ over the lifetime of the WSN.

\reviewquestionopen{There seems two ways to read this section and neither seems valid. You could be saying that (i) each of the four objectives is separate and you are trying to optimise against each separately and so will weight each of the four separately in judging success; or (ii) the four objectives are combined into the weighted sum of system utility which defines how you judge success. But if (i) is what you intend, then why have you defined taq() and system utility u() functions which already seem to give a weighted sum of the first three items; also, how should item 3 be interpreted when it seems to set an independent measure for each atomic task. Or if (ii) is what you intend, then something is wrong because item 4 is not included in the system utility weighted sum, function u(). I don't understand how the overall success metric is defined.
}

\example{Defining the problem in an ocean-based WSN}{}
	\subsection{System utility}
%%%%%%%%%%%%%%%%%%%%%%%%%%%%
\newcommand{\functionSystemUtility}[2]{\functionSignature{u_{\setTime{}{}}}{\setCompositeTask{}{}}}
%%%%%%%%%%%%%%%%%%%%%%%%%%%%%%%%%
\todo[inline]{Simplify this}
Given the system goal is balanced between the multiple objectives of, minimising the overall energy used by composite tasks, the energy variance between system agents, while maximising the quality of composite tasks, and the composite task coverage, then the utility of the system is,

\begin{definition}[System utility]
	The \textit{system utility} over a time period $\setTime{}{}$ is the sum of the values of all the composite tasks $\setCompositeTask{}{}$ completed during that period.
	\begin{equation}
		\functionSystemUtility{}{} = \sum_{\forall \varCompositeTask{}{} \in \setCompositeTask{}{}}
		\functionCompositeTaskQuality{}{}
	\end{equation}
\end{definition}


	\paragraph{Measuring the robustness of a WSN}

Optimising the utility of the system as defined in Section \ref{section:utility} increases the quality of task completions, minimises energy consumption, and increases energy distribution. In optimising these, the necessary WSN properties of coverage, resilience and system lifetime as discussed in Section \ref{section:background:requirements} should also increase. Using the concept of success of an atomic task, we now define how we can measure these within the system.

%%%%%%%%%%%%%%%%%%%%%%%%%%%%%%
\newcommand{\varQualityMin}[2]{q_\epsilon}

\newcommand{\formalAtomicTaskSuccess}[2]{
	\functionFormal{success}
	{\setAtomicTask{}{}}
	{\setIntegersBinary{}{}}
}
\newcommand{\functionAtomicTaskSuccess}[2]{
	\functionSignature{success}{\varAtomicTask{}{}}
}

\newcommand{\formalCompositeTaskCoverage}[2]{
	\functionFormal{taskcov}
	{\setCompositeTask{}{}}
	{\setRealNumbersUnit{}{}}
}
\newcommand{\functionCompositeTaskCoverage}[2]{
	\functionSignature{taskcov}{\varCompositeTask{}{}}
}


\newcommand{\functionSystemCoverage}[2]{
	\functionSignature{syscov}{\setAtomicTaskInstance{}{}}
}
%%%%%%%%%%%%%%%%%%%%%%%%%%%%%%%%%%%%%%%%%%

\paragraph{Successful task completion}
\label{section:success}
In real-life scenarios, when the quality of an atomic task completion falls below a certain \textit{quality threshold}, $\varQualityMin{}{}$, the result may no longer be useful or relevant to the system, e.g. when an agent takes a sensor measurement a large distance away from the corresponding tasks' demand point.  We formalise this concept as an atomic tasks' \textit{success}, the mapping $\formalAtomicTaskSuccess{}{}$ where;

\begin{equation}
	 \functionAtomicTaskSuccess{}{}
	 = 
	\begin{cases}
		1, & \text{if } \functionAtomicTaskQualitySensor{}{} > \varQualityMin{}{} \\
		0, & \text{otherwise}
	\end{cases}
\end{equation}

\paragraph{Coverage}
\label{section:coverage}
Given a set of completed atomic tasks, $\setAtomicTaskInstance{}{} \subseteq \setAtomicTask{}{}$, then the \textit{system coverage} \footnote{See definition of key requirements, \ref{requirement:coverage}, in Section \ref{section:background}} of these tasks is the fraction of those tasks which were successful;
\begin{equation}
	\label{eq:coverage}
	\functionSystemCoverage{}{}
	=
	\frac{1}{\funcSize{\setAtomicTaskInstance{}{}}}
	\sum\limits_{\forall \varAtomicTask{}{} \in \setAtomicTaskInstance{}{}}
	\functionAtomicTaskSuccess{}{}
\end{equation}

%%%%%%%%%%%%%%%%%%%%%%%%%%
\newcommand{\functionSymbolResilence}[2]{\functionSymbol{resilience}{#1}{#2}}
\newcommand{\functionResilence}[2]{
	\functionSignature{\functionSymbolResilence{#1}{#2}}
	{\setAtomicTaskInstance{}{}, \setAgent{}{'}, \setAgent{}{} }
}
\paragraph{Resilience}
\label{section:resilience}
As network connectivity is disrupted by node failures, communication problems, and weather effects, it may become complex to route all of the atomic tasks to nodes that can successfully complete them. Specifically, if a system completes a set of atomic tasks $\setAtomicTaskInstance{}{}$, with available agents $\setAgent{}{'}$, out of the system agents $\setAgent{}{}$, then the systems \textit{resilience} can be defined as; 
\begin{equation}
	\functionResilence{}{}
	= 
	\frac{
		\functionSystemCoverage{}{}
	}{
		\funcSize{\setAgents{}{'}} / \funcSize{\setAgents{}{}}
	}
\end{equation}
In this way, a system whose coverage remains high as the number of available agents falls has a higher resilience that a system who coverage drops lower under the same circumstances.

 \paragraph{Lifetime}
 \label{section:lifetime}
 \newcommand{\varCoverageMinimum}[2]{\varSymbol{c}{lt}{}}
 Despite a systems resilience, given enough agent failures, there will be a deterioration of coverage with time. A system-specific set of tasks $\setAtomicTaskInstance{}{}$ and minimum coverage value, $\varCoverageMinimum{}{}$, considered for the system to remain useful can be chosen. The \textit{system lifetime} can then be defined as the time until $\functionSystemCoverage{}{} < \varCoverageMinimum{}{}$.
 


\example{Success, coverage, and resilience}{
	\label{example:success}
	A system has a large number of sensor nodes deployed across an environment with rough terrain in which we want to monitor the spread of contamination from a radioactive pollutant. An agent $\varAgent{}{}$ takes a measurement $1$ metre away from a demand point of a task $\varAtomicTask{1}{}$ and $100$ metres from that of $\varAtomicTask{2}{}$. The result of the first task is likely to be close to the actual value in that location, whereas that of the second may be uncorrelated, and not be a practically useful measurement. In this case, the systems designer may set $\varQualityMin{}{}$ such that measurements $>10$ metres away, although they are completed with non-zero quality, are judged unsuccessful as $\functionSignature{atq}{\varAtomicTask{1}{}} < \varQualityMin{}{} < \functionSignature{atq}{\varAtomicTask{2}{}}$.  
}

	\subsection{The multi-objective optimisation problem}
\label{section:optimisation_problem}

The goal of a system of agents $\setAgents{}{}$ is to maximise $\functionSystemUtility{}{}$ over the lifetime of the WSN. In doing so the system should optimise over the multiple objectives below, 
\begin{enumerate}
	\item Maximising $\functionEnergyAvailable{}{}$, the energy available to ensure functionality and coverage of nodes.
	\item Maximising $\functionEnergyVariability{}{}$,  the distribution of energy to prolong system lifetime.
	\item Increase $\functionAtomicTaskQualitySignature{}{}$, the quality of the individual atomic tasks.
\end{enumerate}
There should also be an increase in resilience as the system tries to maintain coverage $\functionSystemCoverage{}{}$ through optimising task quality as node failures occur throughout the systems' lifetime.


\section{Solving the multi-objective WSN problem}
\label{section:solution}
	\section{Solving the multi-objective WSN problem}
\label{section:solution}
As defined in the previous Section, we seek a method to optimise multiple objectives in our WSN system. To do so we will incorporate two algorithms. Firstly, the \acronymATARIAExtended{}{} algorithm will enable agents in the system to learn the best agents to allocate tasks to to obtain the best composite task values, as well as explore the system for other agents, and adapt to impacts and outages. The second, the \acronymMGRAOExtended{}{} algorithm, helps agents executing measurement tasks allocate their resources to optimise the composite task value as well.


%%%%%%%%%%%%%%%%%%%%%%%
\newcommand{\formalATARIA}[2]{
	\functionFormal{ataria}
	{XXX}
	{\powerSetAgents{}{} \times \powerSetAgents{}{} \times (\setAction{}{} \times \setRealNumbers{}{})}
}
%%%%%%%%%%%%%%%%%%%%%%%%%

\subsection{Optimising for task allocation with system exploration}
\todo[inline]{Summary of ATARIA algorithm, connected to our problem notation, why we use it and the outcome}

\subsection{Resource allocation in agents taking measurements}
\todo[inline]{Summary of MGRAO algorithm, connected to our problem notation, why we use it and the outcome}

\subsection{Extension to hierarchical task allocation}

To enable agents to form arcs as described we allow atomic tasks to be re-allocated to further agents in order to reach an agent within range of a demand point. The flowchart in Figure \ref{fig:algorithm-flow} shows how this agents utilise \acronymATARIA{}{} and \acronymMGRAO{}{} with these recursive actions enabled. Figure \ref{fig:arc-flow} illustrates an arc where there are two re-allocations needed before a specific atomic task is allocated to an agent whose sensor is in range of the demand point, and can therefore make a measurement.

\todo[inline]{Integrated view of the algorithms plus a flow diagram for illustration}

\begin{figure}
	\centering
	\includegraphics[width=0.9\linewidth]{algorithm-flow}
	\caption{\textbf{Flow chart of combined \acronymATARIA{}{}/\acronymMGRAO{}{} execution}. The two algorithms are combined together to allow recursive allocation of tasks and learning of the network}
	\label{fig:algorithm-flow}
\end{figure}
\begin{figure}
	\centering
	\includegraphics[width=0.9\linewidth]{arc-flow}
	\caption{\textbf{Allocation along an arc}. This dia gram illustrates how allocations can be relayed along an arc using successive applications of the \acronymATARIA{}{} algorithm.}
	\label{fig:arc-flow}
\end{figure}
	%%%%%%%%%%%%%%%%%%%%
\newcommand{\functionANHTAO}[2]{
	\functionSignature{\texttt{anhtao-path}}{\varAtomicTask{}{}, \varAgent{}{}}
}

\subsection{The \acronymWSNOptimisationExtended{}{} algorithm}
\label{section:solution_anhtao}

\paragraph{Integrating \acronymWSNOptimisation{}{} into task paths}

The \acronymWSNOptimisation{}{} algorithm optimises utility of the system as described in Section \ref{section:energy_consumption}, Equation \ref{eq:system_utility}. 
To successfully complete a composite task, a sink must decompose the composite task it received from an external source into atomic tasks. It then selects whether to execute the tasks itself, or allocate them to further agents.  Figure \ref{fig:arc-flow}  illustrates how our algorithm fits into this process. It shows a task-path where there are two re-allocations made before a specific atomic task is allocated to an agent that completes the task by taking a measurement, then returns the results back along the task path.
\begin{figure}[ht]
	\centering
	\includegraphics[width=0.8\linewidth, trim={72pt 0pt 62pt 0pt, clip}]{arc-flow}
	\caption{\textbf{Allocation along a task-path}. This diagram illustrates how allocations can be relayed along a task-path using successive applications of the \acronymATARIA{}{} algorithm.}
	\label{fig:arc-flow}
\end{figure}

\paragraph{Optimisation using \acronymWSNOptimisation{}{}}

The \acronymWSNOptimisation{}{} algorithm optimises this process  in 3 main ways;
\begin{enumerate}
	\item \textit{Selecting actions.} When agents have a composite or atomic task allocated to them, the can choose from a range of actions. By utilising Q-learning techniques, with absolute task values as the reward function, \acronymWSNOptimisation{}{} adapts the probability of agents' taking actions to optimise these values. 
	
	\item \textit{Resource allocation.} As an agent completes an atomic task it will use some resources to do so. The \acronymWSNOptimisation{}{} algorithm predicts the agents' optimal allocation of these resources, given the different atomic tasks it is allocated, and their distribution over time. This allows it to complete the incoming tasks to obtain best atomic task values overall.
	
	\item \textit{Forming task-paths. }The algorithm allows agents to re-allocate atomic tasks to other agents, relaying them through the network.  It also distributes the reward for completing these tasks across agents in these task paths to optimise the actions that were taken by all agents that participated in the task.
\end{enumerate}
The high-level flowchart in Figure \ref{fig:algorithm-flow}(a) shows how sinks decompose composite tasks, choose actions to take, allocate atomic tasks, and return results. In Figure \ref{fig:algorithm-flow}(b) we can see how agents in a task-path choose actions, and how they execute or re-allocate the atomic tasks they have been allocated.
\begin{figure}[ht]
	\centering
	\begin{subfigure}{0.75\textwidth}
		\centering
		\caption{Sink agent flow}
		\includegraphics[width=0.9\linewidth, trim={25pt 0pt 25pt 0pt, clip}]{algorithm-flow-sink}
		\label{fig:algorithm-flow-sink}
	\end{subfigure} \hfill%
	\begin{subfigure}{0.75\textwidth}
		\caption{Task-path agent flow}
		\centering	\includegraphics[width=0.9\linewidth,trim={25pt 0pt 25pt 0pt, clip}]{algorithm-flow-arc}
		\label{fig:algorithm-flow-arc}
	\end{subfigure}
	\caption{\textbf{\acronymWSNOptimisation{}{} execution flowchart.} Shows how \acronymWSNOptimisation{}{} combines the \acronymATARIA{}{} and \acronymMGRAO{}{} algorithms and enables recursive allocation of tasks.}
	\label{fig:algorithm-flow}
\end{figure}

\paragraph{High-level description of  \acronymWSNOptimisation{}{}}

We separate  the \acronymWSNOptimisation{}{} algorithm in two parts for clarity, the \acronymWSNOptimisationSink{}{} and \acronymWSNOptimisationArc{}{} algorithms. These are shown in Algorithms \ref{alg:wsn_optimisation_sink} and \ref{alg:wsn_optimisation_arc} respectively. 

\begin{itemize}
	\item[]  \textbf{\acronymWSNOptimisationSink{}{}} Initially, the sink receives a composite task $\varCompositeTask{}{}$ comprising of multiple atomic tasks $\varAtomicTask{}{}$ to be completed. While these atomic tasks are not complete or allocated, the sink runs the \acronymATARIA{}{} algorithm to select an action (Line \ref{wsnsink:select}).  If the action chosen is to execute the atomic task itself, $\functionExec{}{}$, the algorithm uses the \acronymMGRAO{}{} algorithm to determine the resources that are allocated to that task types' completion. An $\functionAlloc{}{}$ action will allocate the task to another agent it knows about to complete using the \acronymWSNOptimisationArc{}{} algorithm. In both cases, the task is removed from the list of active tasks  (Line \ref{wsnsink:exec_remove}). If a $\functionInfo{}{}$ or $\functionLink{}{}$ action is executed, these will update the agents' neighbourhood and knowledge base respectively using the \acronymATARIA{}{} algorithm. There is no effect on atomic task execution or allocation in that case, and the algorithm loops round to choose another action. The selection and execution of actions using the \acronymATARIA{}{} algorithm is repeated until the tasks are either executed or allocated. Once all the atomic tasks are allocated, the sink must wait for them to complete (Line \ref{wsnsink:wait}). When all the atomic tasks have completed, each atomic tasks' absolute task value is calculated (Line \ref{wsnsink:ataria-taskval}). All of the agents in each atomic tasks' task-path have their Q-values updated using these values, acting as a reward value for the actions they took in completing each task (Line \ref{wsnsink:ataria-update}). Finally, for each atomic task, the corresponding absolute task value is used by each agent that completed the task, to update its allocation of resources using the \acronymMGRAO{}{} algorithm (Line \ref{wsnsink:mgrao}).

	\item[] \textbf{\acronymWSNOptimisationArc{}{}} This part of the algorithm will complete an atomic task and return its quality, either by executing the task itself, or re-allocating to another agent.  Once again,  the \acronymATARIA{}{} algorithm is used to repeatedly select actions until the atomic task is either executed or allocated (Line \ref{wsnarc:select}). After waiting for the atomic task to be completed (Line \ref{wsnarc:wait}),  an atomic task quality is returned (Line \ref{wsnarc:wait}).
\end{itemize}
 \begin{algorithm}[ht]
	\DontPrintSemicolon
	\footnotesize
	
	\caption{\textbf{The \acronymWSNOptimisationSink{}{} algorithm}}
	\label{alg:wsn_optimisation_sink}
	{
		\KwIn{ $\varCompositeTask{}{}$ , The composite task set}
		\KwIn{ $\varAgent{}{}$ , The sink completing the composite task}
		\KwResult{$\functionCompositeTaskQuality{}{}{}{}$ , The composite task quality of $\varCompositeTask{}{}$}		\nonl \;
		\tcp{Copy set of atomic tasks to list of incomplete tasks}
		$ctactive \leftarrow \varCompositeTask{}{}$ \label{wsnsink:copy}\;
		\ForEach{$\varAtomicTask{}{} \in ctactive$\label{wsnsink:composite_tasks}}
		{
			\tcp{Select and execute action through \acronymATARIA{}{}}
			$\varAction{}{} \leftarrow \functionATARIAAction{}{}$ \label{wsnsink:select} \;	
			\If{$\functionAgentActionType{}{} = \functionExec{}{} \lor  \functionAgentActionType{}{} = \functionAlloc{\varAtomicTask{}{}}{\varAgent{}{'}}$?}
			{
				\tcp{If the action was completed by agent $\varAgent{}{}$ or by allocating to another agent $\varAgent{}{'}$, remove atomic task from active task list}
				$ctactive{}{} \leftarrow ctactive - \lbrace \varAtomicTask{}{} \rbrace$ \label{wsnsink:exec_remove}\;
			}
		}
		\tcp{Wait for all the atomic tasks in the composite task to be completed}
		\While{ $\functionNotComplete{\varCompositeTask{}{}}{}$}{
			$\functionWait{}{}$ \label{wsnsink:wait}\; 
		}
		\ForEach{$\varAtomicTask{}{} \in \varCompositeTask{}{}$\label{ataria:composite_tasks}}
		{
			\tcp{Calculate each atomic tasks' absolute task value}
			$\varTaskValue{}{} \leftarrow \functionTaskAbsoluteValue{}{}
			$ \label{wsnsink:ataria-taskval}\;	
			\tcp{Send a proportion of the absolute task value for the atomic task to each agent in its task-path}
			\ForEach{$\varAgent{}{'}\in \functionTaskArc{}{}$}{
				\tcp{Update the Q-values for actions taken by agents in each atomic tasks' task-path}
				$\functionSignature{\texttt{ataria-update}}{
					\varAgent{}{'}, \varAtomicTask{}{}, \frac{\varTaskValue{}{}}{\funcSize{\varCompositeTask{}{}}}
				}$ \label{wsnsink:ataria-update}\;	
			}
			\tcp{Send the agent that completed the atomic task the absolute task value to run the MGRAO update}
			$\functionSignature{\texttt{mgrao-update}}{\functionSinkRole{}{}, \varAtomicTask{}{}, \varTaskValue{}{}}$ \label{wsnsink:mgrao}\;	
		}
		\Return{$\functionCompositeTaskQuality{}{}{}{}$}
	}
\end{algorithm}
\begin{algorithm}[ht]
	\DontPrintSemicolon
	\footnotesize
	
	\caption{\textbf{The \acronymWSNOptimisationArc{}{} algorithm } }
	\label{alg:wsn_optimisation_arc}
	{
		\KwIn{ $\varAtomicTask{}{}$ , The atomic task to be completed}
		\KwIn{ $\varAgent{}{}$ , The agent completing the atomic task}
		\KwResult{$\functionAtomicTaskQualitySignature{}{}$ , The atomic task quality of $\varAtomicTask{}{}$}
		\nonl \;
		
		$taskAllocated \leftarrow False$ \;
		\While{$\neg taskAllocated$ }{
			\tcp{Select and execute action through \acronymATARIA{}{}}
			$\varAction{}{} \leftarrow \functionATARIAAction{}{}$ \label{wsnarc:select} \;
			\tcp{If the action was executed by agent $\varAgent{}{}$ or allocated to another agent $\varAgent{}{'}$, mark it allocated}
			\uIf{$\functionAgentActionType{}{} = \functionExec{}{} \lor \functionAgentActionType{}{} = \functionAlloc{\varAtomicTask{}{}}{\varAgent{}{'}}$?}
			{
				$taskAllocated \leftarrow True$ \;
			}
		}
		\tcp{Wait for the atomic task to be completed}
		\While{ $\functionNotComplete{\varAtomicTask{}{}}{}$}{
			$\functionWait{}{}$ \label{wsnarc:wait}\; 
		}
		\tcp{Return the task quality of the atomic task completion}
		\Return{$\functionAtomicTaskQualitySignature{}{}$\label{wsnarc:return}} \;
	}
\end{algorithm}
	\subsection{Formal definition of the \acronymWSNOptimisation{}{} algorithm}

We formally define the \acronymWSNOptimisation{}{} algorithm in two parts, Algorithm \ref{alg:wsn_optimisation_sink} for sink node agents receiving composite tasks, and Algorithm \ref{alg:wsn_optimisation_arc} for other agents belonging to the task-path for an atomic task.

In the \acronymWSNOptimisationSink{}{} algorithm, Algorithm \ref{alg:wsn_optimisation_sink}, the sink node receives a composite task $\varCompositeTask{}{}$ comprising of multiple atomic tasks $\varAtomicTask{}{}$ to be completed. For each of these atomic tasks the sink node agent runs the \acronymATARIA{}{} algorithm to select an action (Line \ref{wsnsink:select}). It can execute the atomic task itself (Line \ref{wsnsink:exec}), using the \acronymMGRAO{}{} algorithm to determine the resources that will be allocated to its completion, or allocate it to another agent it knows about to complete using the \acronymWSNOptimisationArc{}{} algorithm (Line \ref{wsnsink:alloc}). In both cases, once the task is completed, the sink node agent receives an atomic task quality value of $\functionAtomicTaskQualitySignature{}{}$ and removes the atomic task from the list of tasks to complete. 

If a $\functionInfo{}{}$ or $\functionLink{}{}$ action is executed, there is no effect on atomic task execution or allocation in that step, and the algorithm loops round to choose another action using \acronymATARIA{}{}.

Once all atomic tasks have completed, each atomic tasks' proportional value to the composite task is calculated. For each atomic task, the corresponding value is sent to the last agent in the atomic tasks' task-path so that each sensor agent can carry out an \acronymMGRAO{}{}-update to re-allocate its resources to increase the tasks' value in the future (Lines \ref{wsnsink:arc_last} and \ref{wsnsink:mgrao}). 

\begin{algorithm}[ht]
	\DontPrintSemicolon
	\footnotesize
	
	\caption{\textbf{The \acronymWSNOptimisationSink{}{} algorithm}}
	\label{alg:wsn_optimisation_sink}
	{
		\KwIn{ $\varCompositeTask{}{}$ , The composite task set}
		\KwIn{ $\varAgent{self}{}$ , The sink agent completing the composite task}
		\KwResult{$\functionCompositeTaskQuality{}{}{}{}$ , The composite task quality of $\varCompositeTask{}{}$}		\nonl \;
		\tcp{Copy set of atomic tasks to list of incomplete tasks}
		$ctactive \leftarrow \varCompositeTask{}{}$ \label{wsnsink:copy}\;
		\For{$\varAtomicTask{}{} \in ctactive$\label{wsnsink:composite_tasks}}
		{
			\tcp{Select action through \acronymATARIA{}{}}
			$\varAction{}{} \leftarrow \functionATARIA{}{}$ \label{wsnsink:select} \;	
			\uIf{$\varAction{}{} = \functionExec{}{}$?}
			{
				\tcp{Execute task $\varAtomicTask{}{}$ and get an atomic task quality} 
				$\functionAtomicTaskQualitySignature{}{} \leftarrow \functionExec{}{}$ \label{wsnsink:exec}\;
				\tcp{Remove atomic task from incomplete task list}
				$ctactive{}{} \leftarrow ctactive - \lbrace \varAtomicTask{}{} \rbrace$ \label{wsnsink:exec_remove}\;
			}
			\uElseIf{$\varAction{}{} = \functionAlloc{}{}$?}{
				\tcp{allocate task $\varAtomicTask{}{}$ to agent} $\functionAtomicTaskQualitySignature{}{} \leftarrow \functionANHTAO{}{}$\label{wsnsink:alloc} \;
				\tcp{Remove atomic task from incomplete task list}
				$ctactive{}{} \leftarrow ctactive{}{} - \lbrace \varAtomicTask{}{} \rbrace$ \label{wsnsink:alloc_remove}\;
			}
			\uElseIf{$\varAction{}{} = \functionInfo{}{}$?}{
				\tcp{request information on system agents from agent $\varAgent{}{}$}
				$\functionInfo{}{}$ \label{wsnsink:info}\;
			}
			\uElseIf{$\varAction{}{} = \functionLink{}{}$?}{
				\tcp{allocate resources to information on to agent $\varAgent{}{}$ and maintaining network connection}
				$\functionLink{}{}$ \label{wsnsink:link}\;
			}
		}
		\For{$\varAtomicTask{}{} \in \varCompositeTask{}{}$\label{ataria:composite_tasks}}
		{
			\tcp{Target update at the last agent in the task-path, the agent that completed the atomic task}
			$\varAgent{}{} \leftarrow \functionSenseRole{}{}$ \label{wsnsink:arc_last}\;
			\tcp{Send the sensor agent the atomic task value to run the MGRAO update}
			$\functionMGRAOUpdate{}{}$ \label{wsnsink:mgrao}\;	
		}
		\Return{$\functionCompositeTaskQuality{}{}{}{}$}
	}
\end{algorithm}

The \acronymWSNOptimisationArc{}{} algorithm (Algorithm \ref{alg:wsn_optimisation_arc}) will complete an atomic task and return its quality, either by executing the task itself (Line \ref{wsnarc:exec}), or re-allocating to another agent (Line \ref{wsnarc:alloc}). The \acronymATARIA{}{} algorithm is run and an action selected (Line \ref{wsnarc:select}), in the same way as in the \acronymWSNOptimisationSink{}{} algorithm. The selection and execution of actions using the \acronymATARIA{}{} algorithm is repeated until the task in completed an atomic task quality can be returned (Line \ref{wsnarc:exec}).
\begin{algorithm}[ht]
	\DontPrintSemicolon
	\footnotesize
	
	\caption{\textbf{The \acronymWSNOptimisationArc{}{} algorithm } }
	\label{alg:wsn_optimisation_arc}
	{
		\KwIn{ $\varAtomicTask{}{}$ , The atomic task to be completed}
		\KwIn{ $\varAgent{self}{}$ , The agent completing the composite task}
		\KwResult{$\functionAtomicTaskQualitySignature{}{}$ , The atomic task quality of $\varAtomicTask{}{}$}
		\nonl \;
		
		$taskComplete \leftarrow False$ \;
		\While{$\neg taskComplete$ }{
		\tcp{Select action through \acronymATARIA{}{}}
		$\varAction{}{} \leftarrow \functionATARIA{}{}$ \label{wsnarc:select} \;	
		\uIf{$\varAction{}{} = \functionExec{}{}$?}
		{
			\tcp{Execute task $\varAtomicTask{}{}$ and get an atomic task quality} 
			$\functionAtomicTaskQualitySignature{}{} \leftarrow \functionExec{}{}$ \label{wsnarc:exec}\;
			$taskComplete \leftarrow True$ \;
		}
		\uElseIf{$\varAction{}{} = \functionAlloc{}{}$?}{
			\tcp{allocate task $\varAtomicTask{}{}$ to agent} $\functionAtomicTaskQualitySignature{}{} \leftarrow \functionANHTAO{}{}$\label{wsnarc:alloc} \;
			
			\tcp{allocate task $\varAtomicTask{}{}$ to agent $\varAgent{}{}$}
			$\functionAtomicTaskQualitySignature{}{} \leftarrow \functionAlloc{}{}$\label{wsnarc:alloc} \;
			$taskComplete \leftarrow True$ \;
			}
		\uElseIf{$\varAction{}{} = \functionInfo{}{}$?}{
			\tcp{request information on system agents from agent $\varAgent{}{}$}
			$\functionInfo{}{}$ \label{wsnarc:info}\;
		}
		\uElseIf{$\varAction{}{} = \functionLink{}{}$?}{
			\tcp{allocate resources to information on to agent $\varAgent{}{}$ and maintaining network connection}
			$\functionLink{}{}$ \label{wsnarc:link}\;
		}
	}
	\Return{$\functionAtomicTaskQualitySignature{}{}$\label{wsnarc:return}} \;
	}
\end{algorithm}

	\subsection{Optimisation algorithms for task allocation and resource allocation}
\label{section:algorithm_summaries}
%%%%%%%%%%%%%%%%
\newcommand{\varAction}[2]{\varSymbol{a}{#1}{#2}}
\newcommand{\functionExec}[2]{
	\ifx &#1&
	\texttt{exec}(\varAtomicTask{}{})
	\else
	\texttt{exec}(#1, #2)
	\fi
}
\newcommand{\functionAlloc}[2]{
	\ifx &#1&
	\texttt{alloc}(\varAtomicTask{}{}, \varAgent{}{})
	\else
	\texttt{alloc}(#1, #2)
	\fi
}
\newcommand{\functionInfo}[2]{
	\ifx &#1&
	\texttt{info}(\varAgent{}{})
	\else
	\texttt{info}(#1)
	\fi
}
\newcommand{\functionLink}[2]{
	\ifx &#1&
	\texttt{link}(\varAgent{}{})
	\else
	\texttt{link}(#1)
	\fi
}
\newcommand{\functionATARIA}[2]{
	\functionSignature{
		ataria_{\varAgent{}{}}
	}{
		\varAtomicTask{}{}, \varAgent{self}{}
	}
}	
\newcommand{\formalATARIA}[2]{
	\functionFormal{\texttt{ataria}_{\varAgent{}{}}}
	{\setAtomicTask{}{} \times \setAgents{}{}}
	{
		\texttt{exec}(\setAtomicTask{}{})
		\times \texttt{alloc}(\setAtomicTask{}{}, \setAgents{}{})
		\times \texttt{info}(\setAgents{}{})
		\times \texttt{link}(\setAgents{}{})
	}
}
\newcommand{\functionMGRAOWeighting}[2]{\texttt{mgrao-weight}(\varAtomicTask{}{}, \varAgent{self}{})}
\newcommand{\formalMGRAOWeighting}[2]{
	\functionFormal{\texttt{mgrao-weight}_{\varAgent{}{}}}
	{\setAtomicTask{}{} \times \setRealNumbers{}{}}
	{
		\setRealNumbers{}{}
	}
}
\newcommand{\functionMGRAOUpdate}[2]{
	\texttt{mgrao-update}_{\varAgent{}{}}
	(\varAtomicTask{}{}, \functionTaskAbsoluteValue{}{})}
\newcommand{\formalMGRAOUpdate}[2]{
	\functionFormal{\texttt{mgrao-update}_{\varAgent{}{}}}
	{\setAtomicTask{}{} \times \setRealNumbers{}{}}
	{
		\setRealNumbers{}{}
	}
}
%%%%%%%%%%%%

The \acronymATARIA{}{} algorithm enables agents in the system to learn the best actions to take given their current state. This ranges from deciding which other agents to allocate tasks to and obtain the best composite task values, to exploring the system for other agents, while adapting connectivity to handle network disruption. An agent uses the \acronymATARIA{}{} algorithm to choose an action to take, which can be one of the following,
\begin{enumerate}
	\item $\functionExec{}{}$, The agent will execute the atomic task $\varAtomicTask{}{}$ itself.
	\item $\functionAlloc{}{}$, the agent will allocate the atomic task $\varAtomicTask{}{}$ to another agent $\varAgent{}{}$.
	\item $\functionInfo{}{}$, the agent will request information from another agent $\varAgent{}{}$.
	\item $\functionLink{}{}$, the agent will allocate resources to hold information on the agent $\varAgent{}{}$ and maintain a connection.
\end{enumerate}
\begin{figure}[ht]
	\centering
	\includegraphics[width=0.8\linewidth, trim={25pt 0pt 25pt 0pt, clip}]{ataria-simplified}
	\caption{\textbf{Simplified \acronymATARIA{}{} flowchart}. XXX.}
	\label{fig:arc-flow}
\end{figure}

The \acronymATARIA{}{} algorithm learns to select the actions that generate the best composite task values, and adapt the choice of action depending on how good the composite task values are in comparison to the historical values through the updating of Q-values in its temporal-update algorithm. The algorithm will select one of the possible actions for the agent $\varAgent{}{}$, given it has non-completed, allocated tasks $\setAtomicTask{}{}$, and knows of other agents $\setAgents{}{}$ through the function,
\begin{equation}
	\label{eq:ataria}\formalATARIA{}{}
\end{equation}

To enable agents to form task-paths we allow atomic tasks that have been allocated to an agent to be either executed by that agent, or re-allocated to further agents, through running the \acronymATARIA{}{} algorithm. Figure \ref{fig:arc-flow} illustrates a task-path where there are two re-allocations made before a specific atomic task is allocated to an agent that completes the task by taking a measurement.
\begin{figure}[ht]
	\centering
	\includegraphics[width=0.8\linewidth, trim={25pt 0pt 25pt 0pt, clip}]{arc-flow}
	\caption{\textbf{Allocation along a task-path}. This diagram illustrates how allocations can be relayed along a task-path using successive applications of the \acronymATARIA{}{} algorithm.}
	\label{fig:arc-flow}
\end{figure}
	\subsection{Adapting resource allocation using the \acronymMGRAO{}{} algorithm}
\label{section:solution_mgrao}

An agent requires resources to complete an atomic task. We assume that allocating different amounts of its available resources to completing that task will effect the final absolute task value received. The agent can then optimise its performance by adjusting its allocation of  resources in response to these reward values. The \acronymMGRAO{}{} algorithm does this through maintaining a resource allocation weights matrix $\matrixResourceWeight{}{}$, which it updates as it receives $\functionTaskAbsoluteValue{}{}$ values for completing tasks. It also keeps an eligibility trace matrix, $\matrixEligibilityTrace{}{}$, allowing it to know how long ago it took each type of action. Using these matrices it generates a set of combined resource weights, $\functionCombinedResourceWeightsSignature{}{}$, which gives the actual allocations of its resources to each possible action-type.

The algorithm is comprised of two sub-algorithms, an update algorithm, and a weighting algorithm. The update algorithm will change the resource weightings for an agent, $\functionAgentResources{}{}$, given the type of an atomic task completed, and the absolute task value returned when the corresponding composite task is completed. The weighting algorithm simply returns the resource weighting for calculation of an atomic tasks' quality $\functionAtomicTaskQualitySignature{}{}$ on its completion. A simplified flowchart for the integration of the \acronymMGRAO{}{} algorithm is shown in Figure \ref{fig:mgrao-simplified}, with the steps below.

 \paragraph*{Simplified \acronymMGRAO{}{} flowchart}
\begin{enumerate}

	\item[(1)] The agent processes its current atomic tasks. 
	\item[(2-5)] If the task is a request for information, the agent will randomly select knowledge of other system agents from its knowledge base and return this information to the requester. 
	\item[(6-8)] If the agent receives a reward for a previously completed task, the agent will update its eligibility matrix using the task type. This allows the reward  to be spread across a number of previous past actions. It will also use the reward value to adjust its resource allocations across the possible task types.  
\end{enumerate}

\begin{figure}[ht]
	\centering
	\includegraphics[width=0.8\linewidth, trim={55pt 0pt 55pt 0pt, clip}]{mgrao-simplified}
	\caption{\textbf{Simplified \acronymMGRAO{}{} flowchart}. The diagram shows the decision-making and actions taken by an agent using the \acronymMGRAO{}{}  algorithm. This is a combination of two sub-algorithms, \acronymMGRAO{}{}(weighting), that returns the amount of the agents' resources that are allocated to completing the requested task type, and  \acronymMGRAO{}{}(update), that updates its resource weightings from rewards for previous atomic task completions.}
	\label{fig:mgrao-simplified}
\end{figure}


	\section{Evaluation of Environmental Wireless Sensor Networks (E-WSN)}
\label{section:experimental}	

To define our simulation clearly we will base it on a realistic deployment scenario as covered by Gomez et al \cite{Gomez} and others \cite{Jha2016, Avram}. We will use a UAV to deploy a large number of sensors over a large and remote geographical area, leading to a relatively ad-hoc, randomised placement of devices. They will use solar power cells to maintain enough energy to power themselves over a number of years, making clear the requirement for low power usage. Noting that the transceiver is the significant power drain on these sensors we can see that reducing its utilisation will be one of the keys to managing energy usage. We will design our sensors to use \textit{Wake Up Radio} (WUR) and aim our agent deployment strategy towards minimising its use and range. As mentioned, the other main driver of the way we use the agents to control the sensors will be the resilience and adaptivity of the system over long periods of time, with no human intervention possible. 

We make the following assumptions and simplifications.
\begin{itemize}{
		\item Node deployment is randomly distributed across the grid following the uniform distribution or a radial normal distribution.
		\item Energy harvesting works continuously at constant rate. This assumes consistent solar exposure and limits the simulation to daytime.
		\item Permanent node failure is based on the total energy consumption of that node over time, as a proxy for wear.
		\item Temporary loss of availability of nodes is random and at a constant rate.
	}
\end{itemize}

Example values in \ref{table:components_energy_usage} show the energy and time costs of varios operations given an hourly sampling period for a node.


\begin{table}[h]
	\begin{tabular}{p{0.3\textwidth}p{0.2\textwidth} p{0.2\textwidth} p{0.2\textwidth}}
		\hline
		Function & power(mW) & time (s) & energy (mJ)\\
		\hline
		Data acquisition (\symbolDataAcquisition{}{}) & 0.5 & 60 & 30 \\
		Transmission (\symbolTransmission{}{}) & 7 & 0.1 & 0.7 \\
		Receiver (\symbolReceiver{}{}) & 70 & 0.1 & 0.7 \\
		Idle wake-up-radio (\symbolWakeUpRadio{}{}) & 0.07 & 3509 & 246  \\
	\end{tabular}
	\caption{Component types and energy usage}
	\label{table:components_energy_usage}
\end{table}

\todo[inline]{Describe the comparison of network task layouts}
\begin{figure}[ht]
	\centering
	\includegraphics[width=0.9\linewidth]{network-types}
	\caption{\textbf{Experimental network type}. The diagram shows two arcs possible for the system. In arc A, the maximum quality for the task is achieved, however, there is more energy usage overall. In arc B, energy is conserved but the task is completed to a lesser quality}
	\label{fig:network-types}
\end{figure}


	
The \simulationExtended{}{} system had CTV component weightings where each of the relevant properties were given an $80\%$ dominance over the value of CTV value. The sink node was given $10$ measurements to allocate, with no repetition. It was also placed at a significantly large distance from the demand points associated with the tasks. $25$ nodes were distributed randomly in the system. This system examined the impact of the algorithm optimising the allocation of tasks towards the goals stated in Section \ref{section:optimisation_problem}. Atomic task quality could be maximised, but at the cost of longer arcs and therefore energy usage, or energy consumption could be minimised, with correspondingly lower task qualities. Figure \ref{fig:route_types} illustrates these two route types for task completion.

Labels, descriptions, and configurations for each algorithm are shown in Table \ref{table:summary_of_configurations}. Results for the \algorithmBalanced{}{} algorithm in the \simulationSimple{}{} system, and the \algorithmEnergy{}{}, \algorithmQuality{}{}, \algorithmDistribution{}{} algorithms in the \simulationExtended{}{} system are shown in Table \ref{table:results}. System utility percentages show the summed values of composite tasks per episode, as shown in Equation \ref{eq:system_utility}, compared to the theoretical maximum utility in the system \footnote{Note that the theoretical maximum is not necessarily attainable in all systems, dependent on their randomised node configurations.}, with the percentage optimisations from the first episode to last in Figure \ref{fig:5_ctv-optimal-ctv-gain}. Energy available is presented as a percentage of that of a system containing nodes with full battery charge in Figure \ref{fig:ctv-statistics-energy-available}. We compare the different biases for optimisation across the CTV components using quality-energy balance, task distribution, and arc-depth results. The quality-energy data shown in Figure \ref{fig:ctv-quality-energy-baseline-comparison} uses the \algorithmEnergy{}{} algorithm as a baseline, with the percentage increase or decrease in the average task quality over energy availability components of the CTV equation in Equation \ref{eq:ctv}. The task-distribution in Figure \ref{fig:ctv-task-distribution-comparison} shows the variation in the agents that are completing the tasks, i.e. $\funcSize{set(\setAgents{}{})}{}/\funcSize{\setAgents{}{}}{}$, with higher values representing more tasks being completed by distinct agents, and lower values meaning more agents are completing multiple tasks. Arc depth data in Figure \ref{fig:ctv-arc-depth-comparison} captures how many agents re-allocated each task before it was completed.
	
\begin{table}[h]
	\begin{tabular}
		{|p{0.18\textwidth}|p{0.48\textwidth}|p{0.04\textwidth}|p{0.05\textwidth}|p{0.12\textwidth}|}
		\hline
		\textbf{Algorithm} & \textbf{Summary} & \textbf{Node count} & \textbf{Atomic tasks}  & \textbf{$(\alpha,\beta,\gamma)$}\\
		\hline
		\algorithmBalanced{}{} &  EWSN optimisation algorithm with balanced objectives. & $10$  & $(5,5,5)$   & $(0.33,0.33,0.33)$  \\
		\algorithmQRouting{}{} &  Q-learning algorithm optimising task-paths network routes.  & $10$  & $(5,5,5)$   & $(0.5,0.5,0.0)$  \\
		\algorithmBalancedExt{}{} &  EWSN optimisation algorithm with balanced objectives. & $25$ & $10$    & $(0.33,0.33,0.33)$  \\
		\algorithmEnergy{}{} & EWSN with $80\%$ bias for energy consumption minimisation  & $25$ & $10$   & $(0.80,0.10,0.10)$  \\
		\algorithmQuality{}{} & EWSN with $80\%$ bias for task quality maximisation. & $25$ & $10$   & $(0.10,0.80,0.10)$  \\
		\algorithmDistribution{}{} & EWSN with $80\%$ bias for energy distribution maximisation. & $25$ & $10$  & $(0.10,0.10,0.80)$  \\
		\hline
	\end{tabular}
	\captionsetup{labelfont=bf,singlelinecheck=on}
	\caption{Summary of configurations}
	\label{table:summary_of_configurations}
\end{table}

\begin{table}[ht]
	\centering
	\begin{tabular}{
			|p{0.22\textwidth}|p{0.12\textwidth}|p{0.12\textwidth}|p{0.12\textwidth}|p{0.12\textwidth}|p{0.12\textwidth}|
		}
\hline
		\textbf{Algorithm}
			& \textbf{CTV $\%$ of optimal}
			& \textbf{$\%$ of energy available} 
			& \textbf{Quality/energy available fraction} 
			& \textbf{Average task-path depth}
			& \textbf{Energy distribution} \\
\hline
		\algorithmBalanced{}{}
			& \resultsCTVBalancedEnd{}{} 
			& \resultsEnergyBalancedEnd{}{}
			& & & \\
		\algorithmQRouting{}{}
			& \resultsCTVQRoutingEnd{}{} 
			& \resultsEnergyQRoutingEnd{}{} 
			& & & \\
	%	\algorithmBalancedExt{}{}
		%	& \resultsCTVBalancedExtEnd{}{} 
	%		& \resultsEnergyBalancedExtEnd{}{}
	%		& & & \\
	\hline
	\algorithmEnergy{}{} 
		& & 
		&  n/a  
		& \resultsArcDepthEnergyEnd{}{} 
		& \resultsTaskDistEnergyEnd{}{} \\
	\algorithmQuality{}{} 
		& & 
		& \resultsQEQualityDiff{}{} 
		& \resultsArcDepthQualityEnd{}{} 
		& \resultsTaskDistQualityEnd{}{} \\
	\algorithmDistribution{}{}
		& & 
		& \resultsQEDistDiff{}{} 
		& \resultsArcDepthDistEnd{}{} 
		& \resultsTaskDistDistEnd{}{} \\
	\hline
\end{tabular}
\captionsetup{labelfont=bf,singlelinecheck=on,justification=raggedright}
\caption{Experimental results for after 500 episodes}
\label{table:results}
\end{table}

	\subsection{Analysis}

As seen in Figure \ref{5_ctv-optimal-ctv}, the \algorithmBalanced{}{} algorithm in the simple system optimises the system utility by $XXX\%$, from $XXX\%$ in the first episode through to $XXX\%$ by episode $500$.
\begin{figure}[ht]
	\centering
	\includegraphics[width=0.8\linewidth]{5_ctv-optimal-ctv}
	\captionsetup{labelfont=bf,singlelinecheck=on}
	\caption{System utility compared to the theoretical maximum for the \simulationSimple{}{} system. The algorithms work to increase this value, which impacts the task values, energy consumption, and distribution depending on the weighting given to those various components}
	\label{fig:5_ctv-optimal-ctv}
\end{figure}
In Figures \ref{fig:5_ctv-statistics-energy-available}, \ref{fig:5_ctv-quality}, we see the impact of the CTV optimisation on task values and available system energy. The average quality of the measurement tasks as compared to the optimal quality available in the sytem given the agents distance from the respective tasks' demand points. Over the systems' lifetime this is increased from $XXX\%$ to $XXX\%$. Similarly as the algorithm improves the routing for task allocation, the energy consumption of the system is reduced, the available energy going from  $XXX\%$ to $XXX\%$. These results demonstrate the the algorithm meets the described problems requirements of allocating tasks to increase their value, while optimising the energy usage as it does so.
 
\todo[inline]{Need a quality fdiagram here}
\begin{figure}[ht]
	\centering
	\includegraphics[width=0.8\linewidth]{5_ctv-statistics-energy-available}
	\captionsetup{labelfont=bf,singlelinecheck=on}
	\caption{Energy available in the \simulationSimple{}{} system as percentage of the maximum possible. Higher values show a more efficient use of agents to complete tasks, but may not give the best task values overall}
	\label{fig:5_ctv-statistics-energy-available}
\end{figure}
\begin{figure}[ht]
	\centering
	\includegraphics[width=0.8\linewidth]{5_ctv-arc-depth}
	\captionsetup{labelfont=bf,singlelinecheck=on}
	\caption{The average depth of arcs in the \simulationSimple{}{} system. Longer arcs allow sink agents to reach sensing agents that are closer to the task demand point, at the cost of greater energy usage}
	\label{fig:5_ctv-arc-depth}
\end{figure}
\begin{figure}[ht]
	\centering
	\includegraphics[width=0.8\linewidth]{5_ctv-task-distribution}
	\captionsetup{labelfont=bf,singlelinecheck=on}
	\caption{The task distribution in the system. Higher values spread energy usage across the system better, increasing the systems' lifetime}
	\label{fig:5_ctv-task-distribution}
\end{figure}

We see the ability of the algorithm to weight its optimisation between task quality, energy availability, and distribution given varying values for the $\alpha, \beta, \gamma$ parameters in Figures \ref{fig:5.19_ctv-quality-energy} and  \ref{XXX}. Figure \ref{fig:5.19_ctv-quality-energy} shows the task quality/energy availability of the system. As values range higher, this means that quality is being preferentially optimised for over energy availability. As expected, the \algorithmQuality{}{} algorithm, with its high $\gamma$ value increases this fraction from $XXX\%$ to $XXX\%$ over the system lifetme, whereas the \algorithmEnergy{}{} algorithm moves from only $XXX\%$ to $XXX\%$, a comparatively small increase. As seen, system utility, the sum of CTVs, and energy availability are optimised in both these systems, however it is the relative increase in the task quality and energy availability that is different, showing that the CTV used within the algorithm optimises across these multiple objectives flexibly. 



\begin{figure}
	\centering
	\includegraphics[width=0.7\linewidth]{5.19a_ctv-quality-energy-baseline-comparison}
	\caption{}
	\label{fig:5}
\end{figure}
\begin{figure}
	\centering
	\includegraphics[width=0.7\linewidth]{5.19_ctv-statistics-energy-available-comparison}
	\caption{}
	\label{fig:5}
\end{figure}
\begin{figure}
	\centering
	\includegraphics[width=0.7\linewidth]{5.19_ctv-quality-energy-baseline-comparison}
	\caption{}
	\label{fig:5}
\end{figure}
\begin{figure}
	\centering
	\includegraphics[width=0.7\linewidth]{5.19_ctv-task-distribution-comparison}
	\caption{}
	\label{fig:5}
\end{figure}
\begin{figure}
	\centering
	\includegraphics[width=0.7\linewidth]{5.19_ctv-arc-depth-comparison}
	\caption{}
	\label{fig:5}
\end{figure}

\begin{figure}
	\centering
	\includegraphics[width=0.7\linewidth]{5.19c_ctv-task-distribution-baseline-comparison}
	\caption{}
	\label{fig:5}
\end{figure}

\begin{figure}
	\centering
	\includegraphics[width=1.0\linewidth]{result-types}
	\caption{Three sample arc patterns in the \simulationExtended{}{} system. In the \algorithmEnergy{}{}-optimised configuration (a), the sensing agents are near the sync agent. Energy use is minimised but the measurements task-values are low. In the \algorithmQuality{}{}-optimised configuration (b) the sensing agents are close to the demand points, maximising the task-values, however there is an increase in energy usage as they are more distant from the sink, with increased arc-length. In the \algorithmDistribution{}{}-optimised configuration (c), sensing agents are a mix of close and further away from the demand points, with an the agents participating in the arcs changing with different measurements}
	\label{fig:result-types}
\end{figure}


	
With a CTV balance optimised solely to minimise energy consumption, i.e. $(\alpha=1,\beta=0,\gamma=0)$, the \acronymWSNOptimisation{}{} algorithm behaves similarly to energy-aware algorithms applied to WSN. In this configuration comparisons can be made with algorithms such as PEGASIS \citep{Lindsey2002}, or more closely to Q-routing algorithms like Q-probabilistic routing \citep{Arroyo-Valles2007}. However, the task allocation and resource optimisation component of the \acronymWSNOptimisation{}{} algorithm is not accounted for in these implementations so provides only a comparison for the energy and route adaptation properties. 

	\section{Conclusions and future work}
\label{section:conclusions}

This work detailed and evaluated the \acronymWSNOptimisation{}{} algorithm and its application to wireless sensor network optimisation in dynamic and challenging environments. This is an extension of the previously described work on \acronymATARIA{}{} and \acronymMGRAO{}{} algorithms \citep{creech2021dynamic,creech2021resource} to hierarchical multi-agent systems. The algorithm was shown to optimise the and the quality of task completion and energy available in these systems, and to increase system lifetime through task and energy distribution. The algorithm was evaluated on a model WSN system based on a realistic situation where agents would be randomly distributed across a geographical area, where maintenance and management would be challenging due to harsh or dangerous conditions.  Our evaluation showed that the \acronymWSNOptimisation{}{} algorithm optimised the task quality, energy available, and distribution in the system as describe in Section \ref{section:experimental}, and that these components could be varied in their priorities through altering the $\alpha$, $\beta$ and $\gamma$ values of the CTV function (Section \ref{section:problem:optimising_resource_usage}, Eq. \ref{eq:taq}). This allowed the algorithm to balance across these different properties in the given systems and optimise for these multiple objectives in different ratios. 

Future work would look at implementing the algorithm in a larger scale network through simulation, as well as in the real-world, testing how the algorithm performs in a complex environment. There are a number of applications in WSN in which the agents involved are mobile. As the \acronymWSNOptimisation{}{} algorithm is designed to work in dynamic environments, where optimisation targets are non-stationary, we expect that it will also be useful in these types of system. Evaluation could be extended to simulations with mobile agents, and tested in real-world vehicle-to-vehicle communications (V2X) systems \citep{Gupta2017, Tong2019}. We also expect that testing practical deployment this work in the case of oceanographic monitoring would be a productive next step \citep{Albaladejo2010a}. The combination of harsh environmental conditions, difficulty of providing maintenance for remote agents, and mobility at slow speeds, should provide ideal conditions for successful use of \acronymWSNOptimisation{}{}.


  
	\printcredits

%% Loading bibliography style file
%\bibliographystyle{model1-num-names}
\bibliographystyle{cas-model2-names}

% Loading bibliography database
\bibliography{cas-refs}

	%\vskip3pt

\bio{figs/niall}
Author biography without author photo.
Author biography. Author biography. Author biography.
Author biography. Author biography. Author biography.
Author biography. Author biography. Author biography.
Author biography. Author biography. Author biography.
Author biography. Author biography. Author biography.
Author biography. Author biography. Author biography.
Author biography. Author biography. Author biography.
Author biography. Author biography. Author biography.
Author biography. Author biography. Author biography.
\endbio

\bio{figs/natalia}
Author biography with author photo.
Author biography. Author biography. Author biography.
Author biography. Author biography. Author biography.
Author biography. Author biography. Author biography.
Author biography. Author biography. Author biography.
Author biography. Author biography. Author biography.
Author biography. Author biography. Author biography.
Author biography. Author biography. Author biography.
Author biography. Author biography. Author biography.
Author biography. Author biography. Author biography.
\endbio

\bio{figs/simon}
Author biography with author photo.
Author biography. Author biography. Author biography.
Author biography. Author biography. Author biography.
Author biography. Author biography. Author biography.
Author biography. Author biography. Author biography.
\endbio
	\FloatBarrier
	\newpage
	\begin{appendix}
\label{appendix:algorithms}
%	\subsection{The \acronymATARIA{}{} algorithm \citep{creech2021dynamic}}
\label{appendix:algorithm-ataria}

\newcommand{\acronymNeighbourhood}[2]{\texttt{neighbourhood}}
\newcommand{\acronymActionSamples}[2]{\texttt{action-samples}}
\newcommand{\acronymQValues}[2]{\texttt{q-values}}
\newcommand{\acronymKnowledge}[2]{\texttt{knowledge}}
\newcommand{\varTargetAgent}[2]{\varAgent{}{*}}

\begin{algorithm}[ht]
	\DontPrintSemicolon
	\footnotesize
	
	\caption{\textbf{The simplified \acronymATARIA{}{} algorithm}}
	\label{alg:ataria}
	{
		\KwIn{ $\varAgent{}{}$ , The agent allocated the atomic task}
		\KwIn{ $\varAtomicTask{}{}$ , An atomic task}
		\KwIn{ $\acronymNeighbourhood{}{}$ , The \acronymNeighbourhood{}{} of agent $\varAgent{}{}$}
		\KwIn{ $\acronymKnowledge{}{}$ , The \acronymKnowledge{}{} of agent $\varAgent{}{}$}
		\KwIn{ $\acronymQValues{}{}$ , The \acronymQValues{}{} of agent $\varAgent{}{}$}
		\KwIn{ $\acronymActionSamples{}{}$ , The \acronymActionSamples{}{} of agent $\varAgent{}{}$}
		\nonl
		\;
		\KwResult{\acronymNeighbourhood{}{} , The updated neighbourhood values for $\varAgent{}{}$}
		\KwResult{\acronymKnowledge{}{} , The updated knowledge values for $\varAgent{}{}$}
		\KwResult{\acronymQValues{}{} , The updated q-value mappings values for $\varAgent{}{}$}
		\KwResult{\acronymActionSamples{}{} , The updated action-samples values for $\varAgent{}{}$}
		\nonl
		\;
		
		Select an action $\varAction{}{}$, with an optional target agent $\varTargetAgent{}{}$, using RT-ARP \;
		\uIf{action is $\functionExec{}{}$ }{
			$\varAgent{}{}$ executes the atomic task \;
		}
		\uElseIf{action is $\functionAlloc{\varAtomicTask{}{}}{\varTargetAgent{}{}}$}{
			allocate the atomic task to agent $\varTargetAgent{}{}$ \;
		}
		\uElseIf{action is $\functionInfo{\varTargetAgent{}{}}{}$ }{
			add an agent learned about from $\varTargetAgent{}{}$ to $\varAgent{}{}$'s knowledge\;
			prune agent $\varAgent{}{}$'s knowledge  based on its resource limits \;			}
		\uElseIf{action is $\functionLink{\varTargetAgent{}{}}{}$}{
			add agent $\varTargetAgent{}{}$ to $\varAgent{}{}$'s neighbourhood \;
			prune agent $\varAgent{}{}$'s neighbourhood  based on its resource limits \;
		}
		update the \acronymQValues{}{} for actions \;
		update the TQSM \;
		update the \acronymActionSamples{}{} \;
		\Return{$\acronymNeighbourhood{}{}$, $\acronymKnowledge{}{}$, $\acronymQValues{}{}$, $\acronymActionSamples{}{}$}
	}
\end{algorithm}

			\subsection{The \acronymATARIA{}{} algorithm \citep{creech2021dynamic}}
\label{appendix:algorithm-ataria}

\begin{algorithm}[ht]
	\DontPrintSemicolon
	\footnotesize
	
	\caption{\textbf{The \acronymATARIA{}{} algorithm}}
	\label{alg:ataria}
	{
		\KwIn{ $XXX$ , XXX}

		\KwResult{$XXX$ , XXX}
		\nonl
		\;
		\tcp{XXX}
		$XXX$ \;
		\Return{$XXX$}
	}
\end{algorithm}

%	\input{algorithm-mgrao-simple}

	\FloatBarrier
	\subsection{The \acronymMGRAO{}{} algorithm \citep{creech2021resource}}
\label{appendix:algorithm-mgrao}

\begin{algorithm}[ht]
	\DontPrintSemicolon
	\footnotesize
	\caption{\textbf{The \acronymMGRAO{}{} algorithm}}
	\label{alg:mgrao}
	{
		\KwIn{ $XXX$ , XXX}
		
		\KwResult{$XXX$ , XXX}
		\nonl
		\;
		\tcp{XXX}
		$XXX$ \;
		\Return{$XXX$}
	}
\end{algorithm}
	\end{appendix}
				
\end{document}
			
