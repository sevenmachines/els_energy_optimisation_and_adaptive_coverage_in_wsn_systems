\begin{abstract}
Wireless sensor networks (WSN) are large, distributed, agent-based systems found in a wide range of industries from vehicle-to-vehicle communications to large-scale environmental monitoring. They need to manage energy usage to maintain availability, distribute tasks effectively, and handle disruption to network changes and agent loss. Decentralised algorithms are commonly used to meet these challenges, with hierarchical cluster formation or reinforcement learning techniques. There are challenges however in getting these algorithms to perform well where there are multiple objectives, agents are mobile, or connectivity varies over the systems' lifetime.

In this work we propose the \acronymWSNOptimisationExtended{}{}
algorithm to optimise WSN systems based on maximising energy availability, distribution, and task quality, while maintaining task coverage in a dynamic network. This integrates two previously defined algorithms and extends them to use a hierarchical task allocation strategy, and to balance optimisation amongst multiple system objectives. We evaluate the algorithms' performance in environmental monitoring-based simulated systems where there are a number of measurement tasks to be completed within the system.

Overall, the \acronymWSNOptimisation{}{} algorithm increased the utility of the realistic systems considered, reducing energy consumption while improving the quality of completed tasks. There was a \resultsSimpleCTVBalancedDiff{}{} system utility improvement in the stable $25$ node system over $1000$ episodes, with energy availability increased by \resultsSimpleEnergyBalancedDiff{}{}. Evaluation of changing algorithm parameters to balance between energy availability, distribution, and task quality showed that these individual components could be prioritised in different ratios depending on the requirements of the optimisation required in the system.  
\end{abstract}

%\begin{graphicalabstract}
%	\includegraphics{figs/grabs.pdf}
%\end{graphicalabstract}