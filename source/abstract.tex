\begin{abstract}
Wireless sensor networks (WSN) are applicable across a wide range of industries from vehicle-to-vehicle communications to large-scale environmental monitoring. They need to manage energy usage to maintain availbility, distribute tasks effectively, and handle distruption to network changes and agent loss. Decentralised algorithms are commonly used to meet these challenges, with hierarchical cluster formation or reinforcement learning techniques. There are challenges however in getting these algorithms to perform well in large distributed systems where there are multiple objectives, dynamic agents, or connectivity change. In this work we propose the novel \acronymWSNOptimisationExtended{}{}
algorithm to optimise WSN systems based on the multiple objectives of maximising energy availability, distribution, and task quality, while maintaining task coverage in a dynamic network. This integrates and extends upon the previously defined \acronymATARIAExtended{}{} and \acronymMGRAOExtended{}{} algorithms, adding hierarchical task allocation with multi-objective composite task value calculation. We evaluate the algorithms' performance in environmental monitoring-based simulated systems where there are a number of measurement tasks to be completed within the system. The \acronymWSNOptimisation{}{} algorithm showed a \resultsCTVBalancedDiff{}{} system utility improvement in the simple $10$ node system, and \resultsCTVBalancedExtDiff{}{} in a more complex $25$ node system over $500$ episodes. Energy availability was increased by \resultsEnergyBalancedDiff{}{} and \resultsEnergyBalancedExtDiff{}{} respectively. Evaluation of changing algorithm parameters to balance between energy availability, distribution, and task quality showed that these individual components could be prioritised in different ratios depending on the requirements of the optimisation required in the system.  
\end{abstract}

%\begin{graphicalabstract}
%	\includegraphics{figs/grabs.pdf}
%\end{graphicalabstract}