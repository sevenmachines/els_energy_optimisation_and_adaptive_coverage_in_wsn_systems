\paragraph{Composite tasks}
\label{section:problem:composite_tasks}
%%%%%%%%%%%%%%%%%%%%%%%%%%%%%%
\newcommand{\formalTaskResponsibilities}[2]{
	\functionFormal{resp}
	{\setCompositeTaskType{}{} \times \powerSetAgent{}{}{}{}}
	({\setCompositeTaskType{}{} \times \setAgent{}{}})
}
\newcommand{\functionTaskResponsibilities}[2]{
	\functionSignature{resp}
	{\varCompositeTaskType{}{}, \setAgent{}{}}
}

\newcommand{\formalCompositeTaskType}[2]{
	\functionFormal{type_{\varCompositeTask{}{}}}
	{\setCompositeTask{}{}}
	{\setCompositeTaskType{}{}}
}
\newcommand{\functionCompositeTaskType}[2]{
	\functionSignature{type_{\varCompositeTask{}{}}}
	{\varCompositeTask{#1}{#2}}
}
%%%%%%%%%%%%%%%%%%%%%%%%%%%%%%

Some agents will have \textit{composite tasks}, $\varCompositeTask{}{}$, allocated to them from an external source, throughout the systems' lifetime. We term these agents \textit{sinks}. These composite tasks are typed by the function $\formalCompositeTaskType{}{}$. The \textit{task-responsibilities} function gives a mapping, fixed on system initialisation, of each composite task type  to the group of agents that will receive those tasks and ensure their completion; $\formalTaskResponsibilities{}{}$.

%%%%%%%%%%%%%%%%%%%%%%%%%%%%%
\newcommand{\formalTaskDemandPoint}[2]{
	\functionFormal{demand}
	{\setAtomicTask  {}{}}{\tupleLocation{}{}}
}
\newcommand{\functionTaskDemandPoint}[2]{\functionSignature{demand}{\varAtomicTask{#1}{#2}}}

\newcommand{\functionRequiredResourcesSymbol}[2]{
	\functionSymbol{demandres}
}
\newcommand{\formalRequiredResources}[2]{
	\functionFormal{\functionRequiredResourcesSymbol{}{}}
	{\setAtomicTaskType{#1}{} \times \setResourceType{#2}{}}
	{\setRealNumbersNonNegative{}{}}
}
\newcommand{\functionRequiredResources}[2]{
	\functionSignature{\functionRequiredResourcesSymbol{}{}}{\varAtomicTaskType{#1}{}, \varResourceType{#2}{}}
}
\newcommand{\functionRequiredResourcesInstance}[2]{
	\functionSignature{\functionRequiredResourcesSymbol{}{}}
	{\functionAtomicTaskMapping{\varAtomicTask{#1}{}}{}, \varResourceType{#2}{}}
}

\newcommand{\formalAtomicTaskType}[2]{
	\functionFormal{type_{\varAtomicTask{}{}}}
	{\setAtomicTask{}{}}
	{\setAtomicTaskType{}{}}
}
\newcommand{\functionAtomicTaskType}[2]{
	\functionSignatyre{type_{\varAtomicTask{}{}}}
	{\varAtomicTask{#1}{#2}}
}
%%%%%%%%%%%%%%%%%%%%%%%%%%%%%%%%%%

\paragraph{Atomic tasks}
\label{section:problem:atomic_tasks}
Sinks will decompose composite tasks into their constituent \textit{atomic tasks}, $\setAtomicTask{}{}$, of type $\formalAtomicTaskType{}{}$. Each atomic task can either be; executed by the sink, allocated to other agents to complete,  or relayed to further agents. Each atomic task targets a sensor measurement at a specific location, the tasks' \textit{demand point}, $\formalTaskDemandPoint{}{}$. We assume that the further away the agent is from this point, the less value the task has to the system; e.g. the measurement differs significantly from the actual value at the targeted point. Atomic tasks also have a \textit{resources demand}, a mapping from a task type to the resources it would require for tasks of that type to be completed to their highest quality; $\formalRequiredResources{}{}$. 

\example{Distance to tasks demand points and task resource demands}{
An agent $\varAgent{1}{}$ and an agent $\varAgent{2}{}$ are allocated tasks to take salinity measurements $\varAtomicTask{1}{}$, and $\varAtomicTask{2}{}$ respectively in a ocean-monitoring system, where the demand points of the two tasks are close, $\functionTaskDemandPoint{1}{} \simeq \functionTaskDemandPoint{2}{}$. However, $\varAgent{1}{}$  is much further away from this point than $\varAgent{2}{}$ so that, $\funcSize{\functionTaskDemandPoint{2}{}  - \functionDeployment{2}{}} <<< \funcSize{\functionTaskDemandPoint{1}{}  - \functionDeployment{1}{}}$. Since $\varAtomicTask{2}{}$ is much closer to the demand point targeted by $\varAtomicTask{2}{}$ than in $\varAtomicTask{1}{}$'s case, we assume its salinity measurement is much more likely to be close to the actual value at that location as well\footnote{In this example there may be salinity barriers and effects of water currents that invalidate this assumption, but we assume that it broadly holds.}. If $\varAgent{2}{}$ has dedicated $50$ units of its energy resource $\varResource{e}{}$ to completing tasks of type  $\varResourceType{e}{}$, and $\functionRequiredResourcesInstance{2}{e} = 100$, then it can make the sensor reading more accurate bv increasing its energy allocation to the salinity sensor and improve its sampling rate.
}

