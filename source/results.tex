
The \simulationExtended{}{} system had CTV component weightings where each of the relevant properties were given an $80\%$ dominance over the value of CTV value. The sink node was given $10$ measurements to allocate, with no repetition. It was also placed at a significantly large distance from the demand points associated with the tasks. $25$ nodes were distributed randomly in the system. This system examined the impact of the algorithm optimising the allocation of tasks towards the goals stated in Section \ref{section:optimisation_problem}. Atomic task quality could be maximised, but at the cost of longer task-paths and therefore energy usage, or energy consumption could be minimised, with correspondingly lower task qualities. Figure \ref{fig:route_types} illustrates these two route types for task completion.

Labels, descriptions, and configurations for each algorithm are shown in Table \ref{table:summary_of_configurations}. Results for the \algorithmBalanced{}{} algorithm in the \simulationSimple{}{} system, and the \algorithmEnergy{}{}, \algorithmQuality{}{}, \algorithmDistribution{}{} algorithms in the \simulationExtended{}{} system are shown in Table \ref{table:results}. System utility percentages show the summed values of composite tasks per episode, as shown in Equation \ref{eq:system_utility}, compared to the theoretical maximum utility in the system \footnote{Note that the theoretical maximum is not necessarily attainable in all systems, dependent on their randomised node configurations.}, with the percentage optimisations from the first episode to last in Figure \ref{fig:5_ctv-optimal-ctv-gain}. Energy available is presented as a percentage of that of a system containing nodes with full battery charge in Figure \ref{fig:ctv-statistics-energy-available}. We compare the different biases for optimisation across the CTV components using quality-energy balance, task distribution, and task-path depth results. The quality-energy data shown in Figure \ref{fig:ctv-quality-energy-baseline-comparison} uses the \algorithmEnergy{}{} algorithm as a baseline, with the percentage increase or decrease in the average task quality over energy availability components of the CTV equation in Equation \ref{eq:ctv}. The task-distribution in Figure \ref{fig:ctv-task-distribution-comparison} shows the variation in the agents that are completing the tasks, i.e. $\funcSize{set(\setAgents{}{})}{}/\funcSize{\setAgents{}{}}{}$, with higher values representing more tasks being completed by distinct agents, and lower values meaning more agents are completing multiple tasks. Task-path depth data in Figure \ref{fig:ctv-arc-depth-comparison} captures how many agents re-allocated each task before it was completed.