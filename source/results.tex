\subsection{System configurations and results}
Labels, descriptions, and configurations for each algorithm are shown in Table \ref{table:summary_of_configurations}. Table \ref{table:results_main} shows the average and cumulative system utility, and energy available values, for both algorithms in the \simulationSimple{}{} and  \simulationNodeFailure{}{} systems. For the \simulationNodeFailure{}{} system, it also includes the percentage drop in these values during the period of node failure. The impact of these failures on the percentage coverage (Equation \ref{eq:coverage}) for tasks is shown in Table \ref{table:results_coverage}. The results for the \algorithmEnergy{}{}, \algorithmQuality{}{}, \algorithmDistribution{}{} configurations of the \acronymWSNOptimisation{}{} algorithm in the \simulationExtended{}{} system are shown in Table \ref{table:results_balance}. 

The average utilities  (Equation \ref{eq:system_utility}) are shown in Figures \ref{fig:simple_ctv} and \ref{fig:node_failure_ctv} for the \simulationSimple{}{} and \simulationNodeFailure{}{} systems with the corresponding cumulative sums in Figures \ref{fig:simple_cumulative_ctv} and \ref{fig:node_failure_cumulative_ctv}. For the \simulationNodeFailure{}{} system we graph the percentage of failed agents in the system in Figure \ref{fig:node_failure_failed_agents} and its effect on percentage coverage (	Equation \ref{eq:coverage}) in Figure \ref{fig:node_failure_coverage}.  For the \simulationExtended{}{} system we compare the different biases for optimisation across the CTV components using quality-energy balance, task distribution, and task-path depth results. The quality-energy data shown in Figure \ref{fig:extended_quality_energy} uses the \algorithmEnergy{}{} algorithm as a baseline, with the percentage increase or decrease in the average task quality over energy availability components of the CTV equation in Equation \ref{eq:ctv}. The task-distribution in Figure \ref{fig:extended_task_distribution} shows the variation in the agents that are completing the tasks\footnote{ Simply calculated as the fraction of unique sensing agents $\setAgents{}{}$ completing all the atomic tasks during an episode, $\funcSize{set(\setAgents{}{})}{}/\funcSize{\setAgents{}{}}{}$}, with higher values representing more tasks being completed by distinct agents, and lower values meaning more agents are completing multiple tasks. Task-path depth data in Figure \ref{fig:extended_arc_depth} captures how many agents recursively allocated each task before it was completed.