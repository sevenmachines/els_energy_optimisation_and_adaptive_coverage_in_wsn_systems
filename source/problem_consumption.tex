\subsection{Energy consumption and availability}
The \textit{atomic task energy consumption} is the energy used by the nodes in the system in executing an atomic task. This is composed of,
 \begin{itemize}
 	\item \textit{Transmission energy}, $\varTransmissionEnergy{}{}$, the energy used by nodes to  send a message, such as when allocating a task to another node or returning a task result. 
 	\item \textit{Receiver energy}, $\varReceiverEnergy{}{}$, the energy used by a node to receive a message. 
 	 \item \textit{Sensor energy}, $\varSensorEnergy{}{}$,the energy used by a node to execute an atomic measurement task.
 \end{itemize}
For simplification, we assume that these values are constants in a given system. In which case, for a given task-path, the combined energy usage will be,
\begin{align}
\functionAtomicTaskEnergyConsumptionSignature{}{} 
&= 
\underbrace{(\varTransmissionEnergy{}{} + \varReceiverEnergy{}{})}_{\text{sink node}}
+ \underbrace{
	2 (\varTransmissionEnergy{}{} + \varReceiverEnergy{}{})
 	(\funcSize{(\functionTaskArc{}{}}{} -2)
}_{\text{active nodes}}
+ \underbrace{
	 (
	 	\varTransmissionEnergy{}{}
	 	+ \varReceiverEnergy{}{}
	 	+ \varSensorEnergy{}{}
	 )
 }_{\text{sensor node}}  
\end{align}
Although nodes would still use energy when in an idle power saving mode, or sleep mode, we disregard these for this formulation as we look to optimise the active node power usage only. There are other algorithms that can be used to optimise the cycling of power cell usage \citep{Escolar2014}.

%%%%%%%%%%%%%%%%%%%%%%%%%%%%%%%%%%%%%%%%%
\newcommand{\formalAgentEnergyAvailable}[2]{
	\functionFormal{fae}
	{\setAgents{}{}}
	{\setRealNumbersUnit{}{}}
}
\newcommand{\functionAgentEnergyAvailable}[2]{
	\functionSignature{fae_{\varTime{}{}}}
	{\varAgent{}{}}
}
\newcommand{\functionEnergyVariability}[2]{
	\ifx \\#1\\
	\functionSignature{rev_{\varTime{}{}}}
	{\setAgents{}{}}
	\else
	\functionSignature{rev_{\varTime{}{}}}{#1}
	\fi
}

\newcommand{\functionEnergyAvailable}[2]{
	\ifx \\#1\\
	\functionSignature{ea_{\varTime{}{}}}{\setAgents{}{}}
	\else
	\functionSignature{ea_{\varTime{}{}}}{#1}
	\fi
}
The \textit{fractional agent energy} maps an agent to its available energy, as a fraction of the batteries' maximum capacity, $\formalAgentEnergyAvailable{}{}$. We can then specify the \textit{fractional energy availability}, as the sum-average of the agent energy of all agents in a set $\setAgents{}{}$.
\begin{equation}
	\functionEnergyAvailable{}{} 
	= \dfrac{\sum_{\forall \varAgent{}{} \in \setAgents{}{}} \functionAgentEnergyAvailable{\varAgent{i}{}}{}}
	{\funcSize{\setAgents{}{}}}
\end{equation}
To optimise for distribution of energy usage we minimise the variance\footnote{Using the standard definition of variability of a discrete set $X$, $\sigma^2(X) = \frac{\sum (x_i - \bar{x})^2}{\funcSize{X}-1}$} of the fractional agent energy values. As we look to optimise by maximising across multiple goals, and the values of $\functionAgentEnergyAvailable{}{}$ are bounded by $[0, 1]$, we can rephrase this as maximising the distance between the variance and the maximum possible, $1/4$. So we use the \textit{relative energy variability} as our distribution measurement,
\begin{equation}     	
	\functionEnergyVariability{}{} 
	= \frac{1}{4} - \sigma^2(
	\lbrace \functionAgentEnergyAvailable{}{}
	\rbrace_{\forall \varAgent{}{} \in \setAgents{}{}}
	)
\end{equation}


