\paragraph{Energy consumption}
\label{section:energy_consumption}

\reviewquestionopen{Isn't energy just one resource that an agent can have and use for tasks? You gave it as the example of a resource in Section 3.2. If so, why is there a section (Section 3.5) on just one particular resource type? Are you implying that this type of resource is always an element of R for every system (and if so, where do you say this)? You define energy as being used for things other than enacting an atomic task - transmission and receiving - so are other resources also used up by these actions? In Section 3.2, you define ra as giving the resources allocated to 'completing' task types, but does 'completing' include receiving, transmitting and sensing, or only sensing? If the latter, then where is the allocation of resources for receiving and transmitting? If it is correct that you want to highlight a particular kind of resource, then it would be good to give that resource a name, e.g. $energy \in R$.
}

Given a task path, $\lbrace \functionSinkRoleAtomic{}{}, \functionRelayRole{i}{i+1}, \functionDetectorRole{}{} \rbrace_{i=2}^{n-1}$, 
we can quantify the \textit{atomic task energy consumption}, $\formalAtomicTaskEnergyConsumption{}{}$, as the energy used by a task paths' nodes in executing the atomic task \footnote{See definition of key requirements, \ref{requirement:energy}, in Section \ref{section:background}}. 
\begin{equation}
\begin{aligned}
	\functionAtomicTaskEnergyConsumptionSignature{}{} 
	&= (\functionSignature{energy_{tr}}
		{\functionSinkRoleAtomic{}{},\varAgent{2}{}} + \varReceiverEnergy{}{}) & [\textit{sink energy}]\\
	&+ 2 \sum_{i=2}^{n-1} (\functionTransmissionEnergy{i}{i+1} + \varReceiverEnergy{}{})
	 & [\textit{relay energy}]\\
	&+ (
		\functionTransmissionEnergy{n}{n-1}
		+ \varReceiverEnergy{}{}
		+ \varSensorEnergy{}{}
		)& [\textit{detector energy}]
\end{aligned}
\end{equation}


\newcommand{\functionEnergyVariability}[2]{
	\ifx \\#1\\
	\functionSignature{rev_{\varTime{}{}}}
	{\setAgents{}{}}
	\else
	\functionSignature{rev_{\varTime{}{}}}{#1}
	\fi
}

\newcommand{\functionEnergyAvailable}[2]{
	\ifx \\#1\\
	\functionSignature{ea_{\varTime{}{}}}{\setAgents{}{}}
	\else
	\functionSignature{ea_{\varTime{}{}}}{#1}
	\fi
}
We can then specify the \textit{fractional energy availability}, as the sum-average of the agent energy of all agents in a set $\setAgents{}{}$.
\begin{equation}
	\functionEnergyAvailable{}{} 
	= \dfrac{\sum_{\forall \varAgent{}{} \in \setAgents{}{}} \functionFractionalAgentEnergy{\varAgent{i}{}}{}}
	{\funcSize{\setAgents{}{}}}
\end{equation}
To optimise for distribution of energy usage we minimise the variance\footnote{Using the standard definition of variability of a discrete set $X$, $\sigma^2(X) = \frac{\sum (x_i - \bar{x})^2}{\funcSize{X}-1}$} of the fractional agent energy values. As we look to optimise by maximising across multiple goals, and the values of $\functionFractionalAgentEnergy{}{}$ are bounded by $[0, 1]$, we can rephrase this as maximising the distance between the variance and the maximum possible, $1/4$. So we use the \textit{relative energy variability} as our distribution measurement,
\begin{equation}     	
	\functionEnergyVariability{}{} 
	= \frac{1}{4} - \sigma^2(
	\lbrace \functionFractionalAgentEnergy{}{}
	\rbrace_{\forall \varAgent{}{} \in \setAgents{}{}}
	)
\end{equation}


