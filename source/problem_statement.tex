\subsection{The multi-objective optimisation of tasks in a WSN system}
\label{section:problem:statement}

A system has a set of agents $\setAgent{}{}$ and a set of composite tasks $\setCompositeTask{}{}$. A subset of these agents $\lbrace \functionTaskResponsibilitiesInstance{}{} \rbrace_{\forall \varCompositeTask{}{} \in \setCompositeTask{}{}}$ will receive these tasks over a period of time, decompose them into atomic tasks $\setAtomicTask{}{'}$, and ensure their completion. The goal of a system is to allocate these tasks as to maximise $\functionSystemUtility{}{}$.

\example{Goals in an ocean-based WSN}{
	$100$ agents, each with a sensor to measure salinity, are deployed into an ocean bay $1\ km^2$ in area. Each agent is deployed attached to a buoy to maintain their general positions. One of the agents receives a composite task request every hour from a base-station on the coast consisting of atomic tasks to measure the salinity in $50$ separate locations in the bay. The goal of the system is to;
	\begin{itemize}
		\item form and maintain an ad-hoc communication network, adaptable to changes in the environment.
		\item return the most accurate, high quality results every hour to the base station.
		\item minimise the energy used by agents so that their batteries can charge enough between repeated requests, in order to maintain coverage.
		\item distribute the tasks so that the wear and subsequent failure of agents is reduced, meaning the systems' lifetime exceeds the operational design of $1$ year. 
	\end{itemize}
}