\paragraph{Task quality}
\label{section:task_quality}
%%%%%%%%%%%%%%%%%
\newcommand{\functionAtomicTaskQualitySymbol}[2]{\functionSymbol{atq_{#1}^{#2}}}
\newcommand{\functionAtomicTaskQualitySignature}[2]{
	\ifx&#1&
	\functionSignature{\functionAtomicTaskQualitySymbol{}{}} {\varAtomicTask{}{}, \varAgent{}{}}
	\else
	\functionSignature{\functionAtomicTaskQualitySymbol{}{}}{#1, #2}
	\fi
}
\newcommand{\functionAtomicTaskQualitySensor}[2]{
	\functionSignature{\functionAtomicTaskQualitySymbol{}{}} {\varAtomicTask{}{}, \functionDetectorRole{}{}}
}
%%%%%%%%%%%%%%%%%%%%%%%%%%%%%%

An agent will complete an atomic task with a certain \textit{atomic task quality}, how well it has performed the task. This value is dependant on two components;
\begin{enumerate}
	\item \textit{execution range}, the distance between the agent and the demand point of the task. As the maximum distance between any agent and demand point in our normalised coordinate system\footnote{As discussed in Section \ref{section:task_and_resources:distribution}} is $\sqrt{2}$, we can scale $\funcSize{ \functionTaskDemandPoint{}{} - \functionSignature{deploy}{\varAgent{}{}}}$ to the range $\lbrack 0, 1 \rbrack$, ensuring the quality also stays in this range.
	\item \textit{resource fraction}, the fraction of the atomic tasks' resource demand that the agent has actually dedicated to tasks of that type\footnote{To simplify the task quality function, we assume reducing resource allocation for a task has an equal and linear impact across all tasks and resources. However, a more complex resource fraction component could be used given a specific systems' set of resources and tasks.}.
\end{enumerate}

\begin{equation}
	\label{eq:atomic_task_quality}
	\functionAtomicTaskQualitySignature{}{} = 
\bigg (
\underbrace{
	1 - \frac{\funcSize{
			\functionTaskDemandPoint{}{} - \functionDeployment{}{} 
		}{}}{{\sqrt{2}}}
}_{\text{distance decay}}
\bigg )
\bigg (
\underbrace{
	\frac{1}{\funcSize{\setResource{}{}}}\sum_{\varResource{}{} \in \setResource{}{}} 
	\frac
	{\functionTaskResourceAllocationInstance{}{}}
	{\functionRequiredResourcesInstance{}{}}
}_{\text{resource fraction}}
\bigg )
\end{equation}

\paragraph{Component task value}
%%%%%%%%%%%%%%%%%%%%%%%
\newcommand{\functionComponentTaskValue}[2]{
	\functionSignature{ctv}{\varAtomicTask{}{}}
}
\newcommand{\formalComponentTaskValue}[2]{
	\functionFormal{ctv}{\setAtomicTask{}{}}{\setRealNumbersNonNegative{}{}}
}
%%%%%%%%%%%%%%%%%%%%%%%%%%%%%%

When a composite task is completed, the value of an atomic task to its final outcome can be measured. This is not the same as task quality, as it may be dependent on other factors. For example,  two atomic tasks $\varAtomicTask{1}{}, \varAtomicTask{2}{}$, are part of the sample composite task. If $\functionTaskDemandPoint{1}{} \simeq \functionTaskDemandPoint{2}{}$ then each tasks results may be less valuable to the composite task than if the demand points were more separated, as their results have significant information overlap. Therefore, we define the \textit{component task value} as the mapping, $\formalComponentTaskValue{}{}$,  of each atomic task of a composite task to the fractional value of each of them to the composite tasks' completion, such that $\sum\limits_{\forall \varAtomicTask{}{} \in \varCompositeTask{}{}} \functionComponentTaskValue{}{} = 1$.

