
In this section we will formally set out the problem we look to solve. We define the base concepts of our WSN system in Sections \ref{section:terminology} and \ref{section:tasks}. 

In completing tasks, agents will be able to choose from a number of different actions, and assume different roles in executing each individual task. We look at roles and actions in Sections \ref{section:roles} and \ref{section:actions}. In Section \ref{section:energy_consumption} we describe how energy is consumed by the system. Next, in Section \ref{section:task_quality}, we define to what quality a measurement is completed, and how this, energy availability, and energy distribution in the system, effects the quality of overall composite task completion. Finally, we detail task coverage in Section \ref{section:coverage}, and how this relates to the resilience of the network, and the effective lifetime of the system, giving us the problem definition in Section \ref{section:optimisation_problem}.

