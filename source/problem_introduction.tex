\todo[inline]{REORDER SECTION 3 to match up with Section2?}
In this section we will formally set out the problem we look to solve. We define our WSN system in terms of,
\begin{itemize}
	\item Nodes, the hardware components that comprise each node of the network,
	\item Agents, the software controllers that control each node individually.
	\item Tasks, the actions that nodes should execute, in this case, returning a measurement in a geographic location.
	\item Resources, required by nodes to complete tasks.  
\end{itemize}
We then look at the roles different nodes play in completing a measurement task, and how energy is consumed in doing so. Next, we detail task coverage, and how this relates to the resilience of the network to disruption such through causes such as node failures. Finally, we define to what quality a measurement is completed, and how this, energy availability, and energy distribution in the system, effects the quality of overall composite task completion. 

\subsection{Wireless sensor network components and terminology}

A WSN system is comprised of a set of \textit{nodes}. Each node is equipped with a microcontroller for computation, a battery for power storage, a wireless transceiver for  transmitting and receiving messages from other nodes, and one or more \textit{sensors} for sensing and measuring some property of the environment such as temperature or radiation levels \citep{muhammad_r_ahmed_2012_1072589}. Each node has one \textit{agent}, a software controller that instructs its actions. Nodes may be deployed to precise locations or through a more random distribution method. The network structure formed may be flat or hierarchical, with or without clustering, dependent on the choice of protocol used \citep{Carlos-Mancilla2016b}. 


Requests for measurement from outside the WSN are received from a capable agent, a \textit{sink node}, which may then relay these requests through other agents until a final measurement is made by a \textit{sensing node}. This sequence of agents comprising the measurement task execution are defined as the \textit{task-path}. These concepts are illustrated in Figure \ref{fig:grid_concept} and will be detailed in the following sections.