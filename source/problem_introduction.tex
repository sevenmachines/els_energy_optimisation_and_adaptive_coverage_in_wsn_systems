
In this section we will formally set out the problem we look to solve. We define the base concepts of our WSN system in Sections \ref{section:terminology} and \ref{section:tasks}, looking at,
\reviewquestion{Opening list: For neatness of the sentence, please un-capitalise "Nodes", "Agents", "Tasks" and "Resources" and consistently end the first three items with commas or semi-colons (personally I'd use semi-colons as each item is a long clause).
}
\reviewquestionopen{Final sentence of intro: "in in"}
\begin{itemize}
	\item nodes, the hardware devices connected to the WSN;
	\item agents, the software controllers for each node;
	\item tasks, the actions that agents should execute, in this case, returning a measurement in a geographic location;
	\item resources, required by agents to complete tasks; 
\end{itemize}
In completing tasks, agents will be able to choose from a number of different actions, and play different roles in executing the task, which we look at in Sections \ref{section:roles} and \ref{section:actions}. In Section \ref{section:energy_consumption} we describe how energy is consumed by the system. Next, in Section \ref{section:task_quality}, we define to what quality a measurement is completed, and how this, energy availability, and energy distribution in the system, effects the quality of overall composite task completion. Finally, we detail task coverage in Section \ref{section:coverage}, and how this relates to the resilience of the network, and the effective lifetime of the system, giving us the problem definition in in Section \ref{section:optimisation_problem}.

\subsection{Wireless sensor network components and terminology}
\label{section:terminology}

A WSN system is comprised of a set of \textit{nodes}. Each node is equipped with a microcontroller for computation, a battery for power storage, a solar panel for recharging, a wireless transceiver for  transmitting and receiving messages from other nodes, and one or more \textit{sensors} for sensing and measuring some property of the environment such as temperature or radiation levels \citep{muhammad_r_ahmed_2012_1072589}. Each node has one \textit{agent}, a software controller that instructs its actions. Nodes may be deployed to precise locations or through a more random distribution method. The network structure formed may be flat or hierarchical, with or without clustering, dependent on the choice of protocol used \citep{Carlos-Mancilla2016b}. 

\reviewquestionopen{Reading through Section 3, it feels like there is something missing before Section 3.1 on the characteristics of the environment into which the WSN is deployed. There are hints of things given along the way, but no coherent list of assumptions. In particular, I'm thinking of things like the 'location grid' in Section 3.6 and the opening sentences of Section 3.7 on the relation between spatial distance and radiation levels (i.e. a gradual change over space).
}