\subsection{Coverage and resilience of routing}
%%%%%%%%%%%%%%%%%%%%%%%%%%%%%%
\newcommand{\functionSystemCoverage}[2]{
	\functionSignature{cov_{\setTime{}{}}}{\setCompositeTask{}{}, \setCompositeTask{}{*}}
}
%%%%%%%%%%%%%%%%%%%%%%%%%%%%%%%%%%%%%%%%%%
Since network impacts will mean that some nodes will not be able to be allocated an atomic task, temporarily or permanently, an agent trying to complete a composite task may not be able to do so. We can therefore use the proportion of atomic tasks of a given composite task that are completed successfully as a measure of how resilient the network is, how well it has adapted to impact and loss of nodes. For each completed composite task $\varCompositeTask{}{}$, we have a corresponding set of successfully completed atomic tasks, $\varCompositeTask{}{*} \subseteq \varCompositeTask{}{}$, so the \textit{coverage} of $\varCompositeTask{}{}$ is simply  $\funcSize{\varCompositeTask{}{*}}/\funcSize{\varCompositeTask{}{}}$. We can use this to measure the \textit{system coverage} over a time period $\setTime{}{}$, the average coverage of all composite tasks completed in during that period
\begin{equation}
	\functionSystemCoverage{}{} = \sum\limits_{\forall \varCompositeTask{}{} \in \setCompositeTask{}{}} \frac{ \funcSize{\varCompositeTask{}{*}}/\funcSize{\varCompositeTask{}{}}}{\funcSize{\setCompositeTask{}{}}{}}
\end{equation}

