\subsection{Task success, coverage, and network resilience}
%%%%%%%%%%%%%%%%%%%%%%%%%%%%%%

\newcommand{\varQualityMin}[2]{\epsilon}
\newcommand{\formalTasksSuccess}[2]{
	\functionFormal{success}
	{\setAtomicTask{}{}}
	{\setAtomicTask{}{'}}
}
\newcommand{\functionTasksSuccess}[2]{
\ifx&#1&%
	\functionSignature{success}{\setAtomicTask{}{}}
\else
	\functionSignature{success}{#1}
\fi
}

\newcommand{\formalCompositeTaskCoverage}[2]{
	\functionFormal{taskcov}
	{\setCompositeTask{}{}}
	{\setCompositeTask{}{'}}
}
\newcommand{\functionCompositeTaskCoverage}[2]{
	\functionSignature{taskcov}{\setCompositeTask{}{}}
}


\newcommand{\functionSystemCoverage}[2]{
	\functionSignature{cov_{\setTime{}{}}}{\setCompositeTask{}{}}
}

%%%%%%%%%%%%%%%%%%%%%%%%%%%%%%%%%%%%%%%%%%
In reality, when the quality of the completion of an atomic task falls below a certain level, the result may no longer be useful. Taking a measurement $1$ metre away from a demand point is likely to be close to the demand points' actual value, whereas $100$ metres away it may be completely uncorrelated. The distance at which this is likely to occur is highly dependent on the type of task performed, the environment it is in, and the desired system behaviour. To capture this, we set a value $\varQualityMin{}{}$ to be the quality threshold above which an atomic task is considered successful, defining the set of \textit{successful atomic tasks} as, $\formalTasksSuccess{}{}$ where $\functionAtomicTaskQualitySignature{}{} > \varQualityMin{}{}$ for all $\varAtomicTask{}{} \in \setCompositeTask{}{'}$.  As networks are impacted by node failures, communication problems, and weather effects, it may become complex to route all of the atomic tasks to nodes that can successfully complete them. We use the proportion of successfully completed atomic tasks as a measure of how adaptive the system is to these disruptions, increasing its resilience.  For this purpose, we define the \textit{system coverage} as the proportion of successfully completed atomic tasks in the set of completed composite tasks over a time period $\setTime{}{}$.
\begin{equation}
	\label{eq:coverage}
	\functionSystemCoverage{}{} = \sum\limits_{\forall \varCompositeTask{}{} \in \setCompositeTask{}{}} 
	\frac
	{\funcSize{\functionTasksSuccess{\varCompositeTask{}{}}{}}/\funcSize{\varCompositeTask{}{}}}
	{\funcSize{\setCompositeTask{}{}}{}}
\end{equation}