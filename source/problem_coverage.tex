The utility of the system can be increased by improving the quality of task completions, minimising energy consumption, and distributing energy usage more evenly across agents. In doing so, the necessary WSN properties of coverage, resilience and system lifetime \footnote{As discussed in Section \ref{section:background:requirements}.} should also increase. Using the concept of success of an atomic task, we now define how we can measure these within the system.

%%%%%%%%%%%%%%%%%%%%%%%%%%%%%%
\newcommand{\varQualityMin}[2]{qual_{min}}

\newcommand{\formalAtomicTaskSuccess}[2]{
	\functionFormal{success}
	{\setAtomicTask{}{}}
	{\setIntegersBinary{}{}}
}
\newcommand{\functionAtomicTaskSuccess}[2]{
	\functionSignature{success}{\varAtomicTask{#1}{#2}}
}

\newcommand{\formalCompositeTaskCoverage}[2]{
	\functionFormal{taskcov}
	{\setCompositeTask{}{}}
	{\setRealNumbersUnit{}{}}
}
\newcommand{\functionCompositeTaskCoverage}[2]{
	\functionSignature{taskcov}{\varCompositeTask{}{}}
}


\newcommand{\functionSystemCoverage}[2]{
	\functionSignature{syscov}{\setAtomicTaskInstance{}{}}
}
%%%%%%%%%%%%%%%%%%%%%%%%%%%%%%%%%%%%%%%%%%

\paragraph{Atomic task success}
\label{section:success}
In real-life scenarios, when the quality of an atomic task completion falls below a certain \textit{quality threshold}, $\varQualityMin{}{}$, the result may no longer be useful or relevant to the system; e.g. when an agent takes a sensor measurement far away from the corresponding tasks' demand point, or of such low quality that its masked by background variations in the environment.  We formalise this concept as the atomic task mapping $\formalAtomicTaskSuccess{}{}$ where;

\begin{equation}
	 \functionAtomicTaskSuccess{}{}
	 = 
	\begin{cases}
		1, & \text{if } \functionAtomicTaskQualitySensor{}{} > \varQualityMin{}{} \\
		0, & \text{otherwise}
	\end{cases}
\end{equation}

\paragraph{Coverage}
\label{section:coverage}
Given a set of completed atomic tasks, $\setAtomicTaskInstance{}{} \subseteq \setAtomicTask{}{}$, then the \textit{system coverage} of these tasks is the fraction of those tasks which were successful;
\begin{equation}
	\label{eq:coverage}
	\functionSystemCoverage{}{}
	=
	\frac{1}{\funcSize{\setAtomicTaskInstance{}{}}}
	\sum\limits_{\forall \varAtomicTask{}{} \in \setAtomicTaskInstance{}{}}
	\functionAtomicTaskSuccess{}{}
\end{equation}

%%%%%%%%%%%%%%%%%%%%%%%%%%
\newcommand{\functionSymbolResilence}[2]{\functionSymbol{resilience}{#1}{#2}}
\newcommand{\functionResilence}[2]{
	\functionSignature{\functionSymbolResilence{#1}{#2}}
	{\setAtomicTaskInstance{}{}, \setAgent{}{'}, \setAgent{}{} }
}
\paragraph{Resilience}
\label{section:resilience}
A systems' agents may become either permanently or temporarily unavailable through events such as component failures, communication problems, or weather disruption. In these circumstances, relaying atomic tasks to agents that can successfully complete them can be complex or even impossible.
 The ability to maintain coverage under these circumstances defines a systems' \textit{resilience}. Specifically, if a system completes a set of atomic tasks $\setAtomicTaskInstance{}{}$, with available agents $\setAgent{}{'} \subseteq \setAgent{}{}$, then the systems \textit{resilience} can be defined as; 
\begin{equation}
	\functionResilence{}{}
	= 
	\frac{
		\functionSystemCoverage{}{}
	}{
		\funcSize{\setAgents{}{'}} / \funcSize{\setAgents{}{}}
	}
\end{equation}
In this way, a system whose coverage remains high as the number of available agents falls has a higher resilience that a system whose coverage drops lower under the same circumstances.


%%%%%%%%%%%%%%%%%%%%%%%%%%%%%%%
 \newcommand{\varCoverageMinimum}[2]{\varSymbol{cov}{\textit{min}}{}}
 %%%%%%%%%%%%%%%%%%%%%%%%%%%%%%%
 \paragraph{Lifetime}
 \label{section:lifetime}
 Despite a systems resilience, given enough agents becoming permanently unavailable, some atomic tasks will become impossible to complete successfully, and there will be a deterioration of coverage with time. For a specific system, a minimum coverage value $\varCoverageMinimum{}{}$ for some set of atomic tasks $\setAtomicTaskInstance{}{}$ can be chosen below which the system is judged to be no longer useful. The \textit{lifetime} of the system can then be defined as the time until $\functionSystemCoverage{}{} < \varCoverageMinimum{}{}$.
 
 
\example{Success and coverage}{
	An ocean-monitoring system has a deployment area of $1\text{ km}^2$. An agent $\varAgent{}{}$ takes a salinity measurement $1$ metre away from a demand point of a task $\varAtomicTask{1}{}$, and $900$ metres from that of $\varAtomicTask{2}{}$. The result of $\varAtomicTask{1}{}$ is likely to be close to the actual value at the demand points' location, whereas that of $\varAtomicTask{2}{}$ is so far away as to uncorrelated, and not a practically useful measurement to the system. In this case, $\functionAtomicTaskSuccess{1}{} = 1$ and $\functionAtomicTaskSuccess{2}{} = 0$. So, with $\setAtomicTaskInstance{}{} = \lbrace \varAtomicTask{1}{}, \varAtomicTask{2}{} \rbrace$, the system coverage is $\functionSystemCoverage{}{} = \frac{1}{2}(1 + 0) = 0.5$  
}