\section{Related work}
\label{section:background}
\textit{Wireless sensor networks (WSNs)} are collections of independent battery-powered sensor nodes connected through wireless transmission, used to monitor a geographical area \citep{Akyildiz,Yick2008a}. In many circumstances the areas under study are difficult to access, such as remote locations, ocean-based studies, or contaminated areas. In these circumstances batteries are non-replaceable, and dispersal is ad-hoc, making a high level of autonomous node communication, coordination, and power consumption management, essential. WSNs can consist of static sensor deployments, or as mobile agents, adding to the need for adaptation in the network \citep{ramasamy2017mobile, 4224091} .

\textit{Environmental wireless sensor networks (E-WSNs)} in particular are characterised by intermittent connectivity between nodes, and harsh environmental conditions that lead to degradation of the network, so that the network must be adaptable in response to these changes \citep{Oliveira2011}. The sensors are usually small and inexpensive, capable of utilising limited compute and storage resources. They have one or multiple sensing devices to take measurements from the environment, ranging through chemical, optical, thermal, biological, and radioactivity detection. This collected data can then be transmitted to a base station and retransmitted to be stored remotely and analysed.

\todo[inline]{REFERENCES}
\begin{itemize}
	\item \cite{Sengupta2013}: Multi-objective node deployment in WSNs: In search of an optimal trade-off among coverage, lifetime, energy consumption, and connectivity 
	
	
	\item  \cite{doi:10.1155/2014/765182}: Performance Analysis of Resource-Aware Task Scheduling Methods in Wireless Sensor Networks

	
	\item \cite{Prauzek2018}: Energy harvesting sources, storage devices and system topologies for environmental wireless sensor networks: A review
	
	
	\item \cite{Albaladejo2010}: Wireless sensor networks for oceanographic monitoring: A systematic review
	
	\item \cite{10.1155/2018/8035065}: Prolonging the Lifetime of Wireless Sensor Networks: A Review of Current Techniques
	
\end{itemize}