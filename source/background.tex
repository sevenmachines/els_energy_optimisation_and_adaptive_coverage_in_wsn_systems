\section{Related work}
\label{section:background}

\ifdefined\DEBUG \todo[inline]{Why environmental WSN?} \else \fi
\textit{Wireless sensor networks (WSNs)} are collections of independent battery-powered sensor nodes connected through wireless transmission, used to monitor a geographical area \citep{Akyildiz,Yick2008a}. In many circumstances the areas under study are difficult to access, such as remote locations, ocean-based studies, or contaminated areas. \textit{Environmental wireless sensor networks (E-WSNs)} in particular are characterised by intermittent connectivity between nodes, and harsh environmental conditions that lead to degradation of the network, so that the network must be adaptable in response to these changes \citep{Oliveira2011}. In these circumstances batteries are non-replaceable, and dispersal is ad-hoc, making a high level of autonomous node communication, coordination, and power consumption management, essential. WSNs can consist of static sensor deployments, or as mobile agents, adding to the need for adaptation in the network \citep{ramasamy2017mobile, 4224091}. Sensors are usually small and inexpensive, capable of utilising limited compute and storage resources. They have one or multiple sensing devices to take measurements from the environment, ranging through chemical, optical, thermal, biological, and radioactivity detection. This collected data can then be transmitted to a base station and retransmitted to be stored remotely and analysed.

\ifdefined\DEBUG \todo[inline]{What are the characteristic challenges?} \else \fi
In realistic scenarios, there are five key elements that present challenges in deploying and operating a WSN system.
\begin{enumerate}
\item \textit{Energy consumption.} Each node in a WSN network has limited power available to function supplied by a battery. Depending on the environmental conditions, there may be some form of energy-capture component built into each node, such as solar-harvesting \citep{Prauzek2018}. However, even with this additional energy replenishment, energy is limited, and will eventually be exhausted. Batteries cannot be manually replaced in remote or inhospitable locations, that are often the focus of WSN deployments, so minimising energy consumption is essential \citep{Anastasi2009}. This can be done through the use of low-power components, \citep{4772585, 8108667}, as well as applying energy-aware routing protocols amongst nodes \citep{s90100445}. 

\item \textit{System lifetime.} Whether due to the impact of environmental degradation, power exhaustion, or connectivity loss, nodes in a network have a limited useful lifespan \citep{Mak2009}. As nodes are lost, the system itself becomes degraded, eventually unable to achieve its task to a high enough quality. At this point, the systems' lifetime has been reached and it is no longer useful. To extend this lifetime as far as possible we try to reduce the wear on nodes, principally, by ensuring that energy consumption is distributed through out the system evenly \citep{BABAYO20171176, Engmann2018}.

\item \textit{Quality of measurement data.} A node taking a reading may have a faulty sensor, leading to variations and lack of reliability. Sensors may get more accurate readings using more energy or longer time scales, for example, as the sampling time of a temperature or radiation sensor is increased, the more accurate the reading \citep{s17061221}. Therefore nodes must trade-off the quality of its data acquisition with the restrictive energy available to it across its lifetime \citep{7845391}.

\item \textit{Coverage of sensors.} In the majority of environmental situations, sensors are distributed in an ad-hoc manner, meaning their distribution is initially unknown amongst the nodes. Sensors may also have occlusion problems due to the topography of the environment or objects blocking connectivity or measurement \citep{10.1007/978-3-540-69170-9_23}. Therefore, to initialise the system, the deployed nodes must find which other nodes to communicate to that will allow all demand points to be covered. They must also be resilient to temporary or permanent outages on the network that means a new communication network must be discovered. 

\item \textit{Routing failure.} Wireless sensor networks add substantial additional risks to reliability over standard networking. Nodes are at risk of running out of power or of component failure, often exacerbated by harsh conditions. Environmental effects or obstacles may physical impact transmission or reception of signals. Loss of communication to a node is especially impactful as there are often multi-hop routes involved, which multiplies the risks \citep{Paradis2007}.
\end{enumerate}

\ifdefined\DEBUG \todo[inline]{How are learning techniques applied?} \else \fi
With the unknown nature of optimal network formation, task allocation, and resource allocation at initialisation, as well as its variability throughout the lifetime of the system, the use of learning strategies has had active research. Machine learning and reinforcement learning has been applied to many of the key requirements, routing in ad-hoc networks \citep{Nurmi}, allocating tasks such as sensor measurement \citep{doi:10.1155/2014/765182}, balancing the energy consumption \citep{10.1007/978-3-642-11814-2_4, PraveenKumar2019a}, and to the balancing of these multiple system objectives overall \citep{SENGUPTA2013405, s150717572}.

\ifdefined\DEBUG \todo[inline]{How this links to our problem definition} \else \fi
By setting the system objectives as the minimisation of energy consumption, and the maximisation of system lifetime, measurement quality, and sensor coverage, we can define the problem as a multi-objective learning problem in a distributed multi-agent system. In the next section we formalise the terminology, definitions, and the problem we look to address.


