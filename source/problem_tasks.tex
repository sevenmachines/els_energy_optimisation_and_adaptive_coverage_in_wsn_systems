\paragraph{System utility}
The quality of a composite task will not only be dependent on the value of its atomic tasks, but also on how well nodes in their respective task-paths have minimised energy usage, and energy distribution. Therefore we define the \textit{composite task quality} with these multiple objectives included as terms,
\begin{equation}
	\label{eq:ctv}
	\functionCompositeTaskQuality{}{} = 
	\sum\limits_{\forall \varAtomicTask{}{} \in \varCompositeTask{}{}}
	\big\lbrack
	\alpha\underbrace{\functionEnergyAvailable{\functionTaskArc{}{}}{}}_{\text{energy available}}
	+ \beta\underbrace{\functionEnergyVariability{\functionTaskArc{}{}}{}}_{\text{energy distribution}}
	+ 
	\gamma\underbrace{
		\functionComponentTaskValue{}{}
		\functionAtomicTaskQualitySignature{\varAtomicTask{}{}}{\functionDetectorRole{}{}}
	}_{\text{task quality}}
\big\rbrack
\end{equation}
where $\alpha$, $\beta$, and $\gamma$, are proportions chosen at system initialisation to weight the influence of energy, distribution, and task quality respectively. Each \textit{atomic task's absolute value} to the system will then be the product of the respective composite task's quality and the fractional contribution of the atomic task to that quality,
\begin{equation}
	\functionTaskAbsoluteValue{}{} = 
	\functionCompositeTaskQuality{}{}
	\functionAtomicTaskQualitySignature{}{}
\end{equation}

The overall \textit{ utility} of the system over a time period $\setTime{}{}$ will then be the sum of the composite task qualities of all the composite tasks $\setCompositeTask{}{}$ completed during that period.
	\begin{equation}
		\label{eq:system_utility}
		\functionSystemUtility{}{} = \sum\limits_{\varTime{}{} \in \setTime{}{}}
		\sum\limits_{\forall \varCompositeTask{}{} \in \setCompositeTask{\setTime{}{}}{}}
		\functionCompositeTaskQuality{}{}
	\end{equation}