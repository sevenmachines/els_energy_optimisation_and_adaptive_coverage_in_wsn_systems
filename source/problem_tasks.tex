\paragraph{Composite task quality}
\label{section:composite_task_quality}
As discussed, the value of atomic task $\varAtomicTask{}{}$ to a composite task is dependent on the resources the executing agent $\functionDetectorRole{}{}$ has assigned to the task, its distance away from the tasks demand point, as well as the significance of the atomic tasks results to the composite task overall. However, the quality of a composite task is not purely dependent on these values for all the atomic tasks in the corresponding composite task. For each atomic task there is a task-path that may contain other agents that have used their resources to help the detector receive and return the results of the task, $\functionRelayRole{}{}$. In receiving and transmitting, and taking other actions, these agents will have used some of their resources, in particular energy resources to relay the messages\footnote{As discussed in Section \ref{section:optimising_resource_usage}.}. To meet the system goals\footnote{See Section \ref{section:background:requirements}.}  the quality of a composite task will also be dependent on how well nodes in their respective task-paths have minimised energy usage, and energy distribution.  

We define the \textit{composite task quality} including these multiple objectives, with constants $\alpha$, $\beta$, and $\gamma$ chosen at system initialisation to weight the influence of energy, distribution, and task quality respectively, depending on the desired properties of the specific system; 
\begin{equation}
	\label{eq:ctv}
	\functionCompositeTaskQuality{}{} = 
	\sum\limits_{\forall \varAtomicTask{}{} \in \varCompositeTask{}{}}
	\big\lbrack
	\alpha\underbrace{\functionEnergyAvailable{\functionTaskArc{}{}}{}}_{\text{energy available}}
	+ \beta\underbrace{\functionEnergyVariability{\functionTaskArc{}{}}{}}_{\text{energy distribution}}
	+ 
	\gamma\underbrace{
		\functionComponentTaskValue{}{}
		\functionAtomicTaskQualitySignature{\varAtomicTask{}{}}{\functionDetectorRole{}{}}
	}_{\text{task quality}}
\big\rbrack
\end{equation}

Each \textit{atomic task's absolute value} to the system will then be the product of the respective composite task's quality and the fractional contribution of the atomic task to that quality. This value can be used by agents in a task-path to measure how useful their actions were in completing the atomic task overall, as judged by sinks that completed the associated composite task.
\begin{equation}
	\functionTaskAbsoluteValue{}{} = 
	\functionCompositeTaskQuality{}{}
	\functionAtomicTaskQualitySignature{}{}
\end{equation}

\paragraph{System utility}
\label{section:utility}
The overall \textit{ utility} of the system over a time period $\setTime{}{}$ will then be the sum of the composite task qualities of all the composite tasks $\setCompositeTask{}{}$ completed during that period.
	\begin{equation}
		\label{eq:system_utility}
		\functionSystemUtility{}{} = \sum\limits_{\varTime{}{} \in \setTime{}{}}
		\sum\limits_{\forall \varCompositeTask{}{} \in \setCompositeTask{\setTime{}{}}{}}
		\functionCompositeTaskQuality{}{}
	\end{equation}

\example{Defining the problem in an ocean-based WSN}{}