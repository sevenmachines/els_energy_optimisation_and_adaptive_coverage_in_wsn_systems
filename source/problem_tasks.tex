%%%%%%%%%%%%%%%%%
\newcommand{\functionAtomicTaskQualitySignature}[2]{
	\functionSignature{q_{\varAgent{}{}, \varTime{}{}}} {\varAtomicTask{}{}}
}
\newcommand{\functionCompositeTaskValue}[2]{
	\functionSignature{ctv}{\varAtomicTask{}{}}
}

\newcommand{\functionSystemUtility}[2]{\functionSignature{u}{\setTime{}{}}}
%%%%%%%%%%%%%%%%%%%
\newcommand{\functionCompositeTaskQuality}[2]{
	\functionSignature{taq_\varTime{}{}}{\varCompositeTask{}{}}
}
\newcommand{\functionTaskAbsoluteValue}[2]{
	\functionSignature{atv}{\varCompositeTask{}{}, \varAtomicTask{}{}}
}
%%%%%%%%%%%%%%%%%%%%%
\subsection{Task quality and system utility}

An agent that takes a measurement will complete that task with a certain \textit{atomic task quality}, a quality that is dependant on the resources it has dedicated to tasks of that type, and its distance to the tasks demand point. For example, a sensor is capable of taking a measurement of the radiation levels at the location of its sensor, which may be a distance away from the tasks' demand point. In addition, the longer the sample time the more energy is used, but the more accurate the reading will be. We therefore specify the atomic task quality of an atomic task $\varAtomicTask{}{}$ as a function of the distance of the task from the requested position, and the amount of energy used to make the measurement.
\begin{equation}
\label{eq:atomic_task_quality}
	\functionAtomicTaskQualitySignature{}{} = \functionTaskResourceAllocation{}{} \times \funcSize{
			\functionTaskDemandPoint{}{} - \functionDeployment{\varAgent{}{}}{}
	}{}
\end{equation}
The \textit{composite task value}, the proportional value of an atomic task $\varAtomicTask{}{}$ to the corresponding composite task $\varCompositeTask{}{}$, will not only be dependent on the quality of the atomic task, but also on additional system wide objectives. In a WSN system we want to maximise values for multiple objectives. Firstly, the energy available in the system, therefore minimising the overall battery power consumption. Secondly, the distribution of energy use across the nodes, which will spread component wear across nodes and so maximise the usable lifetime of the system. Lastly, the quality of atomic tasks, which not only improves the accuracy of readings, but will also invoke a penalty of poor task coverage as failed tasks have lower quality. Therefore the \textit{composite task value} of a composite task $\varCompositeTask{}{}$ is defined with these multiple objectives,
\begin{equation}
	\label{eq:ctv}
	\functionCompositeTaskValue{}{} = 
	\alpha\underbrace{\functionEnergyAvailable{\functionTaskArc{}{}}{}}_{\text{energy available}}
	+ \beta\underbrace{\functionEnergyVariability{\functionTaskArc{}{}}{}}_{\text{energy distribution}}
	+ 
	\gamma\underbrace{\functionAtomicTaskQualitySignature{}{}}_{\text{task quality}}
\end{equation}
Where $\alpha$, $\beta$, and $\gamma$ are variables chosen at system initialisation to weight the influence of energy, distribution, and task quality respectively. With the composite task value calculated after its completion  at a time $\varTime{}{}$, we can then calculate \textit{composite task quality} as the combination of each atomic tasks quality and its value to the composite task,
\begin{equation}
\functionCompositeTaskQuality{}{} = \sum\limits_{\forall \varAtomicTask{}{} \in \varCompositeTask{}{}}
\functionCompositeTaskValue{}{}\functionAtomicTaskQualitySignature{}{}	
\end{equation}
Finally we can define each atomic task absolute value to the system as the product of the respective composite tasks' quality and the relative contribution of the atomic task to that quality,
\begin{equation}
	\functionTaskAbsoluteValue{}{} = 
	\functionCompositeTaskQuality{}{}\functionCompositeTaskValue{}{}
\end{equation}

We so define the overall \textit{ utility} of the system over a time period $\setTime{}{}$ as the sum of the composite task qualities of all the composite tasks $\setCompositeTask{}{}$ completed during that period.
	\begin{equation}
		\label{eq:system_utility}
		\functionSystemUtility{}{} = \sum\limits_{\varTime{}{} \in \setTime{}{}}
		\sum\limits_{\forall \varCompositeTask{}{} \in \setCompositeTask{\setTime{}{}}{}}
		\functionCompositeTaskQuality{}{}
	\end{equation}